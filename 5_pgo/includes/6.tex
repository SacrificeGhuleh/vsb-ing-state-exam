%Základní metody úpravy a segmentace obrazu (filtrace, prahování, hrany).

\section{Prahování}
\begin{itemize}
    \item Cílem práhování je \textbf{oddělit pozadí od popředí} na základně stanoveného prahu (nějaká realná hodnota). Výsledkem binární obraz (1 = objekt, 0 = pozadí).
    \item Práh může být buď \textbf{stejný} pro celý obrázek, anebo \textbf{adaptivní} pro jednotlivé části obrazu. Další možností je stanovit práh v \textbf{intervalu} $<a, b>$. Úspěšnost detekci oblastí závisí na správné hodnotě prahu.
    \item Pokud neznáme hodnotu prahu, snažíme se jí stanovit na základně informací získaných z obrazu, který má být \textbf{segmentovaný}.
    \item \textbf{Bimodální histogra}m (dva kopce), \textbf{multimodální histogram }-- práh určit jako \textbf{minimum histogramu} mezi vysokými hodnotami, pak lze dále rekurzivně dělit (předpokládáme, že v obraze jsou převážně dva a více druhů pixelů).
    \item \textbf{Obraz lze rekurzivně dělit} na menší části, ve kterých se vypočte histogram a dle něho určí práh pro konkrétní část (pokud nelze práh určit, lze ho interpolovat pomocí sousedních prahů).
    \item \textbf{Minimalizace rizika chyby}:
          \begin{itemize}
              \item stanovení prahu tak, aby se minimalizovala špatná detekce,
              \item stanovení dle aproximace normálních rozdělení popředí a pozadí \\$\varepsilon = \theta P(t) + (1 - \theta)[1 - Q(t)]$.
              \item nejlepších výsledků lze dosáhnout v \textbf{extrému první derivace}
              \item pokud je zastoupení pozadí a popředí stejné a má stejný rozptyl $(t - \mu )^2 = (t - v)^2 $.
          \end{itemize}
          \begin{figure}[H]
              \centering
              \includegraphics[width=0.8\textwidth]{assets/8_prah_histogram}
          \end{figure}
    \item Na levém obrázku je prah označen $t$ a vyšrafovaná oblast značí chybu, která nastane při prahování, kdy bude špatně rozpoznané popředí/objekt $q(z)$ a pozadí $p(z)$ -- minimalizace chyby.
\end{itemize}


\section{Detekce hran}
\begin{itemize}
    \item Každá oblast je obklopena hranicí.
    \item Hranice se skládá z hran (případně také z jediné zakřivené hrany).

    \item Hrana se skládá z jednotlivých hranových bodů.
    \item Většinou se postupuje tak, že se obraz převede do stupně šedi a následně se naleznou jednotlivé body hran.
    \item Za bod hrany se často považuje místo, kde průběh jasu \textbf{vykazuje náhlou změnu}, případně \textbf{inflexní bod}.
    \item Po nalezení jsou jednotlivé nalezené body hran spojovány různými technikami do hran a celých hranic.
\end{itemize}
\subsection{Detekce hran s využitím gradientu}
\begin{itemize}
    \item Hrana je v obrazu zastoupeny (prudkou) \textbf{změnou jasu}, lze ji tedy najít zkoumáním síly a směru gradientu v jednotlivých bodech.
    \item Pro určení směru gradientu či hrany (​\textbf{směr gradientu je kolmý ke směru hrany​}) je třeba provést ​ \textbf{derivaci​} (nejlépe v x i y), která je při výpočtu nahrazena diferencí.
    \item Diference může být buď \textbf{centrální} nebo \textbf{dopředná}/\textbf{zpětná}.
          \begin{equation*}
              \begin{split}
                  d_x = \dfrac{I(x - 1, y) - I(x + 1, y)}{2} \, , \quad 		d_y= \dfrac{I(x, y - 1) - I(x, y + 1)}{2} \, .
              \end{split}
          \end{equation*}
    \item Velikost hrany lze určit velikostí gradientu (norma), hrana je tam, kde $e > \textrm{práh}$. (hrana je kolmá k gradientu) \\
          $ e(x, y) = \sqrt{(f_x(x,y)^2+ f_y(x,y)^2})$
    \item Směr hrany a gradientu lze určit (kde $\varphi$ -- směr gradientu, $\psi$ -- směr hrany)
          \begin{equation*}
              \begin{array}{c}
                  \varphi(x, y) = \arctan{[\frac{f_y(x,y)}{f_x(x,y)}]}, \psi(x,y) = \varphi(x,y) + \frac{\pi}{2}.
              \end{array}
          \end{equation*}
    \item Výše uvedené derivace lze nahradit \textbf{konvolučními maskami}
          \begin{itemize}
              \item \textbf{Sobel} -- vážený průměr (Prewittove dělá pouze normální)
              \item \textbf{Kirsch} -- počítání hran v 8 směrech
          \end{itemize}
    \item Robertsův operátor:
          \begin{equation*}
              \begin{bmatrix}
                  -1 & 0 \\[0.3em]
                  0  & 1
              \end{bmatrix}.
          \end{equation*}
    \item operátor Previttové:
          \begin{equation*}
              \begin{bmatrix}
                  -1 & -1 & -1 \\[0.3em]
                  0  & 0  & 0  \\[0.3em]
                  1  & 2  & 1  \\
              \end{bmatrix}.
          \end{equation*}
    \item Sobelův operátor:
          \begin{equation*}
              \begin{bmatrix}
                  -1 & -2 & -1 \\[0.3em]
                  0  & 0  & 0  \\[0.3em]
                  1  & 2  & 1  \\
              \end{bmatrix}.
          \end{equation*}
    \item Kirschův operátor:
          \begin{equation*}
              \begin{bmatrix}
                  -5 & -5 & -5 \\[0.3em]
                  3  & 0  & 3  \\[0.3em]
                  3  & 3  & 3  \\
              \end{bmatrix}.
          \end{equation*}
\end{itemize}
\begin{figure}[H]
    \begin{center}
        \includegraphics[width=0.25\textwidth]{assets/8_det_hran_grad}
        \caption{Velikost gradientu a jeho první a druhá derivace}
    \end{center}
\end{figure}
\subsection{Detekce hran hledáním průchodu nulou}
\begin{itemize}
    \item První derivace obrazové funkce nabývá svého maxima v místě hrany.
    \item \textbf{Druhá derivace protíná} v místě hrany \textbf{nulovou hodnotu}.
    \item Spolehlivější metoda, než hledání maxima v první derivaci. \textbf{NE} v případě, že je obraz postižen šumem. V tomto případě selhává, jelikož druhá derivace ještě více zesílí šum.
\end{itemize}
\subsection*{Laplaceův operátor (druhá derivace gradientu)}
\begin{itemize}
    \item Pro výpočet se používá symetrická diference nebo konvoluční masky (na krajích je maska ořezaná)
          \begin{equation*}
              \begin{split}
                  d_x &= I(x - 1, y) - 2I(x, y) + I(x + 1, y) \, ,\\
                  d_y&= I(x, y - 1) - 2I(x, y) + I(x, y + 1) \, .
              \end{split}
          \end{equation*}
          \begin{figure}[H]
              \begin{center}
                  \includegraphics[width=0.5\textwidth]{assets/8_priklady_laplace}
              \end{center}
          \end{figure}
    \item Hrana je detekována jako \textbf{změna znaménka v průchodu mezi dvěma extrémy}.
    \item Je \textbf{více citlivý na šum než první derivace} (i při malém šumu je detekováno množství falešných hran).
    \item Pro redukci šumu a zahlazení vysokých frekvencí lze použít \textbf{Gaussův operátor}:
          \begin{equation*}
              \begin{bmatrix}
                  1 & 2 & 1 \\[0.3em]
                  2 & 1 & 2 \\[0.3em]
                  1 & 2 & 1 \\
              \end{bmatrix}.
          \end{equation*}
\end{itemize}
\subsection{Cannyho detekce hran}
Canny první stanovil požadavky, které by měl detektor splňovat a následně navrhl detektor. %zabili kennyho, parchanti
Požadavky:
\begin{itemize}
    \item \textbf{Minimalizovat} pravděpodobnost \textbf{chybné detekce}.
    \item Najít polohu hrany, co \textbf{nejpřesněji}.
    \item Bod hrany identifikovat \textbf{jednoznačně}.
\end{itemize}
\textbf{Postup:}
\begin{enumerate}
    \item Eliminace šumu Gaussovým filtrem.
    \item Velikost a směr gradientu -- nejčastěji Sobelův operátor (nebo centrální derivace).
    \item Nalezení lokálních maxim a stanovení interpolace v osmi okolí. \textbf{Redukce na hranu velikosti 1 px}.
    \item Eliminace nevýznamných hran (\textbf{double thresholding})
          \begin{itemize}
              \item Všechny body, kde je velikost hrany $\leq t_{high}$ -- \uv{jistá} hrana
              \item Pak ty, které jsou $ > t_{low}$ a sousedí s hranou -- \uv{jistá} hrana
          \end{itemize}
\end{enumerate}

\section{Filtrace}
Rozlišujeme 2 základní druhy filtrů \textbf{rekurzivní} a nerekurzivní. Nerekurzivní i rekurzivní filtrace obrazu je lineární filtrace.

\subsection{Nerekurzivní filtry}
\begin{itemize}
    \item Vytváří výstupní signál v každém bodě jako \textbf{lineární kombinaci} vzorků vstupního signálu.
    \item Jsou to vlastně \textbf{konvoluční filtry}. Jedná se o úpravu signálu (prostorová doména) pomocí předpisu:
          \begin{equation*}
              g(m , n) = f(m, n) * h(m ,n),
          \end{equation*}
          kde $ f(m,n) $ je původní signál a $ h(m,n) $ je \textbf{impulsová charakteristika filtru}.
    \item Přepis při Fourierové transformaci -- \textbf{frekvenční doména}: $G(k ,l)=F(k ,l) \cdot H (k ,l)$.
    \item Tyto filtrace obrazu jsou \textbf{stabilní}.
    \item Gaussův filtr, uniformní, atd.
\end{itemize}

\subsection{Rekurzivní filtry}
\begin{itemize}
    \item U rekurzivní filtrace je výstupní signál $ g $ \textbf{svázán se vstupním signálem} $ f $ prostřednictvím vztahu:
          \begin{equation*}
              b(m,n)*g(m,n)=a(m ,n)*f (m,n),
          \end{equation*}
          kde $ a(m,n) $ a $ b(m,n) $ jsou \textbf{diskrétní funkce}, které popisují rekurzivní filtr.
    \item U rekurzivní filtrace obrazu \textbf{bývá nižší časová náročnost} než u nerekurzivní filtrace, ale \textbf{nebývá vždy stabilní}, proto je potřeba stabilitu sledovat.
    \item Když je filtr nestabilní, mohou se šířit čím dál vyšší zaokrouhlovací \textbf{chyby} a \textbf{šumy}.
    \item S iteračním výpočtem můžeme využít následujícího předpisu:
          \begin{equation*}
              gi (m,n)=a(m,n)*f (m ,n)+c(m,n)*gi−1 (m ,n),
          \end{equation*}
          kde: $ c (0,0)=0 $ a $ c (m ,n)=−b(m ,n) $.
\end{itemize}

\subsection{Inverzní filtr}
\begin{itemize}
    \item Jedná se prakticky o pouhou \textbf{dekonvoluci}. Máme-li rozmazaný (symetricky, v daném směru, \ldots) obraz bez šumu, pak lze obraz rekonstruovat zpět do podoby před rozmazáním.
    \item Nevhodné pokud byl obraz postižen šumem, \textbf{šum se inverzním filtrem zvýrazní}.
    \item Inverzní filtr \textbf{lze použít i pro detekci šumu}. Pracuje ve \textbf{frekvenční doméně} (po DFT).
    \item Pokud je signál zkreslen signálem $ H(x) $ pak stačí jen hodnotu převrátit a získáme původní hodnotu $ F(x): F (x)= \frac{1}{H(x)} $.
    \item Obnovený obraz inverzním filtrem má rozmazané původně ostré hrany.
\end{itemize}

\begin{figure}[H]
    \centering
    \includegraphics[width=0.8\textwidth]{assets/8_inv_filtr}
\end{figure}

\subsection{Wienerův filtr}
\begin{itemize}
    \item Jedná se v podstatě o inverzní filtr, který lze použít i na \textbf{zašumělý obraz}.
    \item Slouží pro rekonstrukci/opravu obrazu, který je poničen šumem, který \textbf{má odhadnutelné statistické vlastnosti} (tedy známe přibližně šum nebo konvoluci, kterou byl obraz poničen).
    \item Většinou neznáme cílový obraz abychom si mohli stanovit střední hodnotu.
    \item Úloha se vyjádří jako \textbf{optimalizace řešením předurčené soustavy lineárních stochastických rovnic} $\rightarrow$ \textbf{minimalizace chyby}.
          \begin{equation*}
              e^2 = \epsilon \{(f(i, j) - \hat{f}(i, j)^2 )\},
          \end{equation*}
          kde $\epsilon$ označuje operátor střední hodnoty.
    \item \textbf{Praktické použití}: Hubbelův teleskop (v době kdy měl poškozené zrcadlo), Rozpoznání automobilových značek (rozmazání pohybu -- známá rychlost).
\end{itemize}

\begin{figure}[H]
    \centering
    \includegraphics[width=0.7\textwidth]{assets/8_wiener}
\end{figure}