\documentclass[11pt]{article}

% Packages
\usepackage[czech]{babel}
\usepackage[utf8]{inputenc}
\usepackage[useregional]{datetime2}
\usepackage[T1]{fontenc}
\usepackage[a4paper, total={15.24cm, 23.32cm}]{geometry}
\usepackage[thinlines]{easytable}
\usepackage{graphicx}
\usepackage[ampersand]{easylist}
\usepackage{changepage}
\usepackage{float}
\usepackage{color}
\usepackage{xcolor}
\usepackage{enumitem}
\usepackage{latexsym}
\usepackage{hyperref}
\usepackage{amssymb}
\usepackage{amsmath}
\usepackage{amsfonts}
\usepackage{minted}
\usepackage{caption}
\usepackage{amsfonts}
\usepackage{makecell}
\usepackage{tikz}
\usepackage{enumitem}
\usepackage{multirow}

% Config
\renewcommand{\baselinestretch}{1.2} 
\setitemize{itemsep=0pt}
\setenumerate{itemsep=0pt}
\hypersetup{
  colorlinks,
  citecolor=black,
  filecolor=black,
  linkcolor=black,
  urlcolor=black
}
\title{\vspace{-10ex}\textbf{I. Matematické základy informatiky}}
\date{\small\vspace{-9ex}Update: \today}
\setminted{fontsize=\small,baselinestretch=1}
\newcommand{\hhline}{\Xhline{2\arrayrulewidth}}
\newcommand{\resline}{\vspace{1.5mm}\hrule\vspace{1.5mm}}
% \DeclareUnicodeCharacter{2212}{-}
\begin{document}
\maketitle
\setcounter{tocdepth}{1}
\tableofcontents{\small\vspace{-15ex}}
\pagebreak
\section{Konečné automaty, regulární výrazy, uzávěrové vlastnosti třídy regulárních jazyků.}
\section{Architektura mikroprocesorů}
\textbf{Architektura procesoru} je náčrt struktury a funkčnosti systému. Je charakterizována výčtem \textbf{registrů} a jejich funkcí, vnitřních a vnějších \textbf{sběrnic}, způsobem \textbf{adresování} a \textbf{instrukčním souborem}.

\textbf{Registr} je malé úložiště dat v mikroprocesoru s rychlým přístupem, které slouží jako \textbf{pracovní paměť} během výpočtů.

\textbf{Sběrnice} je soustava vodičů pro \textbf{přenos informací} mezi více účastníky na principu \uv{jeden vysílá, ostatní přijímájí.} Podle typu přenášené informace je dělíme na \textit{datové}, \textit{adresové} a \textit{řídící}. V praxi však díky multiplexu může jít o jedny dráty.

\section{Procesory CISC a RISC}
V dnešní době se ustálilo dělení počítačů do dvou základních kategorií podle typu používaného procesoru:
\begin{itemize}
    \item{\textbf{CISC} -- počítač se složitým souborem instrukcí (\textit{Complex Instruction Set Computer})}
    \item{\textbf{RISC} -- počítač s redukovaným souborem instrukcí (\textit{Reduced Instruction Set Computer})}
\end{itemize}

\subsection{CISC}
\begin{itemize}
    \item Procesory s \textbf{komplexním instrukčním souborem}.
    \item Instrukce mají \textbf{proměnlivou délku} i \textbf{dobu vykonání}.
    \item \textbf{Vysoká složitost instrukcí} $\,\to\,$ nutný systematický návrh řadiče procesoru.
    \item Vykonání strojové instrukce probíhá posloupností mikrooperací (předepsána mikroinstrukcí v řídící paměti).
    \item Procesor obsahuje relativně \textbf{nízký počet registrů}.
    \item Operace provedená i \textbf{složenou instrukcí} (např. násobení) může být \textbf{nahrazena} sledem jednodušších strojových instrukcí (\textit{sčítání a bitové posuny}) $\,\to\,$ mohou být ve výsledu vykonány rychleji, než hardwarově implementovaná složená varianta.
    \item {Označení \textbf{CISC} bylo zavedeno jako \textbf{protiklad} až poté, co se prosadily procesory RISC, které mají instruční sadu naopak maximálně redukovanou (pouze jednoduché operace, tj. žádné složené, jsou stejně dlouhé a jejich vykonání trvá stejnou dobu).}
    \item {Obvyklou chybou je domněnka, že procesory CISC mají více strojových instrukcí, než procesory RISC. Ve skutečnosti nejde o absolutní počet, ale o \textbf{počet různých druhů operací}, které procesor sám přímo umí vykonat na hardwarové úrovni (tj. již z výroby). Procesor CISC tak může například paradoxně obsahovat jen jednu strojovou instrukci pro danou operaci (např. \textit{logické operace}), zatímco procesor RISC může tuto operaci obsahovat jako několik strojových instrukcí, které stejnou operaci umí provést nad různými registry.}
\end{itemize}
\begin{figure}[H]
    \centering
    \includegraphics[width=0.6\textwidth]{assets/1_cisc_sekv}
    \includegraphics[width=0.6\textwidth]{assets/1_cisc_instrukce}
\end{figure}

\subsection{RISC}
\begin{itemize}
    \item {\textbf{Počet instrukcí a způsobů adresování je malý, ale zůstává úplný}, aby bylo možno provést vše $\,\to\,$ v tomhle se liší od CISC}.
    \item {Instrukce jsou vytvořeny pomocí obvodu $\,\to\,$ jednodušší na výrobu než CISC}.
    \item Je menší počet instrukcí, takže složitější instrukce se nahradí větším počtem jednodušších.
    \item To způsobuje nárust kódu. Zároveň ale vznikly rychlejší sběrnice, tj rychlejší proud dat do procesoru.
    \item {Používá se \textbf{zřetězené zpracování instrukcí} (Blíže popsáno níže)}.
    \item {Instrukce se provádějí jen nad registry}.
    \item {Navýšený počet registrů $\,\to\,$ delší program}.
    \item {Instrukce mají \textbf{jednotný formát} -- délku i obsah}.
    \item {Komunikace s pamětí pouze pomocí instrukcí \textbf{LOAD / STORE}, adresování i práce je celkově rychlejší}.
    \item V návaznosti je využívaná cache pro co nejrychlejší přístup dat.
    \item Využívá se \textbf{predikce skoků}, takže se začnou načítat data, která pravděpodobně budou v další instrukci potřeba.
    \item {\textbf{Každý strojový cyklus znamená dokončení jedné instrukce}}.
    \item {Řešení problémů s frontou instrukcí}.
    \item {Mikroprogramový řadič může být nahrazen rychlejším obvodem}.
    \item {Představitelé \textbf{ARM, MOTOROLA 6800, INTEL i960, MIPS R6000}}.
\end{itemize}
\begin{figure}[H]
    \centering
    \includegraphics[width=0.6\textwidth]{assets/1_risc_zretezeni}
\end{figure}

\section{Von Neumannovo schéma počítače}

John Von Neumann definoval v roce \textbf{1945} základní koncepci počítače (EDVAC) \textbf{řízeného obsahem paměti}. Od té doby se objevilo několik odlišných modifikací, ale v podstatě se \textbf{počítače v dnešní době} konstruují podle tohoto modelu. Ve svém projektu si von Neumann stanovil určitá kritéria a principy, které musí počítač splňovat, aby byl použitelný univerzálně. Můžeme je ve stručnosti shrnout do následujících bodů:
\begin{itemize}
    \item Počítač se skládá z paměti, řídící jednotky, aritmeticko--logické jednotky, vstupní a  výstupní jednotky.
          \begin{enumerate}
              \item \textbf{ALU} -- aritmeticko-logická jednotka (aritmetic-logic unit) => jednotka provádějící veškeré aritmetické výpočty a logické operace. Obsahuje sčítačky, násobičky a komparátory.
              \item  \textbf{Operační paměť} -- slouží k uchování zpracovávaného programu, zpracovávaných dat a výsledků výpočtu.
              \item \textbf{Řídící jednotka} -- řídí činnost všech částí počítače. Toto řízení je prováděno pomocí řídících signálů, které jsou zasílány jednotlivým modulům. Řadiči jsou pak zpět zasílané stavové hlášení. Dnes řadič spolu s ALU tvoří jednu součástku, a to procesor neboli CPU (Central Processing Unit).
              \item \textbf{Vstup/Výstup} -- zařízení určené pro vstup dat, a výstup zpracovaných výsledků.
          \end{enumerate}
    \item Struktura pc je \textbf{nezávislá na typu řešené úlohy} (univerzálnost), \textbf{počítač se programuje obsahem paměti}.
    \item Následující krok počítače je závislý na kroku předešlém.
    \item \textbf{Instrukce} a \textbf{data} jsou v téže paměti.
    \item Paměť je rozdělena do \textbf{paměťových buněk stejné velikosti (Byte)}, jejichž pořadová čísla se využívají jako adresy.
    \item Program je tvořen posloupností instrukcí, které se vykonávají jednotlivě v pořadí, v jakém jsou zapsány do paměti.
    \item Změna pořadí prováděných instrukcí se provádí \textbf{skokovými instrukcemi} (podmíněné nebo nepodmíněné skákání na adresy).
    \item Čísla, instrukce, adresy a znaky se značí v \textbf{binární soustavě}.
\end{itemize}
\begin{figure}[H]
    \centering
    \includegraphics[width=0.6\textwidth]{assets/1_vonNeumann}
\end{figure}

\subsection{Nevýhody Von Neumannovy koncepce ve srovnání s dnešními PC}
\begin{itemize}
    \item Podle von Neumannova schématu počítač pracuje \textbf{vždy nad jedním programem}. Toto vede k velmi špatnému využití strojového času. Dnes je obvyklé, že počítač \textbf{zpracovává paralelně více programů} zároveň - tzv. \textbf{multitasking}.
    \item Počítač může mít i více jak jeden procesor.
    \item Podle Von Neumanova schématu mohl počítač pracovat pouze v tzv. \textbf{diskrétním režimu}, kdy byl do paměti počítače zaveden program, data a pak probíhal výpočet. V průběhu výpočtu již nebylo možné s počítačem dále interaktivně komunikovat.
    \item Dnes existují \textbf{vstupní/výstupní} zařízení, např. pevné disky a páskové mechaniky, které umožňují vstup i výstup.
    \item Program se do paměti nemusí zavést celý, ale je možné zavést pouze jeho část a ostatní části zavádět až v případě potřeby.
\end{itemize}

\noindent\makebox[\textwidth]{\includegraphics[width=8cm]{assets/1_neuman2}}

\subsection*{Výhody}
\begin{itemize}
    \item[$+$] \textbf{Rozdělení paměti} pro kód a data určuje programátor, řídící jednotka přistupuje pro  data i instrukce jednotným způsobem.
    \item[$+$] \textbf{Jedna sběrnice} ->  jednodušší levnější výroba.
\end{itemize}
\subsection*{Nevýhody}
\begin{itemize}
    \item[$-$] \textbf{Společné uložení dat a kódu} může mít za následek přepsání vlastního programu.

    \item[$-$] \textbf{Jedna sběrnice} je omezující.
\end{itemize}

\section{Hardvardské schéma počítače}
Několik let po von Neumannovi, přišel vývojový tým odborníků z Harvardské univerzity s vlastní koncepcí počítače, která se sice od Neumannovy příliš nelišila, ale odstraňovala některé její nedostatky. V podstatě jde pouze o \textbf{oddělení paměti pro data a program}. Abychom si mohli obě koncepce porovnat, můžeme vycházet ze zjednodušených schémat.
\newline

\noindent\makebox[\textwidth]{\includegraphics[width=8cm]{assets/1_harvard}}

\subsection*{Výhody}
\begin{itemize}
    \item[$+$]\textbf{Program se nepřepíše} (oddělené paměti pro data a program).
    \item[$+$]Dvě sběrnice umožňují \textbf{paralelní} načítání instrukcí a dat.
    \item[$+$]Paměti mohou být vyrobeny \textbf{odlišnými technologiemi }a každá může mít jinou nejmenší adresovací jednotku (8 bitů pro instrukce a 8, 16 nebo 32 pro data).
\end{itemize}


\subsection*{Nevýhody}
\begin{itemize}
    \item[$-$]2 sběrnice mají \textbf{vyšší nároky na vývoj} řídící jednotky a jsou také dražší a složitější na výrobu.
    \item[$-$]Paměť je \textbf{rozdělena} už od \textbf{výrobce}.
    \item[$-$]Nevyužitou část dat \textbf{nelze využít }po program a obráceně.
\end{itemize}

\pagebreak
\section{Principy urychlování činnosti procesorů}
\begin{itemize}
    \item{Speciální kódování dle potřeby dané úlohy}.
    \item{Speciální výpočetní jednotky dle potřeby dané úlohy (FFT -- rychlá fourierova transformace)}.
    \item{Paralelní zpracování (násobné výpočetní jednotky)}.
    \item{\textbf{Zřetězové zpracování instrukcí} (pipelining)}.
    \item{Využití cache pamětí (L1, L2, L3)}.
    \item{Predikce skoků}
\end{itemize}
\subsection{Paralelní zpracování}
Zpracování více elementárních úloh běží součastně.
\begin{figure}[H]
    \centering
    \includegraphics[width=0.6\textwidth]{assets/1_paralelni_zpracovani.png}
\end{figure}

\subsection{Zřetězené zpracování instrukcí (pipelining)}
Princip zřetězení se značně překrývá s principy procesorů RISC.
%\section{Problémy zřetězeného zpracování}
Základní myšlenkou je \textbf{rozdělení zpracování jedné instrukce} mezi různé části procesoru a tím i dosažení možnosti \textbf{zpracovávat} \textbf{více instrukcí }najednou. Pro dosažení tohoto zřetězení je nutné rozdělit úlohu do posloupnosti dílčích úloh, z nichž každá může být vykonána \textbf{samostatně}, např. oddělit načítaní a ukládání dat z paměti od provádění výpočtu instrukce a tyto části pak mohou běžet souběžně. To znamená že musíme osamostatnit jednotlivé části sekvenčního obvodu tak, aby každému obvodu odpovídala jedna fáze zpracování instrukcí. Všechny fáze musí být \textbf{stejně časově náročné}, jinak je rychlost \textbf{degradována} na nejpomalejší z nich. Fáze zpracování je rozdělena minimálně na 2 úseky:
\begin{itemize}
    \item \textbf{Načtení} a \textbf{dekódování} instrukce.
    \item \textbf{Provedení} instrukce a případné uložení výsledku.
\end{itemize}
Zřetězení se stále vylepšuje a u novějších procesorů se již můžeme setkat stále s více řetězci rozpracovaných informací (více pipelines), dnes je standardem 5 pipelines.

\subsection{Problém a predikce skoků}
Největší problém spočívá v \textbf{plnění zřetězené jednotky}, hlavně při provádění \textbf{podmíněných skoků}, kdy během stejného počtu cyklů se vykoná více instrukcí. U pipelingu se instrukce následující po skoku vyzvedává dřív, než je skok dokončen. \textbf{Primitivní implementace} vyzvedává vždy \textbf{následující instrukci}, což vede k tomu, že se vždy mýlí, pokud je skok nepodmíněný. Pozdější implementace mají \textbf{jednotku předpovídání skoku (1bit)}, která vždy správně \textbf{předpoví nepodmíněný skok} a s použitím cache se záznamem předchozího chování programu se pokusí předpovědět i cíl podmíněných skoků nebo skoků s adresou v registru nebo paměť. V případě, že se predikce nepovede, bývá nutné vyprázdnit celou pipeline a začít vyzvedávat instrukce ze správné adresy, což znamená relativně \textbf{velké zdržení}. Související problémem je přerušení.

\subsection{Plnění fronty instrukcí}
Pokud se dokončí skoková instrukce, která odkazuje na jinou část kódu, musejí být instrukce za ní zahozeny (\textit{problém plnění fronty instrukcí}).
\begin{itemize}
    \item{U malého zřetězení \textbf{neřešíme}}.
    \item{Používání bublin na vyprázdnění pipeline, \textbf{naplněníní prázdnými instrukcemi}}.
    \item{\textbf{Predikce skoku} -- vyhrazen jeden bit předurčující, zda se skok provede či nikoliv}.
\end{itemize}
\begin{itemize}
    \item{\textbf{Statická} -- součást instrukce $\,\to\,$ řeší programátor nebo kompilátor}
    \item{\textbf{Dynamická}
          \begin{itemize}
              \item{\textbf{jednobitová} -- zaznamenává jestli se skok provedl, či ne (1/ 0)}
              \item{\textbf{dvoubitová} -- metoda zpožděného skoku $\,\to\,$ v procesoru řeší se např. tabulkou s 4 kB instrukcí}
          \end{itemize}
          }
\end{itemize}
Zřetězené zpracování přináší urychlení výpočtu nejen v procesorech, ale i jiných číslicových obvodech (např. pro zpracování obrazu, bioinformatických dat apod.). Pokud použijeme zřetězené zpracování, musíme dodat řadu podpůrných obvodů a řešit řadu nových problémů. \textbf{Moderní procesory používají kromě zřetězení i další koncepty}:
\begin{itemize}
    \item{\textbf{Superskalární architektura} (zdvojení) -- když nastane podmíněný skok, začnou se vykonávat instrukce obou variant, nepotřebná část se pak zahodí. Tento způsob, pak vyžaduje vyřešit ukládání výsledku.}
    \item{\textbf{VLIW procesory} (Very long instruction word) -- umožňuje instrukční paralelismus (vykonávání několik nezávislých instrukcí souběžně), velmi dlouhé instrukční pakety}.
    \item{\textbf{Vektorové procesory} -- je navržený tak, aby dokázal vykonávat matematické operace nad celou množinou čísel v daném čase. Je opakem skalárního procesoru, který vykonává jednu operaci s jedním číslem v daném čase. }
    \item{\textbf{Vícevláknové procesory} }
\end{itemize}


\pagebreak
\section{Bezkontextové gramatiky a jazyky. Zásobníkové automaty, jejich vztah k bezkontextovým gramatikám.}
\section{Relační datový model}
Relační datový model představuje \textbf{způsob uchování dat v tabulkách}. Relační se mu říká proto, jelikož tabulka je definována přes Relaci.

\begin{figure}[H]
    \centering
    \includegraphics[width=0.35\textwidth]{assets/tab_relace.png}
    \includegraphics[width=0.35\textwidth]{assets/relace.png}
    \caption{Tabulka s dvěma atributy jako relace (vlevo), relace zobrazena tabulkou (vpravo).}
\end{figure}

\textbf{Relace} je tabulka definována jako \textbf{podmnožina kartézského součinu domén}. Relace na obrázku je tedy podmnožina kartézského součinu množin \{Dudak, Novák, Dvořák\} $\times$ \{Milan, Martin, Jan, ..., Aleš\}.

Na rozdíl od matematické relace se ta databázová \textbf{mění v čase} (přidáváním a odebíráním prvků relace). Kromě základních \textbf{množinových operací} se u databázové relace setkáme s operaci \textbf{selekce} -- výběr řádků a \textbf{projekce} -- výběr sloupců.

\begin{itemize}
    \item \textbf{Doména} je \textbf{množina všech hodnot, kterých může daný atribut nabývat} (obor hodnot atributu). V praxi je doména dána\textbf{ integritním omezením} (IO). Doména atributu Přijmení z obrázků je množina \{Dudak, Novák, Dvořák\}.
    \item \textbf{Atribut} je vlastnost entity (z pohledu tabulky jde o sloupec).
    \item \textbf{Relační schéma} můžeme chápat jako strukturu tabulky (atributy a domény).
          Relační schéma R je výraz tvaru R(A, f),  kde  R  je jméno schématu, A = {A1, A2,..., An} je \textbf{konečná množina jmen atributů}, f je zobrazení přiřazující každému jménu atributu Ai neprázdnou množinu (obor hodnot atributu), kterou nazýváme \textbf{doménou atributu} Di, tedy \textbf{f(Ai) = Di}.
\end{itemize}

\subsection*{Příklad pro tabulku (relaci) Učitel}
\begin{itemize}
    \item \textbf{Atributy:} \texttt{ID, jméno, příjmení, funkce, kancelář}.
    \item \textbf{Domény:}
          \begin{itemize}
              \item D1 -- tři písmena z příjmení, tří cifry pořadového čísla,
              \item D2 -- kalendář jmen,
              \item D3 -- množina příjmení,
              \item D4 -- množina funkcí (asistent, vědec, učitel,...),
              \item D5 -- A101, A102, ... A160.
          \end{itemize}
    \item \textbf{Relační schéma:} \texttt{Učitel (ID, jméno, příjmení, funkce, kancelář)}.
    \item \textbf{Relace:} \texttt{Učitel = \{(nov001, lukas , novak , vědec, A135),\\ (kom123, jan, komensky, učitel, A111), ...\}}
\end{itemize}

\subsection{Základní úlohy relačního modelu:}
\begin{enumerate}
    \item Návrh „správné“ \textbf{struktury databáze bez redundancí} -- \textbf{funkční závislosti}, \textbf{normální formy}.
    \item \textbf{Vyhledávání informací} z databáze -- (dotazovací) \textbf{relační jazyky}.
\end{enumerate}

\subsection{Vlastnosti relačního datového modelu}
Z definice relace vyplývají tyto jejich tabulkové vlastnosti:
\begin{itemize}
    \item \textbf{Homogenita} (stejnorodost) sloupců (prvky domény).
    \item Každý údaj (hodnota atributu ve sloupci) je \textbf{atomickou položkou}.
    \item Na \textbf{pořadí} řádků a sloupců \textbf{nezáleží} (jsou to množiny prvků/atributů).
    \item Každý řádek tabulky je \textbf{jednoznačně identifikovatelný} hodnotami jednoho nebo několika atributů (primárního klíče).
\end{itemize}

\subsection{Vazby relačního modelu}
Obecně se vazby v relačním modelu realizují pomocí další relace (tabulky). Jedná se o tzv. \textbf{vazební tabulku}. Ta obsahuje ty atributy relací (tabulek, které se vazby účastní), které jednoznačné identifikují jejich entity -- primární klíče. Obsahuje-li tabulka atribut, který slouží jako primární klíč v jiné tabulce, pak obsahuje cizí klíč. Vazební tabulka tedy obsahuje cizí klíče. Příklad vazby M:N:

\begin{figure}[H]
    \centering
    \includegraphics[width=0.5\textwidth]{assets/mn.png}
\end{figure}

\section{SQL (Structured Query Language)}
SQL (Structured Query Language) \textbf{je relační jazyk založen na predikátovém kalkulu}. Na rozdíl od jazyků založených na relační algebře, kde se dotaz zadává algoritmem, tyto jazyky se soustředí na to \textbf{co se má hledat}, ne jak.

\begin{itemize}
    \item Standardizovaný\textbf{ strukturovaný dotazovací jazyk}, který je používán pro práci s daty v \textbf{relačních databázích}. (DQL - Data Query Language).
    \item Navržen IBM jako \textbf{dotazovací jazyk} (původní název Sequel).
    \item Základem je \textbf{n-ticový relační kalkul}.
    \item Standardy podporuje prakticky každá relační databáze, ale obvykle nejsou implementovány vždy\textbf{ všechny požadavky normy}.
    \item Obsahuje i příkazy pro \textbf{vytvoření} a \textbf{modifikace} tabulek, pro \textbf{ukládání}, \textbf{modifikaci} a \textbf{rušení} dat v databázi a řadu dalších příkazů.
    \item \textbf{Příklad}: \texttt{CREATE TABLE Drazitel (jmeno CHAR(20), adresa CHAR(30), aukce NUMBER(4), zisk NUMBER(4)); INSERT INTO  clovek VALUES( 'nj001', 'Jan', 'Novotný', '777111222'); SELECT telefon FROM clovek WHERE prijmeni = "Novotný";}
\end{itemize}

\subsection{DML - Data manipulation language}
\begin{itemize}
    \item \textbf{Modifikací} dat --  \texttt{INSERT}, \texttt{UPDATE}, \texttt{DELETE} = vlož, uprav, smaž.
    \item \textbf{Vyhledávání} v relacích --  \texttt{SELECT}, \texttt{ORDER BY},  \texttt{GROUP BY}, \texttt{JOIN} = vyhledej, seřaď, shlukuj, spoj.
    \item Další, pro podmínky, logické operatory, ... (\texttt{WHERE, LIKE, BETWEEN, IN, IS NULL, DISTINCT/UNIQUE, JOIN, INNER JOIN, OUTER JOIN, EXISTS, HAVING, COUNT, VIEW, INDEX}, ...).
\end{itemize}

\subsection{DDL - Data definition language}
\begin{itemize}
    \item \textbf{Vytváření} a \textbf{modifikace} relačního schematu (tabulek, databází) - \texttt{CREATE}, \texttt{ALTER} (\texttt{MODIFY}, \texttt{ADD}), \texttt{DROP}  = vytvoř, uprav, smaž.
\end{itemize}

\subsection{DCL - Data control language}
\begin{itemize}
    \item \textbf{Správa práv} - příkazy jako \texttt{GRANT}, \texttt{REVOKE}.
\end{itemize}

\subsection{TCL - Transaction control language - transakce}
\begin{itemize}
    \item \texttt{COMMIT} - úspěšně provedená transakce, vše se uloží.
    \item \texttt{ROLLBACK} - zrušení všech změn celé transakce.
    \item \texttt{SAVEPOINT} - uložení bodu, ke kterému lze provést \texttt{ROLLBACK}. Tj zruší se jen část transakce a může se pokračovat jinou větví celé transakce dále. Lze tak transakce dělit na menší atomické části.
\end{itemize}

\section{Relační jazyky}
Jazyky pro formulaci požadavků na výběr dat z relační databáze (dotazovací jazyky) se dělí do dvou skupin:
\begin{itemize}
    \item \textbf{Jazyky založené na  relační algebře}, kde jsou výběrové požadavky vyjádřeny jako posloupnost speciálních operací prováděných nad daty. Dotaz je tedy  \textbf{zadán algoritmem}, jak vyhledat požadované informace.
    \item \textbf{Jazyky založené na  predikátovém kalkulu}, které požadavky na výběr zadávají jako predikát charakterizující \textbf{vybranou relaci}. Je úlohou překladače jazyka nalézt odpovídající algoritmus. Tyto jazyky se dále dělí na
          \begin{itemize}
              \item \textbf{n-ticové} relační kalkuly,
              \item \textbf{doménové} relační kalkuly.
          \end{itemize}
\end{itemize}

\section{Relační algebra}
Relační algebra je velmi silný \textbf{dotazovací jazyk} vysoké úrovně. Nepracuje s jednotlivými enticemi relací, ale \textbf{s celými relacemi}. Operátory relační algebry se aplikují na relace, výsledkem jsou opět relace. Protože relace jsou množiny, přirozenými prostředky pro manipulaci s relacemi budou množinové operace.

I když relační algebra v této podobě \textbf{není vždy implementována v jazycích SŘBD}, je její zvládnutí nutnou podmínkou pro správnost manipulací s relacemi. I složitější dotazy jazyka SQL, který je deskriptivním dotazovacím jazykem, mohou být bez zkušeností s relační algebrou problematické.

\subsection{Základní operace relační algebry}
Jsou dány relace \textbf{R} a \textbf{S}. \textbf{Množinové operace:}
\begin{itemize}
    \item \textbf{Sjednocení} relací téhož stupně:      $R \cup S = \{x | x \in R \vee x \in S\}$
    \item \textbf{Průnik} relací:                                 $R \cap S = \{x | x \in R \wedge x \in S\}$
    \item \textbf{Rozdíl} relací:                                   $R  -  S = \{x | x \in R \wedge x \notin S\}$
    \item \textbf{Kartézský součin} relace R stupně m a relace S stupně n: $R x S = \{rs | r \in R \wedge s \in S\}$,  kde  $rs = \{r1,...,rm,s1,...sn\}$
\end{itemize}
\textbf{Další relační operace: }
\begin{itemize}
    \item \textbf{Projekce} (výběr atributů) relace R, jedná se o unární operaci $\Pi_\mathbf{X}( R )$, kde X je množina názvů atributů.
    \item \textbf{Selekce} (výběr řádků) z relace R podle podmínky P. Selekce je unární relační operace $\sigma_{\varphi(\mathbf{X})}( R )$, kde R je relace, $\varphi(\mathbf{X})$ predikátová formule hovořící o jednotlivých prvcích a jejich příslušnosti do relací.
    \item \textbf{Spojení} relací R s atributy A  a  S  s atributy  B (join).  Značí se $R \bowtie S$, výsledkem je množina všech kombinací prvků relace \textbf{R} a \textbf{S}. Takto definovaný join se nazývá Přirozené spojení (natural join). Exsitují i další (outer, inner, left, right \ldots).
\end{itemize}

Příklad: $\Pi_\textrm{název} \sigma_{\varphi(\textrm{pohlaví=žena})} (\textrm{Úkol}\bowtie\textrm{Pracuje}\bowtie\textrm{Zaměstnanec})$


\section{N-ticový relační kalkul}
\begin{itemize}
    \item Dr. Codd definoval n-ticový relační kalkul pro RDM jazyk matematické logiky - predikátový počet je využit pro výběr informací z relační databáze.
    \item Název odvozen z oboru hodnot jeho proměnných - \textbf{relace je množina prvků = n-tic}.
    \item Je \textbf{základem pro jazyk typu SQL}.
    \item Syntaxe je \textbf{přizpůsobena} programovacímu jazyku: \textbf{matematické vyjádření} $\{ x | F(x) \}$ nahradíme zápisem \textbf{\texttt{x WHERE  F(x)}}
          \begin{itemize}
              \item Kde x je proměnná pro hledané n-tice (struktura relace).
              \item F(x) je \textbf{podmínka}, kterou má x splňovat (výběr prvků relace).
          \end{itemize}
\end{itemize}

\subsection{Definice}
Výraz n-ticového relačního kalkulu je výraz tvaru \textbf{\texttt{x WHERE F(x)}}, kde x je jediná volná proměnná ve formuli F. Základní operace relační algebry se dají vyjádřit pomocí výrazů n-ticového relačního kalkulu, tedy n-ticový relační kalkul je relačně úplný.

\noindent\makebox[\textwidth]{\includegraphics[width=12cm]{assets/ntic}}


\section{Funkční závislost}
Funkční závislost je v databázi \textbf{vztah mezi atributy} takový, že máme-li atribut Y je funkčně závislý na atributu X píšeme X → Y, pak se \textbf{nemůže stát}, aby \textbf{dva řádky mající stejnou} hodnotu atributu \textbf{X} měly \textbf{různou hodnotu Y}. Je-li Y, X říkáme, že závislost X → Y je \textbf{triviální}.
\begin{itemize}
    \item FZ je definována \textbf{mezi dvěma podmnožinami atributů} v rámci jednoho schématu relace. Jde o vztah mezi atributy, nikoliv mezi entitami.
    \item FZ je definována na \textbf{základě všech možných aktuálních relací}, není tedy možné soudit na funkční závislost z vlastností jediné relace. Tak můžeme poznat jen neplatnost funkční závislosti.
    \item FZ jsou \textbf{tvrzení o reálném světě}, o významu atributů nebo \textbf{vztahů mezi entitami}, je nutné realitu brát v úvahu při návrhu schématu databáze.
\end{itemize}

\textbf{Příklad:} Atribut '\textit{datum narození }' je funkčně závislý na atributu '\textit{rodné číslo}' (nemůže se stát, že u záznamů se stejnými rodnými čísly bude různé datum narození).

Pomocí funkčních závislostí můžeme\textbf{ automaticky navrhnout schéma databáze} a předejít problémům jako je \textbf{redundance}, \textbf{nekonzistence databáze}, zablokování při vkládání záznamů, apod.

\section{Armstrongovy axiomy}
K určení \textbf{klíče schématu} a logických implikací množiny závislostí potřebujeme \textbf{nalézt uzávěr F+}, nebo určit, zda daná závislost X → Y je prvkem F+.  K tomu existují pravidla zvaná Armstrongovy axiomy. Jsou \textbf{úplná} (dovolují odvodit z dané množiny závislostí F všechny závislosti patřící do F+) a \textbf{bezesporná} (dovolují z F odvodit pouze závislosti patřící do F+).
\begin{itemize}
    \item \textbf{Reflexivita} -- je-li Y $\subset$ X $\subset$ A, pak X → Y
    \item \textbf{Tranzitivita} -- pokud je X → Y a Y → Z, pak X → Z
    \item \textbf{Pseudotranzitivita} --  pokud je X → Y a WY → Z, pak XW → Z
    \item \textbf{Sjednocení} -- pokud je X → Y a X → Z, pak X  → YZ
    \item \textbf{Dekompozice} -- pokud je X → YZ, pak  X  → Y a X → Z
    \item \textbf{Rozšíření} --  pokud je X → Y a  Z $\subset$ A, pak  XZ  → YZ
    \item \textbf{Zúžení} --  pokud je X → Y a  Z  $\subset$ Y, pak  X → Z
\end{itemize}
Závislost, která má na pravé straně pouze jeden atribut, nazýváme \textbf{elementární}.

\subsection{Určení klíče pomoci funkčních závislostí}
Ze zadání jsme určili atributy A = \{učitel, jméno, příjmení, email, předmět, název, kredity, místnost, čas\} a funkční závislosti F:
\begin{itemize}
    \item učitel → jméno, příjmení, email
    \item předmět → název, kredity
    \item místnost, čas → učitel, předmět
\end{itemize}

\noindent\textbf{Rozšíření:}
\begin{itemize}
    \item učitel, \textbf{místnost}, \textbf{čas} → jméno, příjmení, email, \textbf{místnost}, \textbf{čas}
    \item předmět → název, kredity
    \item místnost, čas → učitel, předmět
\end{itemize}

\noindent\textbf{Dekompozice 1:}
\begin{itemize}
    \item učitel, \textbf{místnost}, \textbf{čas} → jméno, příjmení, email, místnost, čas, \textbf{učitel}, \textbf{předmět}
    \item předmět → název, kredity
\end{itemize}

\noindent\textbf{Dekompozice 2:}
\begin{itemize}
    \item učitel, místnost, čas → jméno, příjmení, email, místnost, čas, učitel, předmět, \textbf{název}, \textbf{kredity}
\end{itemize}

Atributy \textbf{učitel}, \textbf{místnost}, \textbf{čas} je klíč schématu velké relace. V dalším kroku je třeba provést dekompozici a tuto velkou relaci rozbít na menší relace.

\section{Dekompozice}
Dekompozice relačního schématu je \textbf{rozklad relačního schématu na menší} relač. sch. (rozloží velkou tabulku na menší) aniž by došlo k narušení redundance databáze. Mezi základní vlastnosti dekompozice patří - \textbf{zachování informace} a \textbf{zachování funkčních závislostí}.
\begin{itemize}
    \item \textbf{Algoritmus dekompozice (metoda shora dolů)} -- na počátku máme celé relační schéma se všemi atributy, snažíme se od tohoto schématu odebírat funkční závislosti a tvořit schémata nová. \textbf{Exponenciální složitost}, \textbf{BCNF}.
    \item \textbf{Algoritmus syntézy (zdola nahoru)} -- vytvoří pro každou funkční závislost novou relaci. Pak tyto malé relace spojuje do větších celků. \textbf{Menší složitost, 3NF}.
\end{itemize}
\textbf{Binární dekompozice}, kterou budeme dále řešit je rozklad jednoho relačního schématu na dvě. Obecná dekompozice vznikne postupnou aplikací binárních. Dekompozice relačního schématu R(A,f) je množina relačních RO=\{R1(A1, f2), R2(A2, f2), ...\}, kde A = A1 $\cup$ A2 $\cup$ A3 $\cup$ ...

\section{Normální formy}
Normální formy relací (NF) prozrazují jak dobře je databáze navržena (čím vyšší NF tím lepší).
\begin{itemize}
    \item \textbf{0 NF} -- Pokud nesplňuje ani 1 NF, je v 0 NF
    \item \textbf{1 NF} -- definuje tabulky, které obsahují \textbf{pouze atomické atributy}. Žádné složené atributy - např. v jednom atributu je Jméno i Příjmení.
    \item \textbf{2 NF} -- je v 1NF + \textbf{každý sekundární atribut je úplně závislý na každém klíči schématu}. Neboli neexistuje závislost sekundárních na podklíči (pokud se klíč skládá z více atributů). Např.: když AB → CD, pak nesmí být B → C. Atribut adresa není závislý na všech klíčích FZ, ale pouze na F.
          \begin{figure}[H]
              \centering
              \includegraphics[width=0.3\textwidth]{assets/2nf.png}
          \end{figure}
    \item \textbf{3 NF} -- je 2NF + žádný sekundární atribut \textbf{není tranzitivně závislý} na žádném klíči schématu. Nesmí existovat závislosti mezi sekundárními atributy (Model auta -> značka auta). Když AB → CD, pak nesmí C  → D. \textbf{Příklad porušení 3NF} -- atribut počet obyvatel je tranzitivně závislý (přes atr. město) na klíči.
          \begin{figure}[H]
              \centering
              \includegraphics[width=0.3\textwidth]{assets/3nf.png}
          \end{figure}
    \item \textbf{BCNF} (Boyce-Coddova normální forma) -- 3NF + je-li funkční závislost (X → Y) $\in$ F+ a Y $\notin$ X, pak X obsahuje klíč schématu. \textbf{Musí být závislost sekundárních atributů na primárních nikoli naopak}. Když AB → CD, pak nesmí C  → A.

          Často pokud je splněna 3NF je zároveň splněna i BCNF. Pro nesplnění BCNF je nutné: Aby relace měla více kandidátních klíčů, alespoň 2 z nich musí být složené z více atributů a některé složené klíče musí mít společný atribut.

          Příklad relace: PSČ, město, ulice. Toto je validní dle 3NF, ale ne BCNF. Kandidátní klíče jsou tedy PSČ-město a město-ulice. Město je v obou, překrývá se, tudíž není BCNF ale jen 3NF.
\end{itemize}

\pagebreak
\section{Matematické modely algoritmů - Turingovy stroje a stroje RAM. Složitost algoritmu, asymptotické odhady. Algoritmicky nerozhodnutelné problémy.}
\begin{itemize}
    \item Syntetizace fotorealistických obrazů je oblastí PG, která dovoluje vykreslit jakoukoliv uměle vytvořenou scénu tak, jak by vypadala v reálném světě.
    \item Toho dosahuje díky implementaci optických zákonů, které lze běžně pozorovat.
\end{itemize}

\section{Sledování paprsku - ray tracing}
\begin{itemize}
    \item Metoda sleduje šíření paprsků ve scéně.
    \item Tyto paprsky začínají \textbf{ve světelném zdroji}, \textbf{odráží} se o tělesa v prostoru a některé z nich nakonec dopadnou do průmětny (obdobně jako světlo v reálném světě).
    \item Paprsky, které takto prochází scénu, lze znázornit jako strom.
    \item Tento přístup je však \textbf{neefektivní}, protože \textbf{velká část paprsků do průmětny nikdy nedopadne}, takže nemají přínos pro výsledný obraz a zbytečně zvyšují výpočetní čas.
\end{itemize}
\begin{figure}[H]
    \centering
    \includegraphics[width=0.4\textwidth]{assets/6_ray_trace_svetlo}
\end{figure}

\section{Zpětné sledování paprsku}
\begin{itemize}
    \item Funguje stejně jako běžné sledování paprsku, ovšem paprsky jsou \textbf{vysílány z kamery do scény}.
    \item Jakmile paprsky dosáhnou \textbf{ukončovacího kritéria} (jdou mimo scénu/maximální počet odrazů), sledujeme zpět jejich pohyb (navrácení z rekurze) a vypočítáváme osvětlení.
    \item Tím se eliminuje možnost, že by paprsek nepřinesl žádný přínos výslednému obrazu a značně se tak urychluje celý proces ray tracingu.
\end{itemize}
\begin{figure}[H]
    \centering
    \includegraphics[width=0.4\textwidth]{assets/6_ray_trace_kamera}
\end{figure}

\section{Rekurzivní sledování paprsku}
\begin{itemize}
    \item Metoda vyšetřuje \uv{běh} světelných paprsků ve scéně.
    \item Světlo je reprezentováno \textbf{paprsky}, které jsou do scény vyzařovány světelnými zdroji a \textbf{putují prostorem} scény, některé dopadnou na povrchy těles, jiné odletí ze scény.
    \item Paprsek, který dopadne na povrch tělesa se může \textbf{odrazit} (zákon odrazu) nebo pokud je těleso průhledné, může se paprsek \textbf{zlomit} (zákon lomu) -- oba druhy paprsků mohou opět dopadnout na povrch těles, kde se celý proces znovu opakuje. %bitč
    \item Do scény se vyšle \textbf{velké množství paprsků}, ale podstatné jsou ty, které projdou objektivem myšlené kamery, pokud na průmětnu dopadne dostatečný počet paprsků, vykreslí se obrázek.
    \item Metoda je sice jasná a fyzikálně podložená, ale nepoužívá se, protože se obtížně realizuje. (Je potřeba vyslat velké množství paprsků, ale k objektivu kamery by jich dorazilo jen malé množství a ostatní by se sledovaly zbytečně).
    \item Řešením je \textbf{otočit paprsky a vyslat je od kamery ke světelnému zdroji}. Principiálně to pak funguje stejně.
          % \item Rekurzivní sledování paprsku funguje na tom principu, že jakmile primární paprsek (z kamery do průmětny) dopadne na nějaký objekt, dojde v tomto místě k jeho odrazu/lomu, ověří se, zda je bod dopadu osvětlen a vypočte lokální osvětlení (Phongův osvětlovací model).
          % \item Na základně možného odrazu a lomu se do scény vysílají sekundární paprsky, které zapříčiní potřebnou rekurzi.
          % \item Tato rekurze umožní trasovat např. průhlednost.
          % \item Zároveň však prodlužuje výpočetní nároky.
          % \item Hloubka rekurze se zpravidla nastavuje napevno.
          \begin{figure}[H]
              \centering
              \includegraphics[width=0.4\textwidth]{assets/6_rekurzivni_sledovani_paprsku}
          \end{figure}
    \item Rovnice výpočtu lokálního osvětlení: $I = I_l + k_rI_r + k_tI_t$ \\
          $I_l = I_aO_a + \sum\limits_{i} S_if_{att_i}I_i(O_d \cos{\varphi_i} + O_s \cos^n{\alpha_i} )$.
          Hodnota Si představuje viditelnost i-tého zdroje světla v daném bodě.
\end{itemize}
\section{Urychlování trasování}
Největším problémem je hledání průsečíků paprsků s objekty scény.
\begin{enumerate}
    \item Nejjednodušším řešením je využít \textbf{ohraničujících ploch} (\uv{bounding boxů}). Ohraničující plocha se vytvoří kolem každého objektu ve scéně. Nejlépe když má plocha následující vlastnosti
          \begin{itemize}
              \item objekt leží \textbf{celý uvnitř} ohraničující plochy, ale plocha jej obepíná co \textbf{nejtěšněji},
              \item průsečíky paprsků s plochou musí jít spočítat, co nejjednodušším výpočtem,
              \item plochu musí být možné pro jednotlivé objekty dostatečně jednoduše nalézt.
          \end{itemize}
          Je možné použít \textbf{kulovou plochu} (ne moc vhodné, objekty můžou být protáhlé a tato plocha by je neobepínala dostatešně těšně). Další variantou plochy je \textbf{kvádr} (taky sice není moc vhodný, protože těleso může být našikmo a taky by jej neobepínal moc natěsno, nicméně nalezení ohraničující plochy je snadné -- minimální a maximální hodnoty obepínaného tělesa).
          \begin{figure}[H]
              \centering
              \includegraphics[width=0.2\textwidth]{assets/6_boudingbox}
          \end{figure}
          Princip ohranučujících ploch spočívá v tom, že pokud paprsek neprotne ohraničující plochu, pak neprotne ani těleso uvnitř (velmi časté). Odhalením této situace dojde ke značnému zrychlení. Pokud je to naopak, hledají se průsečíky s ohraničeným tělesem.
    \item Rozšířením předchozího je \textbf{organizování ohraničujících ploch do hierarchických struktur}. Princip je stejný, pokud se neprotne rodičovská plocha, nehledají se dále ani průsečíky s potomky. Nevýhodou je, že nelze jednoduše takovouto strukturu automatizovaně nalézt.
          \begin{figure}[H]
              \centering
              \includegraphics[width=0.4\textwidth]{assets/6_organizovani_do_struktur}
          \end{figure}
    \item Další metodou je \textbf{Dělení prostoru scény na podprostory}. Obykle se dělí rovinami souřadné soustavy $xy, xz, yz \,\to\,$ vznikají tak velké kvádry (stejně velké / různě velké). Princip metody:
          \begin{itemize}
              \item u neurychlené metody byly všechny objekty organizovány v 1 velkém seznamu,
              \item nyní je zřízeno tolik seznamů, kolik je objemových elementů vzniklých dělením prostoru, každý element bude mít svůj seznam objektů, které do něj aspoň z části zasahují (pokud objekt zasahuje do více elementů, bude v seznamu každého z nich) -- hledání průsečíků začíná v tom elementu, kde je počátek paprsku, při opouštění elementu lze zjistit, do kterého elementu vstupuje, paprsek kontroluje pouze průsečíky s objekty, které jsou v seznamu daného elementu.
          \end{itemize}
          \begin{figure}[H]
              \centering
              \includegraphics[width=0.3\textwidth]{assets/6_deleni_podprostor}
          \end{figure}
    \item Dalším metodou je \textbf{Adaptivní hloubka rekurze}. Odhaduje se, zda je paprsek pro stanovení intenzity ve zkoumaném obrazovém bodě dostatečně užitečný. Pokud ne, tak se nevyšle (např. u odrazů či průchodů tělesy).
          % \item Protože je ray tracing výpočetně náročný, je vhodné jej urychlit.
          % \item Nejjednodušší způsob je použití obalových struktur – bounding boxů.
          % \item Ty mají nejčastěji tvar koule nebo osově zarovnaného kvádru (AABB).
          % \item S jejich pomocí můžeme zjistit, zda se paprsek pohybuje v okolí objektu (tj. protíná strukturu).
          % \item Pouze v případě, že ano, zjišťujeme jeho průsečíky s objektem.
          % \item Vhodná obalová struktura má následující vlastnosti:
          %   \begin{itemize}
          %    \item Objekt v ní vždy leží celý, ale struktura se jej snaží co nejvíce obepnout.
          %    \item Co nejjednodušší výpočet pro ověření průsečíků s paprskem.
          %    \item Dostatečně jednoduchá konstrukce.
          %  \end{itemize}
\end{enumerate}
\section{Vyzařovací metoda - radiozita}
Na rozdíl od předchozí metody rekurzivního sledování paprsku (dobře zobrazuje lesklé, dobře osvícený předměty) je tato metoda spíše protikladná. Zaměřuje se na \textbf{difúzní odrazy světla} -- vhodná pro matné povrchy a rozptýlené světlo (např. interiéry). Princip:
\begin{itemize}
    \item Vypočítá se, jak jsou osvětlena jednotlivá místa scény.
    \item Podle toho se povrchy těles pokryjí sítí (v místech kde je komplikovaný průběh osvětlení je síť hustá -- zlom světla a stínu).
    \item Pro každou plošku jsou spočítany hodnoty RGB -- vyzařování je konstantní na celém povrchu plošky.
          \begin{itemize}
              \item \textbf{PROBLÉM:} kdyby se takto plošky zobrazovaly, mohly by se sousedící plošky výrazně lišit intenzitou a nebylo by to pěkné.
              \item   \textbf{ŘEŠENÍ:} po výpočtu intenzit plošek se intenzity přenesou do jednotlivých uzlů sítě (zprůměrování intezit okolních plošek, které obklopují uzel) a následně se intenzity interpolují. (proto i to hustší dělení, kde je přechod světlo--stín ...).
          \end{itemize}
    \item Nejjednodušším, ale ne zrovna nejsprávnějším zobrazením scény a interpolací je pomocí \textbf{Gouradova stínování}.
          \begin{itemize}
              \item   \textbf{PROBLÉM:} osvětlení bylo spočítáno v prostoru scény a tam by se měla provádět i interpolace, ale Gouraudovo stínování interpoluje v prostoru obrazu. Problém je, že při středové projekci se nezachovává dělící poměr a proto budou výsledky v prostoru obrazu rozdílné od výsledků z prostoru scény. Nicméně Gouraudovo stínování se používá, protože je rychlé.
          \end{itemize}
    \item Vytváření sítě probíhá v několika iteracích: nejdřív se hustota odhadne, pak se spočítá osvětlení a dle výsledků se síť dohustí tam, kde je třeba. Základní myšlenou je, že na všech ploškách ustanov \textbf{energetická rovnováha}: \textbf{ výkon vyzařovaný + výkon absorvovaný = výkon na plošku dopadající od jiných ploch + výkon, který ploška sama vyzařuje}
\end{itemize}
\begin{equation*}
    B_i = E_i + p_i \sum\limits_{i=i}^n B_j F_{j \,\to\, i} \frac{A_j}{A_i}.
\end{equation*}
kde:
\begin{itemize}
    \item $B_j$ -- výkon vyzářený ploškou $j$,
    \item $p_i$ -- míra odrazu (optické vlastnosti materiálu),
    \item $E_i$ -- hodnota výkonu vlastního vyzařování plošky,
    \item $A_i$,$A_j$ -- velikost plošek (plošný obsah),
    \item $F_{j \,\to\, i}$ -- konfigurační koeficient říká, jaká část výkonu vyzářeného ploškou $j$ dopadne na plošku $i$ (jedná se o $\int \langle0,1\rangle$, záleží na pořadí indexů).
\end{itemize}
\begin{figure}[H]
    \centering
    \includegraphics[width=0.3\textwidth]{assets/6_vyzarovaci_metoda}
\end{figure}

\section{BRDF (Bidirectional Reflectance Distribution Function)}
\begin{itemize}
    \item Charakterizuje \textbf{odrazové schopnosti povrchu materiálu} v určitém bodě $\mathbf{x}$.
    \item Jedná se o \textbf{poměr odraženého zářezí} ke vstupnímu diferenciálnímu zařízení, promítnutému na kolmou plochu.
    \item BRDF v daném bodě zůstává stejná i když změníme směr paprsku.
    \item \textbf{Pozitivita BRDF}: funkce není nikdy záporná.
    \item \textbf{Zákon zachování energie}: plocha nemůže odrazit víc než je celková přijatá energie
    \item \textbf{Odrazivost} $p(x) = \frac{d\Phi_r(x)}{d\Phi_i(x)}; \, d\Phi_r(x)$ je \textbf{odražený} světelný tok, $d\Phi_i(x)$ je \textbf{dopadající} světelný tok.
    \item Obor hodnot odrazivosti je na intervalu $<0,1>, \, 1=$ \textbf{plný odraz}.
    \item \textbf{PRINCIP:} Vysílá se mnoho paprsků v každém bodě s různými offsety, některé padnou do zdroje světla, některé ne. Výsledná hodnota pixelu je poté průměrem všech hodnot paprsků (čím více paprsků vyšleme, čím lepší je výsledek (méně zrní)).
\end{itemize}
\begin{figure}[H]
    \centering
    \includegraphics[width=0.3\textwidth]{assets/6_brdf}
\end{figure}
\section{BTDF (Bidirectional Transmittance Distribution Function)}
Dvousměrná distribuční fuknce lomu. Popisuje \textbf{průchod světla povrchem}.

\section{BSDF (Bidirectional Scattering Distribution Function)}
\begin{itemize}
    \item Obousměrná distribuční funkce \textbf{rozptylu}.
    \item Je to souhrn dvou distribučních fukncí, a to funkce odrazu (BRDF) a lomu (BTDF).
    \item \textbf{BSDF + BTDF + BRDF}
\end{itemize}
\begin{figure}[H]
    \centering
    \includegraphics[width=0.3\textwidth]{assets/6_bsdf}
\end{figure}


\section{Renderovací rovnice}
Rekurzivní diferenciální rovnice
\begin{equation*}
    L(x, \omega_0) = L_e(x,\omega_0) + \int_{\Omega} L(r(x,\omega_i) - \omega_i) \cdot BRDF(\omega_i,x,\omega_0) \cos{\theta_id\omega_i}
\end{equation*}
Zjednodušeně: \textbf{osvětlení povrchu = samovolně vyzařované světlo + součet příchozího osvěrlení ze všech směrů krát BRDF}.

\pagebreak
\section{Třídy složitosti problémů. Třída PTIME a NPTIME, NP-úplné problémy.}
\section{Model ISO/OSI}
Počítačové sítě vyvíjelo více firem, zpočátku to byly uzavřené a nekompatibilní systémy. Hlavním účelem sítí je však vzájemné propojování, a tak vyvstala potřeba stanovit pravidla pro přenos dat v sítích a mezi nimi. \textbf{Mezinárodní ústav pro normalizaci ISO} (International Standards Organization) vypracoval tzv. referenční model \textbf{OSI (Open Systems Interconnection)}, který rozdělil práci v síti do \textbf{7 vzájemně spolupracujících vrstev}.

Princip spočívá v tom, že vyšší vrstva převezme úkol od podřízené vrstvy, zpracuje jej a předá vrstvě nadřízené. Vertikální spolupráce mezi vrstvami (nadřízená s podřízenou) je \textbf{věcí výrobce} sítě. Model \textbf{ISO/OSI doporučuje}, jak mají vrstvy \textbf{spolupracovat horizontálně} – dvě stejné vrstvy modelu mezi různými sítěmi (či síťové prvky různých výrobců). Model je důležitý především pro výrobce síťových komponent. V praktické práci se sítí jej moc nevyužijeme. Umožňuje však pochopit principy práce síťových prvků a zároveň patří k základní terminologii sítí.

Každá vrstva je samostatná \textbf{entita}, jejíž výstup je \textbf{PDU} -- protocol data unit. Těmito PDU komunikují entity mezi sebou. Každá další entita zapouzdří (přidá svou hlavičku) entitu z předchozí vrstvy a pošle dále. Vrstva 7--4 je v koncových zařízeních, vrstva 3--1 je v síťově orientačních zařízeních.

\begin{itemize}
    \item\textbf{7. Aplikační vrstva }- je \textbf{určitou aplikací} (např. oknem v programu) zpřístupňující uživatelům síťové služby. Nabízí a zajišťuje přístup k souborům (na jiných počítačích), vzdálený přístup k tiskárnám, správu sítě, elektronické zprávy (včetně e-mailu). Protokoly \textbf{HTTP}, \textbf{FTP}, \textbf{POP3} apod.... PDU této vrsty se nazývá \textbf{zpráva}.
    \item\textbf{6. Prezentační vrstva} - má na starosti \textbf{konverzi dat}, přenášená data mohou totiž být v různých sítích různě kódována. Tato vrstva zajišťuje sjednocení formy vzájemně přenášených údajů. Dále data komprimuje, případně šifruje. V praxi často splývá s relační vrstvou.
    \item \textbf{5. Relační vrstva} - \textbf{navazuje} a po skončení přenosu \textbf{ukončuje} \textbf{spojení}. Může provádět \textbf{ověřování} uživatelů, \textbf{zabezpečení} přístupu k zařízením.
    \item\textbf{4. Transportní vrstva} - vrstva\textbf{ zajišťuje přenos dat mezi koncovými uzly}. Jejím účelem je poskytnout takovou kvalitu přenosu, jakou požadují vyšší vrstvy. Vrstva nabízí spojově (TCP) a nespojově orientované (UDP) protokoly. (Platí pouze pro \textbf{TCP/IP}). PDU této vrsty se nazývá \textbf{segment}.
    \item\textbf{3. Síťová vrstva }- je zodpovědná za spojení a \textbf{směrování mezi dvěma počítači nebo celými sítěmi} (tj. uzly), mezi nimiž neexistuje přímé spojení. Stará se o síťové adresování. Na této vrstvě pracuje router. PDU u této vrstvy se nazývá \textbf{paket}. Protokoly CLNP (OSI) a IP (TCP/IP).
    \item \textbf{2. Linková (spojová) vrstva }- poskytuje \textbf{spojení mezi dvěma sousedními systémy}. \textbf{Uspořádává data} z fyzické vrstvy do logických celků (neboli PDU) známých jako \textbf{rámce} (frames). Seřazuje přenášené rámce, stará se o nastavení parametrů přenosu linky, oznamuje neopravitelné chyby. Formátuje fyzické rámce, opatřuje je fyzickou adresou a poskytuje synchronizaci pro fyzickou vrstvu. Na této vrstvě pracuje \textbf{switch}, tj vše je řízeno pomocí MAC adres. Protokoly jsou \textbf{Ethernet}, \textbf{TokenRing}, \textbf{FrameRelay}, \textbf{IEEE 802.11}...
    \item\textbf{1. Fyzická vrstva }- fyzická vrstva definuje všechny elektrické a fyzikální vlastnosti zařízení. Jakým signálem je reprezentována logická jednička, jak přijímací stanice rozezná začátek bitu, jaký je tvar konektoru, k čemu je který vodič v kabelu použit. Na této vrstvě pracuje rozbočovač (hub).
\end{itemize}

\section{TCP/IP}
Rodina protokolů TCP/IP (Transmission Control Protocol/Internet Protocol) obsahuje \textbf{sadu protokolů} pro komunikaci v počítačové síti a je hlavním protokolem celosvětové sítě \textbf{Internet}.

\textbf{Komunikační protokol} je množina pravidel, která určují syntaxi a význam jednotlivých zpráv při komunikaci. Architektura TCP/IP je členěna do čtyř vrstev (na rozdíl od referenčního modelu OSI se sedmi vrstvami):

\begin{enumerate}
    \item \textbf{Vrstva síťového rozhraní} (Network interface)
    \item \textbf{Síťová (IP) vrstva} (Internet layer)
    \item \textbf{Transportní vrstva} (Transport layer)
    \item \textbf{Aplikační vrstva }(Application layer)
\end{enumerate}

\noindent\makebox[\textwidth]{\includegraphics[width=.6\textwidth]{assets/4_tcpip}}

Komunikace mezi \textbf{stejnými vrstvami dvou různých systémů} je řízena \textbf{komunikačním protokolem} za použití spojení vytvořeného sousední nižší vrstvou. Architektura umožňuje výměnu protokolů jedné vrstvy bez dopadu na ostatní. Příkladem může být možnost komunikace po různých médiích fyzické vrstvy modelu OSI - ethernet (optické vlákno, kroucená dvojlinka, Wi-Fi, sériová linka).


\section{Vrstva síťového rozhraní}
Nejnižší vrstva umožňuje \textbf{přístup k fyzickému přenosovému médiu}. Je specifická pro každou síť v závislosti na její implementaci. Příklady sítí:
\begin{itemize}
    \item \textbf{Ethernet} --  název souhrnu technologií pro počítačové sítě (LAN, MAN) z větší části standardizovaných jako \textbf{IEEE 802.3}, které používají kabely s \textbf{kroucenou dvoulinkou}, optické kabely (ve starší verzích i koaxiální kabely) pro komunikaci přenosovými rychlostmi od 1 Mbit/s po 100 Gbit/s.
    \item \textbf{Token ring} -- technologie lokální sítě. Principem je předávání vysílacího práva pomocí speciálního rámce (tzv. tokenu) mezi adaptéry, zapojenými do logického kruhu.
    \item \textbf{FDDI} -- síť kruhovou topologií (dva kruhy pro opačné směry přenosu), používá optické kabely.
    \item \textbf{X.25} (později nahrazena Frame Relay), \textbf{SMDS}.
\end{itemize}


\section{Síťová vrstva}
Vrstva zajišťuje \textbf{především síťovou adresaci}, \textbf{směrování} a předávání datagramů (\textbf{packety}). Je implementována ve všech prvcích sítě - \textbf{směrovačích} i \textbf{koncových zařízeních}. Protokoly:
\begin{itemize}
    \item \textbf{IP (Internet Protokol)} -- základní protokol pro přenos dat po sítí. Sám nezaručuje nic. Podle IP adres jen směruje paket k cíli.
    \item \textbf{ICMP (Internet Control Message Protocol)} -- slouží pro odesílání chybových zpráv např. že služba není dostupná.
    \item \textbf{ARP (Adress Resolution Protocol)} -- převádí IP adresy na MAC adresy
    \item \textbf{DHCP (Dynamic Host Configuration Protocol)} -- požívá se pro automatickou konfiguraci počítačů připojených do sítě. Přiděluje například IP adresy.
\end{itemize}

\subsection{Protokol IP (Internet Protocol)}
Od nadřazených protokolů transportní vrstvy obdrží datové segmenty s požadavkem na odeslání. K segmentům připojí vlastní hlavičku a vytvoří IP datagram. V IP hlavičce je především IP adresa příjemce a odesílatele. IP protokol je \textbf{nespojový} (před zahájením výměny dat nevytváří relaci) a \textbf{nespolehlivý} (předání paketů na místo určení není kontrolováno). Paket IP se tedy může ztratit, být doručen mimo pořadí, zdvojen nebo zpožděn. Protokol IP neobsahuje prostředky pro zotavení z chyb tohoto typu. To vše má zajistit nadřízená transportní vrstva – protokol \textbf{TCP}.

\section{Transportní vrstva}
Transportní vrstva je implementována až v \textbf{koncových zařízeních} (počítačích), a umožňuje proto přizpůsobit chování sítě potřebám aplikace. Poskytuje transportní služby kontrolovaným spojením \textbf{spolehlivým} protokolem \textbf{TCP} (transmission control protocol) nebo nekontrolovaným spojením \textbf{nespolehlivým} protokolem \textbf{UDP} (user datagram protocol).

\subsection{TCP (Transmission Control Protocol)}
Vytváří \textbf{virtuální okruh} mezi koncovými aplikacemi a zajišťuje tedy spolehlivý přenos dat.
\begin{itemize}
    \item Spolehlivá transportní služba, doručí adresátovi všechna data \textbf{bez ztráty} a ve \textbf{správném pořadí}.
    \item Služba se spojením, má fáze \textbf{navázání spojení} (3-way handshake), \textbf{přenos dat} a \textbf{ukončení spojení}.
    \item Transparentní přenos libovolných dat.
    \item \textbf{Plně duplexní spojení}, současný obousměrný přenos dat.
    \item Rozlišování aplikací pomocí portů.
    \item Komunikace je řízená pomocí \textbf{příznakových bitů} (ACK, SYN, atd.)
\end{itemize}

\begin{figure}[H]
    \centering
    \includegraphics[width=.6\textwidth]{assets/4_tcp}
\end{figure}

\subsection{UDP (User Datagram Protocol)}
Poskytuje \textbf{nespolehlivou} transportní službu pro takové aplikace, které nepotřebují spolehlivost, jakou má protokol TCP a jde jim čistě o \textbf{rychlost}. Nemá fázi navazování a ukončení spojení a už první segment UDP obsahuje aplikační data. UDP je používán aplikacemi jako je \textbf{DHCP}, \textbf{TFTP}, \textbf{SNMP}, \textbf{DNS} a \textbf{BOOTP}.
\begin{itemize}
    \item Nespolehlivá transportní služba, \textbf{neověřuje} zda data došla v pořádku nebo ve správném pořadí.
    \item Nižší režie než u TCP (\textbf{rychlejší}).
    \item Zajištění spolehlivosti je na aplikacích vyšší vrstvy.
    \item Nemá fázi navázání a ukončení spojení, rovnou zasílá data.
    \item Hlavička UDP má pouze 4 části (délku, zdrojový/cílový port, checksum)
\end{itemize}

\noindent\makebox[\textwidth]{\includegraphics[width=.6\textwidth]{assets/4_udp}}

\section{Aplikační vrstva}
Jedná se přímo o programy (procesy), které využívají přenosu dat po síti ke konkrétním službám pro uživatele. Příklady: \textbf{Telnet} (TCP 23), \textbf{FTP} (TCP 20, 21), \textbf{HTTP} (TCP 80), \textbf{DHCP} (TCP/UDP 67, 68), \textbf{DNS} (TCP/UDP 53), \textbf{SSH} (TCP 22), \textbf{SMTP} (TCP/UDP 25), \textbf{POP3} (TCP 110), \textbf{IMAP} (TCP 143).

Aplikační protokoly používají vždy jednu ze dvou základních služeb transportní vrstvy: \textbf{TCP} nebo \textbf{UDP}, případně obě dvě (např. DNS). Pro rozlišení aplikačních protokolů se používají tzv. \textbf{porty}, což jsou domluvená číselná označení aplikací. Každé síťové spojení aplikace je jednoznačně určeno \textbf{číslem portu} a \textbf{transportním protokolem} (a samozřejmě adresou počítače).

\pagebreak
\section{Jazyk predikátové logiky prvního řádu. Práce s kvantifikátory a ekvivalentní transformace formulí.}
Důsledná a přesná specifikace {objektů} a jejich {tříd} v etapě návrhu umožňuje\textbf{ automatické generování zdrojových kódů}. K mapování UML diagramů na kód dochází v první části implementační fáze vývoje.

Z \textbf{diagramu tříd} lze vygenerovat jednotlivá rozhraní, třídy s proměnnými a metodami (bez implementace) -- \url{http://www.milosnemec.cz/clanek.php?id=199}. Úkolem samotné implementace je pak dopsat těla metod, jejichž chování může být popsáno v \textbf{diagramu aktivit}.

\begin{table}[H]
  \centering
  \begin{tabular}{|l|l|}
    \hline
    \textbf{Analýza a návrh (UML)} & \textbf{Zdrojový kód (Java)}                                \\ \hline
    Třída                          & Struktura typu class                                        \\ \hline
    Role, Typ a Rozhraní           & Struktura typu interface                                    \\ \hline
    Operace                        & Metoda                                                      \\ \hline
    Atribut třídy                  & Statická proměnná označená static                           \\ \hline
    Atribut                        & Instanční proměnná                                          \\ \hline
    Asociace                       & Instanční proměnná                                          \\ \hline
    Závislost                      & Lokální proměnná, argument nebo návratová hodnota zprávy    \\ \hline
    Interakce mezi objekty         & Volání metod                                                \\ \hline
    Případ užití                   & Sekvence volání metod                                       \\ \hline
    Balíček, Subsystém             & Kód nacházející se v adresáři specifikovaném pomocí package \\ \hline
  \end{tabular}
\end{table}

Cílem implementace je \textbf{doplnit} navrženou architekturu (kostru) aplikace \textbf{o programový kód} a vytvořit tak kompletní systém.

\textbf{Implementační model }specifikuje, jak jsou jednotlivé elementy (objekty a třídy) implementovány ve smyslu softwarových komponent.

\textbf{Softwarová komponenta} je definována jako \textbf{fyzicky existující} a \textbf{zaměnitelná část} systému, která vyhovuje požadované množině rozhraní a poskytuje jejich realizaci. Typy softwarových komponent dělíme na:

\begin{itemize}
\item \textbf{Zdrojové kódy} -- části systému zapsané v programovacím jazyce.
\item \textbf{Binární a spustitelné kódy} -- přeložené do strojové kódu procesoru.
\item \textbf{Ostatní části} -- databázové tabulky, dokumenty apod.
\end{itemize}

Jestliže jsme ve fázi analýzy a návrhu pracovali pouze s abstrakcemi dokumentovanými v podobě jednotlivých {diagramů}, pak v \textbf{průběhu implementace dochází k jejich fyzické realizaci}. Implementační model se tedy také zaměřuje na specifikaci toho, jak budou tyto {komponenty fyzicky organizovány podle implementačního prostředí}, pro splnění těchto cílů lze využít následující diagramy:
\begin{itemize}
\item \textbf{Diagram komponent} -- ilustruje organizaci a závislosti mezi softwarovými komponentami.
\item \textbf{Diagram nasazení} -- upřesňuje nejen konfiguraci technických prostředků, ale především rozmístění implementovaných softwarových komponent na těchto prostředcích.
\end{itemize}

\begin{figure}[H]
  \centering
  \includegraphics[width=.5\textwidth]{assets/mapovani_uml.jpg}
\end{figure}

\begin{figure}[H]
\centering
\includegraphics[width=.5\textwidth]{assets/diagram_komponent.jpg}
\end{figure}

\pagebreak
\section{Pojem relace, operace s relacemi, vlastnosti relací. Typy binárních relací. Relace ekvivalence a relace uspořádání.}
%Základní metody úpravy a segmentace obrazu (filtrace, prahování, hrany).

\section{Prahování}
\begin{itemize}
    \item Cílem práhování je \textbf{oddělit pozadí od popředí} na základně stanoveného prahu (nějaká realná hodnota). Výsledkem binární obraz (1 = objekt, 0 = pozadí).
    \item Práh může být buď \textbf{stejný} pro celý obrázek, anebo \textbf{adaptivní} pro jednotlivé části obrazu. Další možností je stanovit práh v \textbf{intervalu} $<a, b>$. Úspěšnost detekci oblastí závisí na správné hodnotě prahu.
    \item Pokud neznáme hodnotu prahu, snažíme se jí stanovit na základně informací získaných z obrazu, který má být \textbf{segmentovaný}.
    \item \textbf{Bimodální histogra}m (dva kopce), \textbf{multimodální histogram }-- práh určit jako \textbf{minimum histogramu} mezi vysokými hodnotami, pak lze dále rekurzivně dělit (předpokládáme, že v obraze jsou převážně dva a více druhů pixelů).
    \item \textbf{Obraz lze rekurzivně dělit} na menší části, ve kterých se vypočte histogram a dle něho určí práh pro konkrétní část (pokud nelze práh určit, lze ho interpolovat pomocí sousedních prahů).
    \item \textbf{Minimalizace rizika chyby}:
          \begin{itemize}
              \item stanovení prahu tak, aby se minimalizovala špatná detekce,
              \item stanovení dle aproximace normálních rozdělení popředí a pozadí \\$\varepsilon = \theta P(t) + (1 - \theta)[1 - Q(t)]$.
              \item nejlepších výsledků lze dosáhnout v \textbf{extrému první derivace}
              \item pokud je zastoupení pozadí a popředí stejné a má stejný rozptyl $(t - \mu )^2 = (t - v)^2 $.
          \end{itemize}
          \begin{figure}[H]
              \centering
              \includegraphics[width=0.8\textwidth]{assets/8_prah_histogram}
          \end{figure}
    \item Na levém obrázku je prah označen $t$ a vyšrafovaná oblast značí chybu, která nastane při prahování, kdy bude špatně rozpoznané popředí/objekt $q(z)$ a pozadí $p(z)$ -- minimalizace chyby.
\end{itemize}


\section{Detekce hran}
\begin{itemize}
    \item Každá oblast je obklopena hranicí.
    \item Hranice se skládá z hran (případně také z jediné zakřivené hrany).

    \item Hrana se skládá z jednotlivých hranových bodů.
    \item Většinou se postupuje tak, že se obraz převede do stupně šedi a následně se naleznou jednotlivé body hran.
    \item Za bod hrany se často považuje místo, kde průběh jasu \textbf{vykazuje náhlou změnu}, případně \textbf{inflexní bod}.
    \item Po nalezení jsou jednotlivé nalezené body hran spojovány různými technikami do hran a celých hranic.
\end{itemize}
\subsection{Detekce hran s využitím gradientu}
\begin{itemize}
    \item Hrana je v obrazu zastoupeny (prudkou) \textbf{změnou jasu}, lze ji tedy najít zkoumáním síly a směru gradientu v jednotlivých bodech.
    \item Pro určení směru gradientu či hrany (​\textbf{směr gradientu je kolmý ke směru hrany​}) je třeba provést ​ \textbf{derivaci​} (nejlépe v x i y), která je při výpočtu nahrazena diferencí.
    \item Diference může být buď \textbf{centrální} nebo \textbf{dopředná}/\textbf{zpětná}.
          \begin{equation*}
              \begin{split}
                  d_x = \dfrac{I(x - 1, y) - I(x + 1, y)}{2} \, , \quad     d_y= \dfrac{I(x, y - 1) - I(x, y + 1)}{2} \, .
              \end{split}
          \end{equation*}
    \item Velikost hrany lze určit velikostí gradientu (norma), hrana je tam, kde $e > \textrm{práh}$. (hrana je kolmá k gradientu) \\
          $ e(x, y) = \sqrt{(f_x(x,y)^2+ f_y(x,y)^2})$
    \item Směr hrany a gradientu lze určit (kde $\varphi$ -- směr gradientu, $\psi$ -- směr hrany)
          \begin{equation*}
              \begin{array}{c}
                  \varphi(x, y) = \arctan{[\frac{f_y(x,y)}{f_x(x,y)}]}, \psi(x,y) = \varphi(x,y) + \frac{\pi}{2}.
              \end{array}
          \end{equation*}
    \item Výše uvedené derivace lze nahradit \textbf{konvolučními maskami}
          \begin{itemize}
              \item \textbf{Sobel} -- vážený průměr (Prewittove dělá pouze normální)
              \item \textbf{Kirsch} -- počítání hran v 8 směrech
          \end{itemize}
    \item Robertsův operátor:
          \begin{equation*}
              \begin{bmatrix}
                  -1 & 0 \\[0.3em]
                  0  & 1
              \end{bmatrix}.
          \end{equation*}
    \item operátor Previttové:
          \begin{equation*}
              \begin{bmatrix}
                  -1 & -1 & -1 \\[0.3em]
                  0  & 0  & 0  \\[0.3em]
                  1  & 2  & 1  \\
              \end{bmatrix}.
          \end{equation*}
    \item Sobelův operátor:
          \begin{equation*}
              \begin{bmatrix}
                  -1 & -2 & -1 \\[0.3em]
                  0  & 0  & 0  \\[0.3em]
                  1  & 2  & 1  \\
              \end{bmatrix}.
          \end{equation*}
    \item Kirschův operátor:
          \begin{equation*}
              \begin{bmatrix}
                  -5 & -5 & -5 \\[0.3em]
                  3  & 0  & 3  \\[0.3em]
                  3  & 3  & 3  \\
              \end{bmatrix}.
          \end{equation*}
\end{itemize}
\begin{figure}[H]
    \begin{center}
        \includegraphics[width=0.25\textwidth]{assets/8_det_hran_grad}
        \caption{Velikost gradientu a jeho první a druhá derivace}
    \end{center}
\end{figure}
\subsection{Detekce hran hledáním průchodu nulou}
\begin{itemize}
    \item První derivace obrazové funkce nabývá svého maxima v místě hrany.
    \item \textbf{Druhá derivace protíná} v místě hrany \textbf{nulovou hodnotu}.
    \item Spolehlivější metoda, než hledání maxima v první derivaci. \textbf{NE} v případě, že je obraz postižen šumem. V tomto případě selhává, jelikož druhá derivace ještě více zesílí šum.
\end{itemize}
\subsection*{Laplaceův operátor (druhá derivace gradientu)}
\begin{itemize}
    \item Pro výpočet se používá symetrická diference nebo konvoluční masky (na krajích je maska ořezaná)
          \begin{equation*}
              \begin{split}
                  d_x &= I(x - 1, y) - 2I(x, y) + I(x + 1, y) \, ,\\
                  d_y&= I(x, y - 1) - 2I(x, y) + I(x, y + 1) \, .
              \end{split}
          \end{equation*}
          \begin{figure}[H]
              \begin{center}
                  \includegraphics[width=0.5\textwidth]{assets/8_priklady_laplace}
              \end{center}
          \end{figure}
    \item Hrana je detekována jako \textbf{změna znaménka v průchodu mezi dvěma extrémy}.
    \item Je \textbf{více citlivý na šum než první derivace} (i při malém šumu je detekováno množství falešných hran).
    \item Pro redukci šumu a zahlazení vysokých frekvencí lze použít \textbf{Gaussův operátor}:
          \begin{equation*}
              \begin{bmatrix}
                  1 & 2 & 1 \\[0.3em]
                  2 & 1 & 2 \\[0.3em]
                  1 & 2 & 1 \\
              \end{bmatrix}.
          \end{equation*}
\end{itemize}
\subsection{Cannyho detekce hran}
Canny první stanovil požadavky, které by měl detektor splňovat a následně navrhl detektor. %zabili kennyho, parchanti
Požadavky:
\begin{itemize}
    \item \textbf{Minimalizovat} pravděpodobnost \textbf{chybné detekce}.
    \item Najít polohu hrany, co \textbf{nejpřesněji}.
    \item Bod hrany identifikovat \textbf{jednoznačně}.
\end{itemize}
\textbf{Postup:}
\begin{enumerate}
    \item Eliminace šumu Gaussovým filtrem.
    \item Velikost a směr gradientu -- nejčastěji Sobelův operátor (nebo centrální derivace).
    \item Nalezení lokálních maxim a stanovení interpolace v osmi okolí. \textbf{Redukce na hranu velikosti 1 px}.
    \item Eliminace nevýznamných hran (\textbf{double thresholding})
          \begin{itemize}
              \item Všechny body, kde je velikost hrany $\leq t_{high}$ -- \uv{jistá} hrana
              \item Pak ty, které jsou $ > t_{low}$ a sousedí s hranou -- \uv{jistá} hrana
          \end{itemize}
\end{enumerate}

\pagebreak
\section{Pojem operace a obecný pojem algebra. Algebry s jednou a dvěma binárními operacemi.}
\section{Algebra}
Algebra je naukou o \textbf{algebraických strukturách}, tedy \textbf{množinách}, na nichž jsou zavedeny nějaké \textbf{operace}. Slouží pro popis objektů reálného světa a operací prováděných s těmito objekty. Příklady algeber:
\begin{itemize}
    \item $( \mathbb{N} , \{+^2\} )$ - sčítání nad množinou přirozených čísel,
    \item $ ( 2^M , \{\cup, \cap\} ) $ - množina všech podmnožin $ M $ s operací průnik a sjednocení.
\end{itemize}

\subsection{Definice}
Každý objekt algebry je reprezentován \textbf{datovým nosičem} (množina popisující data, se kterými pracujeme). A \textbf{operacemi} -- nejjednoduššími transformacemi, které nad daty můžeme realizovat. \textbf{Algebraická struktura} je definována jako $(A, \circ)$, kde:
\begin{itemize}
    \item $A$ -- nosič algebry (množina objektů -- čísel, proměnných, \ldots),
    \item $\circ$ -- množina operací nad nosičem $ X $.
\end{itemize}

\section{Operace}
Operace na množině $A$ je definována jako zobrazení
\begin{equation}
    f: A^n \rightarrow A,
\end{equation}
tedy zobrazení, které každé n-tici prvků množiny $A$, jednoznačně přiřazuje prvek z množiny $A$.Číslo n nazýváme \textbf{arita operace} a podle něj operace označujeme jako \textbf{nulární} (n = 0), \textbf{unární} (n = 1), \textbf{binární} (n = 2), \textbf{ternární} (n = 3).

\section{Algebraické struktury s jednou binární operací}
Definována jako $(A, \circ)$ s jedním nosičem ($ A $) a jednou homogenní binární operací ($ \circ $). Nejprve je nutné zmínit \textbf{vlastnosti binárních operací}:
\begin{itemize}
    \item \textbf{asociativita}: $a * (b * c) = (a * b) * c$,
    \item \textbf{komutativita}: $a * b = b * a$.
\end{itemize}
Kromě již zmíněné asociativity a komutativity algebraické struktury také zavádí existenci:
\begin{itemize}
    \item \textbf{Jednotkového prvku}: $e$ takové, že $\forall x \in X: x \circ e = e \circ x = x$. Tedy prvek, který nezmění výsledek (1 u násobení, 0 u sčítání).
    \item \textbf{Inverzního prvku}: $\overline{x}$ takové, že $\forall x \in X: x \circ \overline{x} = \overline{x} \circ x = e$. Tedy prvek, který převede výsledek na jednotkový prvek.
\end{itemize}
\textbf{neutrálního} či \textbf{inverzního prvku}, a další charakteristiky.

\subsection{Klasifikace algebraických struktur}
Všechny níže uvedené klasifikace algebraické struktury $(A, \circ)$ zahrnují i ty co jsou pod nimi. Tedy pokud je nějaká algebraická struktura (AS) Monoid, je i Pologrupa a Grupoid.
\begin{enumerate}
    \item \textbf{Grupoid} -- \textbf{uzavřenost} (univerzalita) na nosiči (po výpočtu je výsledek stále v množině $A$).
    \item \textbf{Pologrupa} -- splňuje vlastnost \textbf{asociativity}.
    \item \textbf{Monoid} -- existence \textbf{jednotkového prvku}.
    \item \textbf{Grupa} -- existence \textbf{inverzního prvku}.
    \item \textbf{Abelova grupa} -- splňuje vlastnost \textbf{komutativity} (symetrická podle diagonály).
\end{enumerate}

\noindent \textbf{Kongruence} -- označuje ekvivalenci na algebře, která je slučitelná se všemi operacemi na algebře.

\subsection{Morfismy}
\begin{itemize}
    \item \textbf{Homomorfismus} -- zobrazení, které převádí jednu algebraickou strukturu na jinou: $f(a_1 \cdot a_2) \rightarrow f(a_1) \circ f(a_2)$.
    \item \textbf{Izomorfismus} -- bijektivní homomorfismus.
    \item \textbf{Epimorfismus} -- surjektivní homomorfismus.
    \item \textbf{Monomorfismus} -- injektivní homomorfismus.
    \item \textbf{Endomorfismus} -- homomorfismus z objektu do sebe sama (stejná množina).
    \item \textbf{Automorfismus} -- endorfismus, který je izomorfní.
\end{itemize}

\section{Okruhy (Algebraické struktury s dvěma binárními operací)}
\textbf{Okruh} je algebraický systém $(A, +, \cdot)$ se dvěma základními binárními operacemi, kde první $(A, +)$ je \textbf{abelova grupa} a druhá $(A, \cdot)$ je alespoň \textbf{pologrupa}. Podobně jako u předchozí AS, i zde se zavádí nový pojem:
\begin{itemize}
    \item \textbf{Existence dělitele nuly} -- říká, že ve struktuře existují 2 nenulové prvky, pro něž platí $a \circ b = 0$.
\end{itemize}
U všech typů okruhů musí být splněna podmínka první struktury, která musí být \textbf{abelova grupa}, a druhá musí být:
\begin{itemize}
    \item \textbf{Okruh} - uzavřená ({U}), asociativní ({A}) [pologrupa].
    \item \textbf{Unitární okruh} - {U}, {A}, existence jednotkového prvku (J) [monoid].
    \item \textbf{Obor Integrity} - {U}, {A}, {J} [monoid] + \textbf{nesmí} obsahovat dělitele nuly.
    \item \textbf{Těleso} - {U}, {A}, {J} a existence inverzního prvku (I) [grupa] + \textbf{nesmí} obsahovat dělitele nuly.
    \item \textbf{Pole} - {U}, {A}, {J}, {I} a komutativita [grupa] + \textbf{nesmí} obsahovat dělitele nuly.
\end{itemize}

\pagebreak
\section{FCA – formální kontext, formální koncept, konceptuální svazy. Asociační pravidla, hledání často se opakujících množin položek.}
\begin{itemize}
    \item Deep learning neboli \textbf{hluboké učení}, známé také jako hierarchické učení, je \textbf{sbírka algoritmů} používaných ve strojovém učení.
    \item Používají se k~modelování abstrakcí na vysoké úrovni v~datech za pomocí modelových architektur, které se skládají z~několika nelineárních transformací.
    \item Hluboké učení je součástí široké skupiny metod používané pro strojové učení, které jsou založeny na učení reprezentace dat.
\end{itemize}
Hluboké strukturované učení může být:
\begin{itemize}
    \item{\textbf{Kontrolované (s~učitelem)} - všechna data jsou kategorizovaná do tříd, algoritmy se učí předpovídat výstup ze vstupních dat.}
    \item{\textbf{Částečně kontrolované} - data jsou částečně kategorizovaná do tříd. Pří tomto přístupu učení lze využít kombinaci kontrolovaného a~nekontrolovaného přístupu učení.}
    \item{\textbf{Nekontrolované (bez učitele)} - data nejsou kategorizovaná do tříd, algoritmy se učí ze struktury vstupních dat.}
\end{itemize}
Hluboké učení je specifický přístup, použitý k~budování a~učení neuronových sítí, které jsou považovány za velmi spolehlivé rozhodovací uzly. Jestliže vstupní data algoritmu procházejí řadou nelinearit a~nelineárních transformací, tak tento algoritmus je považován za \uv{deep} algoritmus.

Odstraňuje také ruční identifikaci příznaků (obrázek \ref{fig:ml_vs_ann}) z~dat a~místo toho se spoléhá na jakýkoliv trénovací proces, které má za úkol zjistit užitečné vzory ve vstupních příkladech. To dělá neuronovou síť jednodušší a~rychlejší, a~může přinést lepší výsledky než z~oblasti umělé inteligence.

\begin{figure}[H]
    \centering
    \includegraphics[width=.5\linewidth]{assets/9_ml_vs_ann}
    \caption{Hlavním rozdílem mezi strojovým a~hlubokým učením je ten, že u~strojového se příznaky musí extrahovat manuálně.}
    \label{fig:ml_vs_ann}
\end{figure}

\subsection{Konvoluční neuronové sítě \textit{CNN} - Convolution neural network}
\begin{itemize}
    \item Speciálním druhem vícevrstvých neuronových sítí a~jsou navrženy tak, aby rozpoznaly vizuální vzory přímo z~pixelu obrazu s~minimálním předzpracováním.
    \item Mohou rozpoznat vzory s~extrémní variabilitou (například ručně psané znaky) a~odolnost vůči deformacím a~jednoduchým geometrickým transformacím.
    \item Síť využívá matematickou operaci zvanou konvoluce alespoň v~jedné jejich vrstvě.
\end{itemize}

Nejznámější a~nejvíce používanou konvoluční neuronovou sítí jsou modely LeNet.
Hlavní kroky LeNet sítě jsou:
\begin{itemize}
    \item{\textbf{Konvoluce} - tyto vrstvy provádějí konvoluci nad vstupy do neuronové sítě.}
    \item{\textbf{Nelinearita (ReLU)} - tato vrstva je použita po každé konvoluční vrstvě a~jejím cílem je nahrazení všech negativních pixelů nulou ve výstupu této vrstvy (příznaková mapa).}
    \item{\textbf{Pooling/sub sampling} - ze vstupního obrazu vyextrahuje pouze zajímavé části pomocí některých matematických operací (max, avg, sum), a~tím se \textbf{redukuje jeho dimenzionalita}.}
    \item{\textbf{Fully connected layer/klasifikace} - tato vrstva vychází z~původních umělých neuronových sítí, konkrétně z~vícevrstvého perceptronu. Tato vrstva je typicky umístěna na konci sítě a~je propojena s~klasifikační vrstvou pro predikci.}
\end{itemize}
\begin{figure}[H]
    \centering
    \includegraphics[width=.9\linewidth]{assets/9_cnn.pdf}
    \caption{Řetězec LeNet konvoluční neuronové sítě}
    \label{fig:cnn}
\end{figure}

\pagebreak
\section{Metrické a topologické prostory – metriky a podobnosti.}
\section{Metrický prostor}
Metrický prostor je \textbf{matematická struktura}, pomocí které lze formálním způsobem definovat pojem \textbf{vzdálenosti}. Na metrických prostorech se poté definují další topologické vlastnosti jako např. \textbf{otevřenost} a \textbf{uzavřenost} množin, jejichž zobecnění pak vede na ještě abstraktnější matematický pojem \textbf{topologického prostoru}.

\subsection{Formální definice}
Metrický prostor je dvojice $(M, \rho)$, kde $M$ je libovolná neprázdná množina a $\rho$ je \textbf{metrika}, což je zobrazení:
\begin{equation*}
    \rho: M \times M \rightarrow \mathbb{R},
\end{equation*}
které splňuje následující axiomy:
\begin{enumerate}
    \item \textbf{Nezápornost}: $\forall x, y \, \rho(x, y) \geq 0$.
    \item \textbf{Totožnost}: $\forall x, y \, \rho (x, y)  = 0 \Leftrightarrow x = y$.
    \item \textbf{Symetrie}: $\forall x, y \in M: \, \rho(x, y) = \rho(y,x)$.
    \item \textbf{Trojúhelníková nerovnost}: $\forall x, y, z \in M: \, \rho(x, y) + \rho(y,z) \geq \rho(x, z)$.
\end{enumerate}

\subsection{Metriky v $\mathbb{R}^n$}
U metrik v $\mathbb{R}^n$ platí:
\begin{itemize}
    \item Každý normovaný vektorový prostor je metrickým prostorem.
    \item Množina reálných čísel spolu s metrikou $\rho(x, y) = |x - y|$, kde $x, y$ jsou libovolné body množiny $\mathbb{R}$ tvoří \textbf{úplný metrický prostor}.
\end{itemize}
Mezi nejpoužívanější \textbf{metriky} na Euklidovském prostoru $\mathbb{R}^n$ patří:
\begin{enumerate}
    \item \textbf{Manhattanská} -- $\rho_1(x, y) = \sum_{i = 0}^n |x_i - y_i|$.
    \item \textbf{Euklidovská} -- $\rho_2(x, y) = \sqrt{\sum_{i = 0}^n |x_i - y_i|^2}$.
    \item \textbf{Minkowského} -- $\rho_3(x, y) = (\sum_{i = 0}^n |x_i - y_i|^P)^{1/P}$, kde $P \geq 1; \, P \in \mathbb{R}$.
    \item \textbf{Čebyševova (Maximova)} -- $\rho_{\max}(x, y) = \max_{\forall i} |x_i - y_i|$, speciální případ Minkowského metriky pro $P = \infty$.
\end{enumerate}

\subsection{Podobnosti a nepodobnosti}
\begin{itemize}
    \item \textbf{Cosinova podobnost} -- míra podobnosti dvou vektorů, která se získá výpočtem kosinu úhlu těchto vektorů:
          \begin{equation*}
              S_c(x, y) = \frac{x \cdot y}{||x|| \, ||y||} = \frac{\sum_{i = 0}^{n} x_i - y_i}{\sqrt{\sum_{i = 0}^{n} x_i^2} \sqrt{\sum_{i = 0}^{n} y_i^2}}
          \end{equation*}
    \item \textbf{Jaccardova podobnost} -- $S_j(X, Y) = \frac{|X \cap Y|}{X \cup Y}$ používá se pro porovnání podobnosti dvou množin.
    \item \textbf{Jaccardova nepodobnost} (svým způsobem vzdálenost) -- $d = 1 - S$, $d = \frac{1}{S} - 1$ pro $S \neq 0$.
\end{itemize}

\subsection{Vzdálenost mezi slovy}
K určení vzdálenosti mezi textovými řetězci definujeme tyto \textbf{vzdálenosti}:
\begin{enumerate}
    \item \textbf{Hammingova} -- počet rozdílných písmen na stejných pozicích: $d_h(\textrm{Karolin, Kathrin}) = 3$, lze použít pouze pro \textbf{stejně dlouhá slova!}.
    \item \textbf{LCS (Longest common subsequence)} -- počet operací \textbf{vkládání} a \textbf{mazání} nutných k převodu jednoho slova na druhé: $d_{\textrm{LCS}  }(\textrm{kitten, sitting}) \Rightarrow \textrm{itten [-K]} \rightarrow \textrm{sitten [+S]} \rightarrow \textrm{sittn [-E]} \rightarrow \textrm{sittin [+I]} \rightarrow \textrm{sitting [+G]} =$ \textbf{5 operací}.
    \item \textbf{Levenshteinova} -- počet operací \textbf{vkládání}, \textbf{mazání}, \textbf{substituce}  nutných k převodu jednoho slova na druhé: $d_l(\textrm{kittne, sitting}) \Rightarrow \textrm{sitten [K }\sim \textrm{ S]} \rightarrow \textrm{sittin [E }\sim \textrm{ I]} \rightarrow \textrm{sitting [+G]} =$ \textbf{3 operace}.
\end{enumerate}
\textbf{Editační vzdálenost} patří zde LSC a Levenstheinova vzdálenost (při určení se používají úpravy).

\subsection{Normalizace}
Snažíme se dostat všechny hodnoty atributů ve sloupci do intervalu $\langle 0, 1\rangle$, proto dělíme celý sloupce jeho maximem (dělíme v rámci daného sloupce, pro každý sloupec zvlášť vlastním maximem).

\subsection*{Příklad}
$d'_1$ a $d'_2$ reprezentují \textbf{normovaný tvar} vzdáleností $d_1$ a $d_2$:
\begin{table}[H]
    \centering
    \begin{tabular}{l|lllllll}
               & $t_1$ & $t_2$ & $t_3$ & $t_4$         & $t_5$         & $t_6$ & $t_7$ \\\hhline
        $d_1$  & 1     & 0     & 0     & 1             & 3             & 0     & 1     \\
        $d_2$  & 1     & 0     & 0     & 2             & 2             & 0     & 0     \\
        $d'_1$ & 1     & 0     & 0     & $\frac{1}{2}$ & 1             & 0     & 1     \\
        $d'_2$ & 1     & 0     & 0     & 1             & $\frac{2}{3}$ & 0     & 0
    \end{tabular}
\end{table}
\begin{table}[H]
    \centering
    \begin{tabular}{l|l|l|l|l|l|l}
                             & $i = 1$        & $i = 2$       & $i = 3$                  & $i = \infty$ & Cosinova                          & Jaccard        \\\hhline
        $\rho_i(d_1, d_2)$   & 3              & $\sqrt{3}$    & $ \sqrt{3}^3 $           & 1            & $\frac{1}{6 \sqrt(3)}$            & $\frac{3}{4}?$ \\
        $\rho_i(d'_1, d'_2)$ & $\frac{11}{6}$ & $\frac{7}{6}$ & $\frac{3 \sqrt{251}}{6}$ & 1            & $\frac{5 \sqrt(154)}{3 \sqrt(2)}$ & $\frac{3}{4}?$
    \end{tabular}
\end{table}

\section{Topologický prostor}
Jedná se o \textbf{rozšíření (zobecnění) metrického prostoru}. Cílem topologie je studium \textbf{vlastností prostorů}. Na rozdíl od teorie metrických prostorů se v topologii \textbf{nezajímáme o vzdálenosti mezi body} prostoru a prostory považujeme za stejné, pokud \textbf{se na sebe dají vzájemně přeměnit} nějakou spojitou deformací. Takže např. nerozlišujeme mezi koulí a krychlí, ostatně koule se změní v krychli již při přechodu mezi dvěma ekvivalentními metrikami v $\mathbb{R}^3$.

Základním pojmem, který se v topologii studuje je \textbf{spojitost zobrazení}. Proto není úplně potřeba vědět přesně, jak jsou od sebe které body daleko. Vystačíme si s informacemi, že jisté body se nekonečně blíží k nějakému bodu prostoru.

\subsection{Formální definice}
Topologickým prostorem nazveme množinu $X$ společně s kolekcí $\tau$ podmnožin $X$, tedy \textbf{dvojici} $(X, \tau)$, splňující následující axiomy:
\begin{enumerate}
    \item $\emptyset, X \in \tau$,
    \item $\forall A, B \in \tau \Rightarrow  A \cup B \in \tau $, tedy \textbf{sjednocení} libovolného počtu (tj. konečného, spočetného i nespočetného) množin z $\tau$ leží v $\tau$,
    \item $\forall A, B \in \tau \Rightarrow A \cap B \in \tau $, tedy \textbf{průnik} konečného počtu množin z $\tau$ leží v $\tau$.
\end{enumerate}

\subsection{Uzavřená, otevřená množina}
Pro topologický prostor $(X, \tau)$ je každá množina $A \in \tau$ otevřená množina, a její doplněk je uzavřená množina.

\subsection{Souvislost nesouvislost}
Pokud sjednocením dvou neprázdných množin z $\tau$ získáme všechny prvky topologie (celé $ X $), tak je topologie \textbf{nesouvislá}. Příklad:

\begin{center}
    \begin{minipage}[t]{0.50\textwidth}
        $X = \{a, b, c\}$\\
        $\tau = \{\emptyset, X, \{a, b\}, \{c\}\}$\\
        $\{a, b\} \cup \{c\}  = \{a, b, c\} \Rightarrow \, \tau$ je \textbf{nesouvislá}
    \end{minipage}
    \begin{minipage}[t]{0.40\textwidth}
        \textbf{Poznámka:} sjednocované množiny musí být \textbf{disjunktní}, tedy nesmí mít společné prvky, např.: $\{a, b\}$ a $\{a, c\}$ již disjunktní nejsou, ale nenaruší souvislost.
    \end{minipage}
\end{center}
\smallskip
Topologický prostor $ \tau $, je \textbf{souvislý} právě tehdy, když jedině podmnožiny v $\tau$, které jsou současně otevřené i uzavřené jsou $X$ a $\emptyset$. V opačném případě je $\tau$ \textbf{nesouvislý}.

\subsection{Uzávěrový systém}
Uzávěrový systém $C$ nad množinou $X$ obsahuje $X$ a $\forall A, B \in C$ platí, že $A \cap B \in C$.
\begin{equation*}
    \begin{aligned}
        R_i \subseteq A \times A \quad R_i^* = \textrm{[tranzitivně-reflexivní uzávěr]} \\
        \textrm{pro } R_i^* \textrm{ platí } \quad R_1^* \neq R_2^*; \quad R_1^* \cap R_2^* \in G;\quad R_1^* \cup R_2^* \notin G
    \end{aligned}
\end{equation*}

\subsection{Uzávěrový operátor $cl(A)$}
Uzávěr (\textit{closure}) $A \cup C \rightarrow cl(A)$ je \textbf{nejmenší uzavřená množina obsahující daný prvek}. Analogie u konceptů $cl(B) = B^{\downarrow\uparrow}$ a $cl(A) = A^{\uparrow\downarrow}$. Vlastnosti uzávěrového operátoru:
\begin{enumerate}
    \item \textbf{Idempotence} -- $cl(cl(A)) = cl(A)$.
    \item \textbf{Extensionalita} -- $A \subseteq cl(A)$.
    \item \textbf{Monotónost} -- $A \subseteq B \Rightarrow cl(A) \subseteq cl(B)$.
\end{enumerate}

Platí-li navíc $cl(\emptyset) = \emptyset$ a $cl(A \cup B) = cl(A) \cup cl(B)$, pak je uzávěr $A = cl(A)$. Jinými slovy se jedná o nejmenší komplement (doplněk) množin TP obsahující daný prvek.
\\
\begin{figure}[H]
    \centering
    \includegraphics[width=0.4\textwidth]{assets/9_tpl_space}
\end{figure}




\pagebreak
\section{Shlukování.}
\textbf{Shluková analýza} je vícerozměrná statistická metoda, která se používá ke \textbf{klasifikaci objektů}. Slouží k \textbf{třídění jednotek do skupin} (shluků) tak, aby si jednotky náležící do stejné skupiny byly podobnější než objekty z ostatních skupin.
\begin{itemize}
    \item \textbf{Shlukování} -- proces \textbf{seskupování dat} do skupin na základě podobnosti.
    \item \textbf{Shluk} -- množina maximálně si \textbf{podobných} v rámci shluku a maximálně \textbf{odlišných} mezi shluky.
    \item \textbf{Podobnost} mezi objekty se stanoví na základě \textbf{vzdálenosti} (některé z metrik zmíněných výše -- Manhattan, Euklidovská, Minkowského, Čevyševova (Maximova)).
\end{itemize}

Metody shlukování můžeme klasifikovat do dvou základních kategorií \textbf{hierarchické} a \textbf{nehierarchické metody}.

\section{Hierarchické shlukování}
Rozlišujeme následující přístupy:
\begin{enumerate}
    \item \textbf{divizní} (vycházíme z celku, jednoho shluku, a ten dělíme),
    \item \textbf{aglomerativní} (vycházíme z jednotlivých objektů, shluků o jednom členu, a ty spojujeme).
\end{enumerate}

\noindent \textbf{Výhody/nevýhody}:
\begin{itemize}
    \item[$+$] \textbf{Není třeba předem specifikovat počet shluků}.
    \item[$+$] Uživatel dokáže často dobře hierarchické struktury interpretovat (odpovídají intuici).
    \item[$-$] V každém kroku řeší \textbf{pouze lokálně nejlepší řešení}, nebere odhad na další postup.
    \item[$-$] \textbf{Problém s rozsáhlými daty}, neboť dolní odhad složitosti je $O(n^2)$.
\end{itemize}

\begin{figure}[H]
    \centering
    \includegraphics[width=0.8\textwidth]{assets/10_clustering}
\end{figure}

\subsection{Dendrogram}
\begin{figure}[H]
    \centering
    \includegraphics[width=0.8\textwidth]{assets/10_dendrogram}
\end{figure}

\subsection{Metody aglomerativního shlukování (měření vzdálenosti mezi shluky)}
Nevýhodou těchto metod je, že mohou vzniknout nejednoznačnosti už na začátku shlukování, které se projeví až později ve velkých shlucích. Předchozí kroky není možné změnit.
\begin{itemize}
    \item \textbf{Single linkage} (nearest neighbor) -- vzdálenost shluků je určena vzdáleností \textbf{dvou nejbližších} prvků z různých shluků. Použití této metody vede k tomu, že jsou objekty taženy k sobě, výsledkem jsou dlouhé řetězy menších shluků.
          \begin{figure}[H]
              \centering
              \includegraphics[width=0.3\textwidth]{assets/10_single_linkage}
          \end{figure}
    \item \textbf{Complete linkage} (furthest neighbor) -- vzdálenost shluků je určena naopak vzdáleností \textbf{dvou nejvzdálenějších} prvků z různých shluků. Funguje dobře, obvykle tvoří poměrně kompaktní shluky.
          \begin{figure}[H]
              \centering
              \includegraphics[width=0.3\textwidth]{assets/10_complete_linkage}
          \end{figure}
    \item \textbf{Average linkage} (průměrná vazba) -- vzdálenost shluků je určena jako \textbf{průměr vzdáleností všech párů objektů} z různých shluků. Může být ve \textbf{vážené} i \textbf{nevážené} podobě. Nejčastěji používaná míra pro vzdálenost.
          \begin{figure}[H]
              \centering
              \includegraphics[width=0.3\textwidth]{assets/10_avg_linkage}
          \end{figure}
    \item \textbf{Centroidní metoda} -- pro spočítání nepodobnosti objektů se v této metodě využívá \textbf{euklidovská metrika}, v které se změří vzdálenosti \textbf{těžišť shluků} nebo objektů. Následně dojde ke sloučení shluků, které mají \textbf{nejmenší vzdálenost mezi těžišti}. Může být nevážená (mediánová metoda) nebo vážená (váží se podle velikosti shluku).
    \item \textbf{Wardova metoda} -- metoda \textbf{založena na ztrátě informací}, která vzniká při shlukování. Kritériem pro shlukování je celkový součet druhých mocnin \textbf{odchylek} každého objektu od těžiště shluku, do kterého náleží.  Hodí pro práci s objekty, které mají stejný rozměr proměnných.
\end{itemize}

\subsection{Metody divizního shlukování}
Divizní metody berou množinu objektů, kterou mají zpracovat, jako \textbf{jeden shluk}, ten dále dělí na menší shluky a tím vytváří hierarchický systém. Každý shluk je rozdělen na dva nové, tak aby byl rozklad optimální vůči nějakému kritériu, a na konci tohoto postupu budou všechny shluky jednoprvkové.

Tento postup je kvůli \textbf{exponenciální časové složitosti} (nalezení optimálního rozkladu množiny $n$ objektů vyžaduje prozkoumat až $2^{n-1}-1$ možností) prakticky proveditelný jen pro malý počet objektů.

\subsection*{MacNaughton–Smithova metoda}
Tato metoda se snaží \textbf{snížit časovou náročnost} divizních algoritmů až na \textbf{kvadratickou}, ale za cenu toho, že výsledné rozdělení \textbf{nemusí být optimální}. Je tedy aplikovatelná i na rozsáhlejší množiny objektů při nevelkých nárocích na čas počítače.

V porovnání s aglomerativními přístupy je ale \textbf{stále pomalejší}. Pomocí středních vzdáleností se vybere objekt uvnitř shluku, který vytvoří nový shluk a na základě rozdílů středních vzdáleností objektů z původního a objektů z nového shluku se objekt přiřadí do nového shluku, zůstane v původním. Výhodou oproti aglomerativnímu přístupu jsou j\textbf{ednoznačnější výsledky} pro větší shluky.

\section{Nehierarchické shlukování}
\begin{itemize}
    \item Takový potup postup, při němž se každý objekt vloží do jednoho z $ k $ disjunktních shluků.
    \item Předpokládá se, že uživatel předem stanoví $ k $, tj. požadovaný \textbf{cílový počet shluků}.
\end{itemize}

\subsection{K-means}
K-means \textbf{minimalizuje průměrnou vzdálenost mezi prvky} téhož shluku.
\begin{enumerate}
    \item Stanov \textbf{požadovaný počet} $ k $ shluků.
    \item Vyber náhodně výchozích $ k $ jader.
    \item Přiřaď každému z N objektů číslo shluku, které odpovídá číslu \textbf{nejbližšího jádra}.
    \item \textbf{Předefinuj pozice jader} všech k shluků tak, že bude použit \textbf{průměr hodnot} prvků v daném shluku.
    \item Opakuj kroky 3. a 4. až do situace, kdy se příslušnost do shluků stabilizuje (po iteraci není žádný objekt zařazen do jiného shluku než před ní).
\end{enumerate}

\begin{figure}[H]
    \centering
    \includegraphics[width=\textwidth]{assets/10_kmeans}
\end{figure}

\noindent\textbf{Výhody}
\begin{itemize}
    \item[$+$] \textbf{Jednoduchý} (lehká implementace i ladění).
    \item[$+$] Intuitivní objektivní funkce, která optimalizuje podobnost uvnitř shluků.
    \item[$+$] Poměrně \textbf{efektivní}: složitost $ O(T  K  m N) $, kde $ m $ je počet objektů, $ K $ je počet shluků, $ N $ počet atributů a $ T $ počet iterací. Obvykle bývají hodnoty $ T$ a $ K << m $ .
\end{itemize}
\textbf{Nevýhody}
\begin{itemize}
    \item[$-$] Použitelné, jen tam, kde \textbf{umíme spočítat průměr}. Co kategorická data?
    \item[$-$] Velmi \textbf{záleží na inicializaci} – nebezpečí uvíznutí v lokálním minimu.
    \item[$-$] \textbf{Požaduje} se znalost \textbf{počtu shluků}.
    \item[$-$] Nevhodné pro zašuměná data s výjimkami (outliers).
    \item[$-$] Nevhodné pro situace, kdy shluky nemají konvexní tvar .
\end{itemize}

\subsection{Odhad počtu shluků}
Sledujeme \textbf{jak rychle klesá} hodnota objektivní funkce (výpočet vzdálenosti) pro zvyšující se počet jader $ k $. Obecně hledáme ,,prudký pohyb'' v poklesu, který značí použití dané hodnoty $k$ (jakmile přestane $k$ prudce klesat, dosáhli jsme požadovaného počtu jader).

\pagebreak
\section{Náhodná veličina. Základní typy náhodných veličin. Funkce určující rozdělení náhodných veličin.}
\begin{itemize}
    \item \textbf{Náhodný pokus} -- děj, jehož výsledek není předem jednoznačně určen podmínkami, za nichž probíhá.
    \item \textbf{Náhodný jev} -- tvrzení o výsledku náhodného pokusu, přičemž o pravdivosti tohoto tvrzení lze po ukončení pokusu rozhodnout.
          Náhodná veličina \uv{NV} -- \textbf{číselné} vyjádření výsledku náhodného pokusu.
\end{itemize}

\section{Náhodná veličina}
\begin{itemize}
    \item \textbf{Funkce}, která \textbf{každému elementárnímu jevu} $\omega \in \Omega$ \textbf{přiřadí reálné číslo} (převádí elementární jevy (abstraktní objekty) na čísla).
    \item Obvykle značíme velkými písmeny.
    \item \textbf{Hodnota} $X(\omega)$ \textbf{NV X závisí} na tom, který \textbf{elementární jev} $\omega$ \textbf{nastal} $\rightarrow$ víme-li, který elementární jev $\omega$ nastal, známe hodnotu $X(\omega)$ NV X.
\end{itemize}
\subsection*{Příklady}
\begin{itemize}
    \item[]	\begin{center}
              $X$... číselný výsledek hodu kostkou\\
          \end{center}
          \textbf{Náhodný pokus}: Hod kostkou. \\
          \textbf{Náhodný jev}: Padne sudé číslo. ($X \in \{2;4;6\}$) \\\\

    \item[]	\begin{center}
              $X$... rychlost připojení k internetu (Mb/s) \\
          \end{center}
          \textbf{Náhodný pokus}: Měření rychlosti připojení k internetu (download). \\
          \textbf{Náhodný jev}: Rychlost připojení k internetu je vyšší než 20 Mb/s.  ($X > 20$)\\\\

    \item[]	\begin{center}
              $X$... počet dívek mezi 1 000 náhodně vybranými dětmi \\
          \end{center}
          \textbf{Náhodný pokus}: Náhodný výběr 1 000 dětí a zjištění počtu dívek mezi nimi. \\
          \textbf{Náhodný jev}: Mezi 1 000 náhodně vybranými dětmi bude více než 500 dívek. ($X > 500$)
\end{itemize}





\section{Distribuční funkce F(x)}
\begin{itemize}
    \item Distribuční funkce F(t) udává pravděpodobnost, že náhodná veličina X bude \textbf{menší než} dané reálné číslo $t$.
          $$F(t) = P(X <t)$$
    \item Distribuční funkce jednoznačně určuje rozdělení NV, tj. známe--li distribuční funkci, umíme určit pravděpodobnost $P(X \in M)$ pro libovolnou $M\subset \mathbb{R}$.
          \begin{figure}[H]
              \centering
              \includegraphics[width=0.6\textwidth]{assets/11_dist_fce}
          \end{figure}
\end{itemize}

\section{Základní typy náhodných veličin}
\begin{itemize}
    \item \textbf{Diskrétní NV} (nabývá spočetně mnoha hodnot)
    \item \textbf{Spojité NV} (jakákoliv hodnota na daném intervalu)
\end{itemize}

\section{Diskrétní náhodná veličina (\uv{DNV})}
\begin{itemize}
    \item Může \textbf{nabývat spočetně mnoha hodnot} (počet dní hospitalizace, počet dní nemocenské, počet zákazníků v lékárně během jednoho dne..).
    \item DNV X s distribuční funkcí $F_x(t)$ je charakterizována \textbf{Pravděpodobnostní funkcí} $P(X = x_i)$, tj. funkcí pro níž platí:
          \begin{equation*}
              F_x(t) = \sum_{x_i<t}P(X = x_i)= \sum_{x_i<t}P(x_i).
          \end{equation*}
\end{itemize}
\subsection{Pravděpodobnostní funkce}
\begin{itemize}
    \item $P(x_i) \geq 0$
    \item $\sum_i P(X = x_i) = 1$
    \item Lze zadat předpisem, tabulkou, grafem.
          \begin{figure}[H]
              \centering
              \includegraphics[width=0.5\textwidth]{assets/11_dnv_prav}
          \end{figure}
    \item[$\Rightarrow$] \textit{Příklad} -- Při ověřování kvality výroby jsou náhodně vybrány dva výrobky a je testována jejich kvalita. Počet vadných výrobků mezi vybranými modelujeme náhodnou veličino X. Z dlouhodobého pozorování jsou známy údaje uvedené v následující tabulce.
          \begin{table}[H]
              \centering
              \begin{tabular}{c|c}
                  \textbf{Počet vadných výrobků} & \textbf{Pravděpodobnost} \\\hhline
                  0                              & 0,25                     \\
                  1                              & 0,50                     \\
                  2                              & 0,25
              \end{tabular}
          \end{table}
          \textit{Určete pravděpodobnostní a distribuční funkci počtu vadných výrobků v testovaném vzorku.}
          \begin{figure}[H]
              \centering
              \includegraphics[width=0.5\textwidth]{assets/11_vztah_prav_dist_dnv}
          \end{figure}
    \item \textbf{Body nespojitosti} distribuční funkce jsou body, v nichž je pravděpodobnostní funkce nenulová.
    \item $P (X= a) = \lim_{x\to a+} F(x) - F(a)$ , tj. velikost \uv{skoku} distribuční funkce v bodech nespojitosti je rovna příslušným hodnotám pravděpodobnostní funkce.
\end{itemize}
\subsection{Distribuční funkce}
\begin{itemize}
    \item $F(t) = \sum_{x_i < t} P(x_i)$
    \item Lze zadat předpisem, tabulkou, grafem.
          \begin{figure}[H]
              \centering
              \includegraphics[width=0.5\textwidth]{assets/11_dnv_dist}
          \end{figure}
    \item Distribuční funkce $F(t)$ \textbf{udává pravděpodobnost}, že náhodná veličina $X$ bude \textbf{menší než} dané reálné číslo $t$. $$F(t) = P(X < t)$$
\end{itemize}
\subsection*{Vlastnosti distribuční funkce}
\begin{itemize}
    \item $0 \leq F(t) \leq 1$,
    \item je neklesající,
    \item je zleva spojitá,
    \item má nejvýše spočetně mnoho bodů nespojitosti,
    \item $F(t) \rightarrow 0 $ pro $ t \rightarrow -\infty $ (\uv{začíná} v 0),
    \item $F(t) \rightarrow 1 $ pro $ t \rightarrow \infty $ (\uv{končí} v 1).
\end{itemize}
\begin{figure}[H]
    \centering
    \includegraphics[width=0.5\textwidth]{assets/11_vztah_prav_dist_dnv2}
\end{figure}
$$P(X = a) \lim_{x \to a+} F(x) - F(a)$$

\section{Spojitá náhodná veličina (\uv{SNV})}

\begin{itemize}
    \item Náhodná veličina $X$ má spojité rozdělení pravděpodobnosti (zkráceně \uv{je spojitá}) právě když má spojitou distribuční funkci.
    \item Mohou \textbf{nabývat všech hodnot na nějakém intervalu} (mají spojitou distribuční funkci) (doba do remise onemocnění, výška, váha, BMO, IQ, vitální kapacita plic, chyba měření...).
    \item SNV X s distribuční funkcí $F_x(t)$ je charakterizována \textbf{hustotou pravděpodobností} $f(x)$, tj. funkcí pro níž platí:
          \begin{equation*}
              F_x(t) = \int_{-\infty}^{t} f(x) dx.
          \end{equation*}
\end{itemize}
\subsection{Hustota pravděpodobnosti f(x)}
\begin{equation*}
    F(x) = \int_{-\infty}^{t} f(x) dx \Rightarrow f(x) = \frac{dF}{dx}
\end{equation*}
\subsection*{Vlastnosti hustoty pravděpodobnosti}
\begin{itemize}
    \item $f(x)$ je reálná nezáporná funkce,
    \item $\int_{-\infty}{\infty} f(x) = 1$ (plocha pod křivkou hustoty je 1),
    \item $\lim_{x \to -\infty} f(x) = 0$ (\uv{začíná v 0}),
    \item $\lim_{x \to \infty} f(x) = 0$ (\uv{končí v 0}).
          \begin{figure}[H]
              \centering
              \includegraphics[width=0.35\textwidth]{assets/11_snv_muze}
              \caption{$f(x)$ může nabývat hodnot vyšších než 1}
          \end{figure}
\end{itemize}
\subsection{Distribuční funkce}
$$F_x(t) = \int_{-\infty}^{t} f(x) dx\, \quad P(X = a) = 0$$
\begin{figure}[H]
    \centering
    \includegraphics[width=0.4\textwidth]{assets/11_dist_fce_snv}
\end{figure}
\begin{figure}[H]
    \centering
    \includegraphics[width=0.5\textwidth]{assets/11_vztah_prav_dist_snv}
    \caption{Vztah mezi pravděpodobností a distribuční funkcí}
\end{figure}

\section{Číselné charakteristiky NV}
\begin{itemize}
    \item Distribuční funkce (pravděpodobnostní funkce, hustota pravděpodobnosti) popisují rozdělení NV jednoznačně, do všech podrobností.
    \item Někdy nás zajímá pouze některý aspekt NV, který se dá popsat jedním číslem:
          \begin{itemize}
              \item očekávaná hodnota NV,
              \item variabilita možných hodnot,
              \item hodnota, pod níž leží pouze malé množství hodnot NV,
              \item šikmost rozdělení,
              \item koncentrace hodnot NV kolem očekávané hodnoty (špičatost rozdělení).\\

              \item \textbf{Obecný moment r--tého řádu} (značí se $\mu_r$ nebo $E(X^r)$, pro r=1,2...) \\pro diskrétní NV: $\qquad \mu_r = \sum_{(i)}x_i^r \cdot P(x_i)$ \\pro spojitou NV: $\qquad \mu_r = \int_{-\infty}^{\infty} x^r \cdot f(x) dx$
              \item \textbf{Střední hodnota} (expected value, mean, značí se jako $\mathbf{E(X)}$ nebo $\mu$) \\pro diskrétní NV: $\qquad E(X) = \mu = \sum_{(i)}x_i \cdot P(x_i)$\\pro spojitou NV: $\qquad E(X) = \mu = \int_{-\infty}^{\infty} x \cdot f(x)dx$
              \item \textbf{Centrální moment r--tého řádu} $\mathbf{\mu_r}$' (značíme $\mu_r$' $= E[(X - E(X))]^r$) \\pro diskrétní NV: $\qquad \mu_r' = \sum_{(i)}(x_i -E(X))^r \cdot P(x_i)$\\pro spojitou NV: $\qquad \mu_r' = \int_{-\infty}^{\infty} (x - E(X))^r \cdot f(x)dx$
              \item \textbf{Rozptyl} (dispersion, variance; značí se $\mu_2'$ nebo $DX$ nebo $\sigma^2$) \\pro diskrétní NV: $\qquad D(X) = \mu_2' = \sum_{(i)}(x_i -E(X))^2 \cdot P(x_i)$\\pro spojitou NV: $\qquad D(X) = \mu_2' = \int_{-\infty}^{\infty} (x - E(X))^2 \cdot f(x)dx$
          \end{itemize}
\end{itemize}

\subsection{Význam střední hodnoty}
Střední hodnotu $E(X)$ náhodné veličiny $X$ lze chápat jako:
\begin{itemize}
    \item průměrnou (očekávanou) hodnotu NV $X$, kolem níž hodnoty NV kolísají,
    \item míru polohy, populační průměr,
    \item vážený průměr všech možných hodnot ($E(X) = \sum_{(i)}x_i \cdot P(x_i)$),
    \item \uv{těžiště} možných hodnot.
          \begin{figure}[H]
              \centering
              \includegraphics[width=0.7\textwidth]{assets/11_stredni_hodnota}
          \end{figure}
\end{itemize}

\subsection{Kvantily}
\textbf{p--kvantil} $\mathbf{x_p}$ (také $100_p\%$--ní kvantil je číslo, pro které platí): $$P(X < x_p) = p.$$
\begin{center}
    $\Rightarrow F(x_p) = p \Rightarrow x_p = F^{-1}(p)$ \\
    (tj. kvantilová funkce $F^{-1}(p)$ je funkcí inverzní k distribuční funkci $F(x_p)$)
    \begin{figure}[H]
        \centering
        \includegraphics[width=0.5\textwidth]{assets/11_kvantily}
    \end{figure}
\end{center}

\noindent Kvantily obvykle určujeme \textbf{pouze pro SNV}. \textbf{Význačné kvantily}:
\begin{itemize}
    \item \textbf{Kvartily}
          \begin{itemize}
              \item[] Dolní kvartil $x_{0,25}$
              \item[] Medián $x_{0,5}$
              \item[] Horní kvartil $x_{0,75}$
          \end{itemize}
    \item \textbf{Decily} -- $x_{0,1}; x_{0,2}; ... ;x_{0,9}$
    \item \textbf{Percentily} -- $x_{0,01}; x_{0,02}; ... ;x_{0,99}$
    \item \textbf{Minimum} $x_{min}$ a \textbf{Maximum} $x_{max}$
\end{itemize}

\subsection{Modus}
\textbf{Modus} $\mathbf{\hat{x}}$ -- typická hodnota náhodné veličiny \\
\textbf{pro diskrétní NV}: $\forall x_i: P(X = \hat{x}) \geq P(X = x_i)$ (tzn. modus je taková hodnota DNV, v níž $P(x_i)$ nabývá svého maxima) \\
\textbf{pro spojité NV}: $\forall  x_i: f(\hat{x}) \geq f(x)$ (tzn. modus je taková hodnota SNV, v níž $f(x)$ nabývá svého maxima)
\begin{itemize}
    \item \textbf{pro diskrétní NV} -- $x_i: P(X = \hat{x}) \geq P(X = x_i)$ (tzn. modus je taková hodnota DNV, v níž $P(x_i)$ nabývá svého maxima).
    \item \textbf{pro spojité NV} -- $ x_i: f(\hat{x}) \geq f(x)$ (tzn. modus je taková hodnota SNV, v níž $f(x)$ nabývá svého maxima).
    \item Modus není těmito podmínkami určen jednoznačně, tzn. náhodná veličina může mít několik modů (např. výsledek hodu kostkou). Má-li NV právě jeden modus, mluvíme o \textbf{unimodálním rozdělení NV}.
    \item Má-li NV unimodální symetrické rozdělení, pak $E(X) = x_{0,5} = \hat{x}$.
\end{itemize}

\subsection{Význam rozptylu}
\begin{itemize}
    \item \textbf{Míra variability dat kolem střední hodnoty}.
    \item Střední kvadratická odchylka od střední hodnoty ($D(X) = E(X - E(X))^2$).
    \item Malý rozptyl $\approx$ hodnoty NV se s vysokou pravděpodobností objevují blízko $E(X)$.
    \item Velký rozptyl $\approx$ hodnoty NV se často objevují ve velké vzdálenosti od $E(X)$.
    \item Jednotka rozptylu je kvadrátem jednotky náhodné veličiny.
          \begin{figure}[H]
              \centering
              \includegraphics[width=0.6\textwidth]{assets/11_rozptyl}
          \end{figure}
\end{itemize}

\subsection{Směrodatná odchylka $\mathbf{\sigma}$}
$$\sigma(Y) = \sqrt{D(X)}$$
Jedná se o odmocninu rozptylu náhodné veličiny. Směrodatná odchylka \textbf{neumožňuje} srovnávat variabilitu náhodných veličin \textbf{měřených v různých jednotkách}!

\subsection{Variační koeficient}
\textbf{Variační koeficient} $\mathbf{\gamma}$ je definován pouze pro \textbf{nezáporné} náhodné veličiny. $$\gamma = \frac{\sigma}{\mu'}, \textrm{ resp. } \frac{\sigma}{\mu'} \cdot 100 [\%]$$
\begin{itemize}
    \item Variační koeficient -- \textbf{směrodatná odchylka v procentech} střední hodnoty.
    \item Čím nižší var. koeficient, tím \textbf{homogennější} soubor.
    \item $V_X > 50\%$ značí silně rozptýlený soubor.
\end{itemize}

\pagebreak
\section{Vybraná rozdělení diskrétní a spojité náhodné veličiny - binomické, hypergeometrické, negativně binomické, Poissonovo, exponenciální, Weibullovo, normální rozdělení.}
\newcommand{\pointing}{\includegraphics[width=1em]{assets/point.png}}
\begin{itemize}
    \item \textbf{Náhodná veličina} -- číselné vyjádření výsledku náhodného pokusu.
    \item Základní typy NV -- \textbf{diskrétní} NV, \textbf{spojitá} NV.
    \item \textbf{Rozdělení pravděpodobnosti} -- Předpis, který jednoznačně určuje všechny pravděpodobnosti typu $P(X \in M)$, kde $M \subset \mathbb{R}$

          (tj. $P (X = a), P (X < a), P(X > a), P(a < X < b), ...,$ kde $a, b \in \mathbb{R}$).
          \begin{itemize}
              \item Rozdělení pravděpodobnosti \textbf{DNV} -- distribuční funkcí $F(x) = P(X < x),$ resp. \textbf{pravěpodobnostní funkcí} $P(x_i)$.
              \item Rozdělení pravděpodobnosti \textbf{SNV} -- distribuční funkcí $F(x) = P(X < x),$ resp. \textbf{hustotou pravěpodobnostní} $f(x)$.
          \end{itemize}
    \item \textbf{Číselné charakteristiky pro popis NV} -- střední hodnota $\mathbf{E(X)}$, rozptyl $\mathbf{D(X)}$, směrodatná odchylka $\mathbf{\sigma(X)}$, p--kvantily $\mathbf{x_p}$
\end{itemize}

\section{Diskrétní náhodná veličina}
\textit{Příklady}
\begin{itemize}
    \item[$\circ$] počet šroubů typu M10 mezi 10 vybranými (víme-li, že do dodávky 100 šroubů bylo omylem zařazeno 20 šroubů typu M50)
    \item[$\circ$] počet pacientů (z 10 očkovaných) u nichž byla použita prošlá očkovací látka (víme-li, že v balení bylo 20 dávek očkovací látky, přičemž 5 z nich bylo prošlých)

          \begin{itemize}
              \item[$\rhd$] \textbf{Obecně:} -- počet úspěchů v $n$ (závislých) pokusech
          \end{itemize}
    \item[$\circ$] počet správně přenesených bitů předtím než dojde ke $4.$ chybě (víme-li, že pravděpodobnost chybného přenosu bitu je $0,12$)
    \item[$\circ$] počet dobrovolníků, které budeme muset testovat dříve než najdeme 5 dárců s krevní skupinou AB (předpokládejme, že dobrovolníci neznají svou krevní skupinu, pravděpodobnost výskytu krevní skupiny (populační frekvence) krevní skupiny AB je $0,05$)
          \begin{itemize}
              \item[$\rhd$] \textbf{Obecně:} -- počet (nezávislých) pokusů do $k.$ úspěchu
          \end{itemize}
    \item[$\circ$] počet škrábanců na $1 m^2$ lakovaného povrchu (víme-li, že průměrně lze očekávat 3 škrábance na $10 m^2$)
    \item[$\circ$] počet červených krvinek v $10 ml$ krve ženy (víme-li, že u průměrně lze pozorovat $4,8 \cdot 10^{12}$ červených krvinek v $1l$ krve (u žen))
          \begin{itemize}
              \item[$\rhd$] \textbf{Obecně:} -- počet událostí v časovém intervalu, na ploše, v objemu
          \end{itemize}
\end{itemize}
\subsection{Bernoulliho pokusy}
\begin{itemize}
    \item \textbf{Posloupnost nezávislých pokusů majících pouze dva možné výsledky} (takových pokusů, kdy úspěch v libovolné skupině pokusů neovlivňuje pravděpodobnost úspěchů v pokusu, který do této skupiny nepatří).
    \item \textbf{v nichž jev A} (úspěch) \textbf{nastává s pravděpodobností } $\mathbf{\pi}$ a neúspěch s pravděpodobností $1 - \pi$.
\end{itemize}

\textit{Příklad:} Předpokládejme, že pravděpodobnost narození dívky je $0,49$. Jaká je pravděpodobnost toho, že mezi čtyřmi dětmi v rodině je právě jedna dívka?
\begin{itemize}
    \item X… počet dívek mezi 4 dětmi
    \item D ... narodí se dívka, $P(D) = \pi$
    \item $\bar{D}$ ...  narodí se kluk, $P(D) = 1 - \pi$
\end{itemize}
\begin{equation*}
    \begin{split}
        &(X = 1) \,... \, \{D\bar{D}\bar{D}\bar{D},\bar{D}D\bar{D}\bar{D},\bar{D}\bar{D}D\bar{D},\bar{D}\bar{D}\bar{D}D\} \\
        &P(D\bar{D}\bar{D}\bar{D}) = P(\bar{D}D\bar{D}\bar{D}) = P(\bar{D}\bar{D}D\bar{D}) = P(\bar{D}\bar{D}\bar{D}D) = \pi \cdot (1 - \pi)^3 \\
        &P(X = 1) = P(D\bar{D}\bar{D}\bar{D} \cup \bar{D}D\bar{D}\bar{D} \cup \bar{D}\bar{D}D\bar{D} \cup \bar{D}\bar{D}\bar{D}D) = \binom{4}{1} \cdot \pi \cdot (1 - \pi)^3 \\
        &P(X = 1) = 0,260
    \end{split}
\end{equation*}

\subsection{Binomické rozdělení}
\begin{itemize}
    \item \textbf{X .. počet úspěchů v $n$ Bernoulliho pokusech} $$X \sim Bi(n;\pi)$$ \\ $n$ -- počet pokusů \\ $\pi$ -- pravděpodobnost úspěchu
          \begin{figure}[H]
              \centering
              \includegraphics[width=0.8\textwidth]{assets/12_binom}
          \end{figure}
    \item \textbf{Střední hodnota}: $\qquad$ $E(X) = n\pi$
    \item \textbf{Rozptyl}: $\qquad\qquad\qquad\;\; D(X) = n\pi(1 - \pi)$
    \item Vlastnosti
          \begin{figure}[H]
              \centering
              \includegraphics[width=0.5\textwidth]{assets/12_binom_vlast}
              \caption{$\pi \not= 0,5$, s rostoucím $n$ se rozdělení stává více a více symetrickým}
          \end{figure}
          \begin{figure}[H]
              \centering
              \includegraphics[width=0.5\textwidth]{assets/12_binom_vlast2}
          \end{figure}
    \item[$\circ$] \textit{Příklad} Mezi 200 vajíčky určenými pro prodej v jisté maloobchodní prodejně je 50 vajíček prasklých. Jaká je pravděpodobnost, že vybereme--li si náhodně 20 vajec, bude 8 z nich prasklých?
          \begin{itemize}
              \item $X$ … počet prasklých vajec z 20 vybraných (výběr bez vracení)

                    \begin{center}
                        \begin{tikzpicture}[sibling distance=8em,
                                every node/.style = { align=center,
                                        top color=white, bottom color=white}]]
                            \node {celkem vajec\\200}
                            child { node {prasklých vajec\\50} }
                            child { node {neprasklých vajec\\150} };
                        \end{tikzpicture}
                    \end{center}
                    $$P(X = 8) = \frac{\binom{50}{8} \binom{150}{12}} {\binom{200}{20}}$$
              \item[] \textbf{počet příznivých možností} -- vybíráme 8 prasklých z 50 prasklých a zároveň 12 ze 150 neprasklých \\ \textbf{počet všech možností }-- vybíráme 20 vajec z 200
          \end{itemize}
\end{itemize}

\subsection{Hypergeometrické rozdělení}
\begin{itemize}
    \item \textbf{ $\mathbf{X}$ ... počet prvků se sledovanou vlastností ve výběru $n$ prvků}
    \item V souboru \textbf{N} prvků je \textbf{M} s danou vlastností a zbylých (\textbf{N -- M}) prvků tuto vlastnost nemá. Postupně vybereme ze souboru \textbf{n} prvků, z nichž žádný \textbf{nevracíme zpět}.
          $$X \sim H(N;M;n)$$ \\ $N$ -- rozsah základního souboru \\ $M$ -- počet prvků s danou vlastností \\ $n$ -- rozsah výběru
    \item \textbf{Pravděpodobnostní funkce}: $P(X = k) = \frac {\mathbin{\color{blue} \binom{M}{k} \binom{N-M}{n-k}}} {\mathbin{\color{magenta}\binom{N}{n}}}$
          \begin{itemize}
              \item \textcolor{blue}{Počet příznívých možností, tj. počet možností jak vybrat $k$ prvků s danou vlastností z $M$ a zároveň $(n-k)$ prvků, které uvedenou vlastnost nemají z $(N - M)$ prvků.}
              \item \textcolor{magenta}{Počet všech možností, jak vybrat $n$ prvků z $N$ (nezáleží na pořadí).}
          \end{itemize}
    \item \textbf{Střední hodnota}: $\qquad$ $E(X) = n \cdot \frac{M}{N}$
    \item \textbf{Rozptyl}: $\qquad\qquad\qquad\;\; D(X) = n \cdot \frac{M}{N}(1 - \frac{M}{N})(\frac{N-n}{N-1})$
    \item \textbf{Možnost aproximace} -- je--li $n/N$, tzv. \textbf{výběrový poměr}, menší než $0,05$, lze hypergeometrické rozdělení nahradit binomickým s parametry $n$ a $M/N$. $$(\frac{n}{N} < 0,05) \Rightarrow [H(N;M;n) \approx Bi(n;\frac{M}{N})]$$
\end{itemize}

\subsection{Negativně binomické (Pascalovo) rozdělení}
\begin{itemize}
    \item \textbf{$X$ ... počet Bernoulliho pokusů do $k.$ výskytu události (úspěchu), včetně $k.$ výskytu}
          $$X \sim NB(k;\pi)$$ \\ $k$ -- požadovaný počet úspěchů (výskytu události) \\ $\pi$ -- pravděpodobnost úspěchu
    \item \textbf{Pravděpodobnostní funkce:}
          \begin{figure}[H]
              \centering
              \includegraphics[width=0.7\textwidth]{assets/12_neg_binom}
          \end{figure}
    \item \textbf{Střední hodnota}: $\qquad$ $E(X) = \frac{k}{\pi}$
    \item \textbf{Rozptyl}: $\qquad\qquad\qquad\;\; D(X) = \frac{k(1- \pi)}{\pi^2}$
    \item V případě negativně binomické náhodné veličiny není definice ustálená. Někteří statistici (popř. statistický software) ji definují jako \textbf{počet neúspěchů před $k.$ úspěchem}.
\end{itemize}

\subsection{Poissonův proces}
\textbf{Bodový proces}
\begin{itemize}
    \item Sledujeme chod procesu, v němž čas od času dochází k nějaké význačné události \\
          \begin{figure}[H]
              \centering
              \includegraphics[width=0.5\textwidth]{assets/12_poisson_proces}
          \end{figure}
    \item Registrujeme počet událostí $N(s;t)$ v časovém intervalu $<s;s + t>$
    \item \textbf{Rychlost výskytu události $\lambda$} je parametrem Poissonova procesu
    \item Poissonův proces lze obdobně jako v časovém intervalu definovat na libovolné uzavřené prostorové oblasti (na ploše, v objemu).
\end{itemize}
Jako \textbf{Poissonův proces} označujeme proces, který je:
\begin{itemize}
    \item \textbf{ordinální} -- pravděpodobnost výskytu více než jedné události v limitně krátkém časovém intervalu $(t \to 0)$ je nulová. (tzv. \textbf{řídké jevy}),
    \item \textbf{stacionální} -- rychlost výskytu událostí $\lambda$ je konstantní v průběhu sledovaného intervalu,
    \item \textbf{beznásledný} -- pravděpodobnost výskytu událostí není závislá na čase, který uplynul od minulé události.
\end{itemize}

\subsection{Poissonovo rozdělení}
\begin{itemize}
    \item \textbf{$X$ ... počet výskytu události v uzavřené oblasti (v čase, na ploše, v objemu)}
    \item náhodný pokus jako Poissonův proces (nezávislé události probíhající v čase $t$, s rychlostí výskytu $\lambda$; popř. nezávislé události objevující se na ploše $t$, resp. v objemu $t$ s hustotou výskytu $\lambda$).
          $$X \sim Po(\lambda t)$$
    \item \textbf{Pravděpodobnostní funkce}: $P(X = k) = \frac{(\lambda t)^k}{k!}e^{-\lambda t}$
    \item \textbf{Střední hodnota}: $\qquad$ $E(X) = \frac{k}{\pi}$
    \item \textbf{Rozptyl}: $\qquad\qquad\qquad\;\; D(X) = \frac{k(1- \pi)}{\pi^2}$
\end{itemize}


\section{Spojitá náhodná veličina}
\begin{itemize}
    \item Hustota pravděpodobnosti f(x) -- funkce $f(x)$ je \textbf{hustotou pravděpodobnosti} (na intervalu $a \leq x \leq b$), jestliže splňuje následující podmínky:
          \begin{itemize}
              \item $f(x) \geq 0; x \in \mathbb{R}$
              \item plocha pod křivkou hustoty je rovna 1 ($\int_{-\infty}^{\infty} f(x)dx = 1$).
          \end{itemize}
\end{itemize}

\subsection{Exponenciální rozdělení}
\begin{itemize}
    \item \textbf{$X$ ... délka časových intervalů mezi událostmi v Poissonově procesu}
          $$X \sim Exp(\lambda)$$
    \item Předpokládá konstantní intenzitu události $\lambda(t)$- rozdělení \uv{bez paměti}.
    \item \textbf{Hustota pravděpodobnosti}:
          $f(t) =  \begin{cases}
                  \lambda \cdot e^{-\lambda t} & \quad t > 0     \\
                  0,                           & \quad t \leq 0.
              \end{cases}$
    \item \textbf{Distribuční funkce}:
          $\qquad\qquad F(t) =  \begin{cases}
                  1 -  e^{-\lambda t} & \quad t > 0     \\
                  0,                  & \quad t \leq 0.
              \end{cases}$
    \item \textbf{Střední hodnota}: $\qquad\qquad\quad E(X) = \frac{1}{\lambda}$
    \item \textbf{Rozptyl}: $\qquad\qquad\qquad\qquad\quad D(X) = (E(X))^2 = \frac{1}{\lambda^2}$ \\
    \item \textbf{Riziková funkce} (intenzita poruch) $\lambda (t)$
          \begin{itemize}
              \item Pro nezápornou náhodnou veličinu $X$ se spojitým rozdělením popsaným distribuční funkcí $F(t)$ definujeme pro $F(t) \not= 1$ rizikovou funkci jako  $$\lambda(t) = \frac{f(t)}{1 - F(t)}.$$
              \item $\lambda(t)$ -- Představuje-li náhodná veličina X dobu do poruchy nějakého zařízení, pak pravděpodobnost, že pokud do času t nedošlo k žádné poruše, tak k ní dojde v následujícím krátkém úseku délky $\Delta t$, je přibližně
                    $$P(t \geq X <t + \Delta t|X > t) = \lambda(t) \cdot \Delta t$$
                    \begin{figure}[H]
                        \centering
                        \includegraphics[width=0.5\textwidth]{assets/12_exp_rizk_fce}
                    \end{figure}
              \item[$\circ$] \textit{Příklad} Střední doba čekání zákazníka na obsluhu v prodejně je 50 sekund. Doba čekání se řídí exponenciálním rozdělením (pravděpodobnost, že zákazník nebude obsloužen klesá s rostoucím časem exponenciálně). Jaká je pravděpodobnost, že náhodný zákazník bude obsloužen dříve než za 30 sekund?
                    \begin{itemize}
                        \item[] X -- doba čekání na obsluhu
                        \item[] $X \sim Exp (\lambda = \frac{1}{50}s^-1), E(X) = \frac{1}{\lambda} = 50 s$
                    \end{itemize}
          \end{itemize}
\end{itemize}

\subsection{Weibullovo rozdělení}
\begin{itemize}
    \item \textbf{X ... délka časových intervalů mezi událostmi v Poissonově procesu}
    \item Zobecnění exponenciálního rozdělení
    \item Tímto rozdělením lze modelovat i dobu do výskytu události u systémů (jedinců), které jsou v období dětských nemocí, resp. období stárnutí.
          $$X \sim W(\theta;\beta)$$
          \\ $\theta$ -- parametr měřítka (scale) $\theta = \frac{1}{\lambda}; \theta > 0$ \\ $\beta$ -- parametr tvaru (shape); $\beta > 0$
    \item \textbf{Hustota pravděpodobnosti}:
          $f(t) =  \begin{cases}
                  \beta\lambda^{\beta} t^{\beta-1}e^{-(\lambda t)^{\beta}}, & \quad t > 0     \\
                  0,                                                        & \quad t \leq 0.
              \end{cases}$
    \item \textbf{Distribuční funkce}:
          $\qquad\qquad F(t) =  \begin{cases}
                  1 -  e^{-(\lambda t)^{\beta} } & \quad t > 0     \\
                  0,                             & \quad t \leq 0.
              \end{cases}$
    \item \textbf{Riziková funkce}:
          $\qquad\qquad\quad\; \lambda(t) =  \begin{cases}
                  \beta\lambda^\beta t^{\beta-1} & \quad t > 0     \\
                  0,                             & \quad t \leq 0.
              \end{cases}$
          \begin{figure}[H]
              \centering
              \includegraphics[width=0.5\textwidth]{assets/12_weib_riz_fce}
              \caption{Riziková funkce}
          \end{figure}
\end{itemize}
\subsection{Normální rozdělení}
\begin{itemize}
    \item Bývá vhodné k popisu náhodných veličin, které lze interpretovat jako aditivní výsledek mnoha nepatrných a vzájemně nezávislých faktorů (výška člověka, IQ, délka končetiny).
    \item Popisuje náhodné veličiny, jejichž hodnoty se symetricky shlukují kolem střední hodnoty a vytvářejí tak charakteristický tvar hustoty pravděpodobnosti známý pod názvem \textbf{Gaussova křivka}. $$X \sim N (\mu;\sigma^2)$$  \\$\mu$ -- střední hodnota \\ $\sigma^2$ -- rozptyl
    \item \textbf{Hustota pravděpodobnosti}: $f(x) = \frac{1}{\sigma \sqrt{2\pi}} e^{-\frac{1}{2}({\frac{x - \mu}{\sigma}})^2}$
          \begin{figure}[H]
              \centering
              \includegraphics[width=0.5\textwidth]{assets/12_norm_roz}
              \caption{Riziková funkce}
          \end{figure}
    \item \textbf{Distribuční funkce}: $F(x) = \frac{1}{\sigma \sqrt{2\pi}} \int_{-\infty}^{x} e^{-\frac{1}{2}(\frac{t - \mu}{\sigma})^2} dt$\\ (integrál nelze řešit analyticky)
\end{itemize}

%BTICH END

\pagebreak
\section{Popisná statistika. Číselné charakteristiky a vizualizace kategoriálních a kvantitativních proměnných.}
\textbf{Popisná statistika} zjišťuje a \textbf{sumarizuje} informace, zpracovává je ve formě grafů a tabulek a vypočítává jejich \textbf{číselné charakteristiky} jako průměr, rozptyl percentily, rozpětí a pod.

\section{Kvantitativní -- Numerická proměnná}
\begin{enumerate}
    \item \textbf{Míry polohy} --  určující typické rozložení hodnot proměnné (jejich rozmístění na číselné ose).

          \begin{itemize}
              \item Průměr -- $\bar{x} = \frac{\sum\limits_{i=1}^n x_i}{n}$
              \item Modus -- střed shorthu
              \item Kvantily -- dolní kvartil, médián, horní kvartil,..
          \end{itemize}
    \item \textbf{Míry variability} -- určující variabilitu (rozptyl) hodnot kolem své typické polohy.
          \begin{itemize}
              \item Variační rozpětí $x_{max} - x_{min}$
              \item Inverkvartilové rozpětí -- $IQR = x_{0,75} - x_{0,25}$
              \item Výběrový rozptyl
              \item Výběrová směrodatná odchylka
              \item Variační koeficient
          \end{itemize}
    \item \textbf{Míra šikmosti a špičatosti}
          \begin{itemize}
              \item Výběrová šikmost
              \item Výběrová špičatost
          \end{itemize}
    \item \textbf{Identifikace odlehlých pozorování}
          \begin{itemize}
              \item Vnitřní hradby dolní mez: $h_D = x_{0,25} - 1,5IQR$, horní mez: $h_H = x_{0,75} + 1,5IQR$
              \item Vnější hradby dolní mez: $h_D = x_{0,25} - 3IQR$, horní mez: $h_H = x_{0,75} + 3IQR$
              \item Z--souřadnice
              \item Mediánová souřadnice
          \end{itemize}
\end{enumerate}

\subsection{Grafické zobrazení numerické proměnné}
\begin{itemize}
    \item Empirická distribuční funkce
    \item Krabicový graf (box plot)
    \item Číslicový histogram (lodyha s listy, steam and leaf)
\end{itemize}

\section{Kvalitativní -- Kategoriální proměnná}
\begin{enumerate}
    \item \textbf{Nominální proměnná} -- nemá smysl uspořádaní.
          \begin{itemize}
              \item \textbf{Základní statistiky pro popis nominální proměnné:}
                    \begin{itemize}
                        \item četnost
                        \item relativní četnost
                        \item modus
                    \end{itemize}
              \item \textbf{Grafické zobrazení nominální proměnné:}
                    \begin{itemize}
                        \item histogram
                        \item výsečový graf
                    \end{itemize}
          \end{itemize}
    \item \textbf{Ordinální proměnná} -- má smysl uspořádání.
          \begin{itemize}
              \item \textbf{Základní statistiky pro popis ordinální proměnné:}
                    \begin{itemize}
                        \item četnost
                        \item relativní četnost
                        \item kumulativní četnost
                        \item relativní kumulativní četnost
                        \item modus
                    \end{itemize}
              \item \textbf{Grafické zobrazení ordinální proměnné:}
                    \begin{itemize}
                        \item histogram
                        \item výsečový graf
                        \item Lorenzova křivka
                        \item Paretův graf
                    \end{itemize}
          \end{itemize}
\end{enumerate}
\textbf{Paretův princip} – 80\% následků pramení z 20\% příčin.\\
\textbf{Paretova analýza} – postup vedoucí k nalezení \uv{životně důležité menšiny} (spektra příčin ovlivňujících rozhodujícím způsobem následky).

\section{Míry polohy a variability}
Snad nejpoužívanějšími mírami polohy jsou \textbf{průměry} proměnných. Průměry představují průměrnou nebo typickou hodnotu výběrového souboru. Zřejmě nejznámějším průměrem pro kvantitativní proměnnou je \textbf{aritmetický průměr}.
\begin{itemize}
    \item \textbf{Aritmetický průměr} $\mathbf{\bar{x}}$ (mean)
    \item \textbf{Aritmetický vážený průměr}
    \item \textbf{Harmonický průměr}  -- Pro výpočet průměru v případech, kdy proměnná má charakter části z celku (úlohy o společné práci, ...).
    \item \textbf{Harmonický vážený průměr} -- Pokud máme údaje setříděné do tabulky četností.
    \item \textbf{Geometrický průměr} -- Pracujeme-li s kladnou proměnnou představující relativní změny (růstové indexy, cenové indexy...). $\bar{x}_G = \sqrt[n]{x_1 \cdot x_2 \cdot ... \cdot x_n}$
    \item \textbf{Geometrický vážený průměr} -- Pracujeme-li s kladnou proměnnou představující relativní změny (růstové indexy, cenové indexy...). $\bar{x}_G = \sqrt[n]{x_1^{n_1} \cdot x_2^{n_2} \cdot ... \cdot x_n^{n_k}}$ kde $n = \sum\limits_{i=1}^k n_i$. \\

    \item \textbf{Modus} -- Pro \textbf{diskrétní proměnnou} definujeme modus jako \textbf{hodnotu nejčetnější varianty proměnné} (podobně jako u kvalitativní proměnné), u \textbf{spojité} proměnné považujeme za modus $\mathbf{\hat{x}}$ hodnotu kolem níž je největší koncentrace hodnot proměnné. Mnohdy mluvíme o typické hodnotě proměnné.

          Pro určení této hodnoty využijeme tzv. \textbf{short}, což je nejkratší interval, v němž leží alespoň 50\% hodnot proměnné (v případě výběru o rozsahu $n = 2k(k \in N)$ (sudý počet hodnot), leží v shorthu $k$ hodnotě - což je 50\% $(n/2)$ hodnot proměnné, v případě výběru o rozsahu $n = 2k + 1(k \in N)$ (lichý počet hodnot), leží v shorthu $k + 1$ hodnot - což je o 1 více než je 50\% hodnot proměnné). \textbf{Modus pak definujeme jako střed shorthu}.

    \item \textbf{Výběrové kvantily} (quantile, resp. percentile) -- jsou to statistiky, které charakterizují polohu jednotlivých hodnot v rámci proměnné. Podobně jako modus, jsou i výběrové kvantily \textbf{rezistentní} (odolné) vůči odlehlým pozorováním.

          Obecně je výběrový kvantil chápán jako hodnota, která rozděluje výběrový soubor na dvě části -- hodnoty, které jsou menší než daný kvantil, druhá část obsahuje hodnoty, které jsou větší nebo rovno danému kvantilu. Pro určení kvantilu je nutné \textbf{výběr uspořádat} od nejmenší hodnoty k největší. \textbf{Kvantily}:
          \begin{itemize}
              \item \textbf{Dolní kvartil} $x_{0,25}$ -- $25\%$--ní kvartil  (rozděluje datový soubor tak, že 25\% hodnot je menších než tento kvartil a zbytek, tj. 75\% větších (nebo rovných))
              \item \textbf{Medián} $x_{0,5}$ -- $50\%$--ní kvartil
              \item \textbf{Horní kvartil} $x_{0,75}$ -- $75\%$--ní kvartil\\
                    Kvartily dělí výběrový soubor na 4 přibližně stejně velké části.
              \item \textbf{Decily} -- $x_{0,1};x_{0,2};...;x_{0,9}$ -- Decily dělí výběrový soubor na 10 přibližně stejně četných částí.
              \item \textbf{Percentily} -- $x_{0,01};x_{0,02};...;x_{0,99}$ -- Percentily dělí výběrový soubor na 100 přibližně stejně četných částí.
          \end{itemize}

    \item \textbf{Empirická distribuční funkce F(x) pro kvantitativní proměnnou}
          \begin{itemize}
              \item Empirická distribuční funkce je monotónně rostoucí, zleva spojitou funkcí, která \uv{skáče} podle relativních četností příslušných jednotlivým hodnotám proměnné.
              \item Označme si $p(x_i)$ relativní četnost hodnoty xi seřazeného výběrového souboru $x_1 < x_2 < ... < x_n$. Pro empirickou distribuční funkci $F(x)$ pak platí:
                    \begin{figure}[H]
                        \centering
                        \includegraphics[width=0.6\textwidth]{assets/13_empiricka}
                    \end{figure}
          \end{itemize}

    \item \textbf{Interkvartilové rozpětí IQR} -- Tato statistika je mírou variability souboru a je definována jako vzdálenost mezi horním a dolním kvartilem.
    \item \textbf{MAD} (median absolute deviation from the median) -- medián absolutních odchylek od mediánu.
          \begin{enumerate}
              \item Výběrový soubor uspořádáme podle velikosti.
              \item Určíme medián souboru.
              \item Pro každou hodnotu souboru určíme absolutní hodnotu její odchylky od mediánu.
              \item Absolutní odchylky od mediánu uspořádáme podle velikosti.
              \item Určíme medián absolutních odchylek od mediánu, tj. MAD.
          \end{enumerate}
    \item \textbf{Výběrový rozptyl} $\mathbf{s^2}$ (\uv{s kvadrát}, sample variance) --
          \begin{itemize}
              \item  je nejrozšířenější mírou variability výběrového souboru
                    \begin{equation*}
                        s^2 = \frac{\sum\limits_{i=1}^n (x_1 - \bar{x})^2} {n - 1}
                    \end{equation*}
              \item Výběrový rozptyl je dán podílem součtu kvadrátu odchylek jednotlivých hodnot od průměru a rozsahu souboru sníženého o jedničku.
              \item Nevýhodou použití výběrového rozptylu jakožto míry variability je to, že jednotka této charakteristiky je \textbf{druhou mocninou} jednotky proměnné. Např. je--li proměnnou denní tržba uvedena v Kč, bude výběrový rozptyl této proměnné vyjádřen v Kč$^2$.
              \item Následující míra variability tuto vlastnost nemá.
          \end{itemize}
    \item \textbf{Výběrová směrodatná odchylka s} -- je definována jako kladná odmocnina výběrového rozptylu
          \begin{equation*}
              s = \sqrt{s^2} = \sqrt{\frac{\sum\limits_{i=1}^n (x_1 - \bar{x})^2} {n - 1}}.
          \end{equation*}
          Nevýhodou výběrového rozptylu i výběrové směrodatné odchylky je skutečnost, že neumožňují porovnávat varibilitu proměnných vyjádřených v různých jednotkách. Která proměnná má větší variabilitu – výška nebo hmotnost dospělého člověka? Na tuto otázku nám dá odpověď tzv. variační koeficient.
    \item \textbf{Variační koeficient} $\mathbf{V_x}$ -- vyjadřuje relativní míru variability proměnné x. Podle níže uvedeného vztahu jej lze stanovit pouze pro proměnné, které nabývají výhradně kladných hodnot. Variační koeficient je bezrozměrný. Uvádíme-li jej v [\%], hodnotu získanou z definičního vzorce vynásobíme 100\%.
          \begin{equation*}
              V_x = \frac{V}{\bar{x}}, popr. V_x = \frac{V}{\bar{x}} \cdot 100 [\%].
          \end{equation*}
\end{itemize}

\section{Identifikace odlehlých pozorování}
\begin{itemize}
    \item \textbf{Vnitřní hradby} -- Za odlehlé pozorování lze považovat takovou hodnotu $x_i$ , která je od dolního, resp. horního kvartilu vzdálená více než 1,5 násobek interkvartilového rozpětí. Tedy: $[(x_i < x_{0,25} - 1,5 \cdot IQR) \vee (x_i > x_{0,75} + 1,5 \cdot IQR)] \Rightarrow x_i$ je odlehlým pozorováním.
    \item \textbf{z--souřadnice (z--skóre)} Za odlehlé pozorování lze považovat takovou hodnotu $x_i$, jejíž absolutní hodnota z-souřadnice je větší než 3, tj. hodnota, která je od průměru vzdálenější než 3s. Tedy: $z-skore_i = \frac{x_i - \bar{x}}{s}$ \\
          $|z-skore_i| > 3 \Rightarrow |\frac{x_i - \bar{x}}{s}| >3s \Rightarrow x_i$ je odlehlým pozorováním.

    \item $\mathbf{x_{0,5}}$\textbf{--souřadnice ($\mathbf{x_{0,5}}$--skóre)} -- : Za odlehlé pozorování lze považovat takovou hodnotu $x_i$, jejíž absolutní hodnota mediánové souřadnice je větší než 3, tj. hodnota, která je od mediánu vzdálenější než $3 \cdot 1,483\cdot \textrm{MAD}$. Tedy: $x_{0,5}-\textrm{skore}_i = \frac{x_i - x_{0,5}}{1,483\textrm{MAD}}$

          $|x_{0,5}-\textrm{skore}_i| > 3 \Rightarrow |\frac{x_i - x_{0,5}}{1,483\textrm{MAD}}| >3 \Rightarrow |x_i - x_{0,5}| >3 \cdot 1,483\textrm{MAD} \Rightarrow x_i$ je odlehlým pozorováním.
\end{itemize}

\section{Míra šikmosti a špičatosti}
\begin{itemize}
    \item \textbf{Výběrová šikmost a} (skewness) -- vyjadřuje asymetrii rozložení hodnot proměnné kolem jejího průměru.
          \begin{itemize}
              \item $a = 0$ -- hodnoty proměnné jsou kolem jejího průměru rozloženy symetricky
              \item $a > 0$ -- u proměnné převažují hodnoty menší než průměr
              \item $a < 0$ -- u proměnné převažují hodnoty větší než průměr
                    \begin{figure}[H]
                        \centering
                        \includegraphics[width=0.6\textwidth]{assets/13_sikmost}
                    \end{figure}
          \end{itemize}
    \item \textbf{Souvislost mezi šikmostí a charakteristikami polohy}
          \begin{itemize}
              \item Symetrické rozdělení: $\bar{x} = x_{0,5}$
              \item Pozitivně zešikmené rozdělení: $\bar{x} > x_{0,5}$
              \item Negativně zešikmené rozdělení: $\bar{x} < x_{0,5}$
          \end{itemize}
    \item \textbf{Výběrová špičatost b} (kurtosis) -- vyjadřuje koncentraci hodnot proměnné kolem jejího průměru.
          \begin{itemize}
              \item $b = 0$ -- špičatost odpovídá normálnímu rozdělení (bude definováno později)
              \item $b > 0$ -- špičaté rozdělení proměnné
              \item $b < 0$ -- ploché rozdělení proměnné
                    \begin{figure}[H]
                        \centering
                        \includegraphics[width=0.6\textwidth]{assets/13_spicatost}
                    \end{figure}
          \end{itemize}
\end{itemize}

\section{Grafické znázornění kvalitativní proměnné}
\begin{itemize}
    \item \textbf{Krabicový graf} (Box plot)
          \begin{itemize}
              \item Odlehlá pozorování jsou znázorněna jako izolované body, konec horního (popř. konec dolního) vousu představují maximum (popř. minimum) proměnné po vyloučení odlehlých pozorování,\uv{víko} krabice udává horní kvartil, \uv{dno} dolní kvartil, vodorovná úsečka uvnitř krabice označuje medián.
              \item Z polohy mediánu vzhledem ke \uv{krabici} lze dobře usuzovat na symetrii vnitřních 50\% dat a my tak získáváme dobrý přehled o středu a rozptýlenosti proměnné.
          \end{itemize}
          \begin{figure}[H]
              \centering
              \includegraphics[width=0.5\textwidth]{assets/13_box_plot}
          \end{figure}
    \item \textbf{Číslicový histogram} (Lodyha s listy, angl. Stem and leaf plot)
          \begin{itemize}
              \item Jak jsme si ukázali, výhodou krabicového grafu je jeho jednoduchost, někdy nám však chybí informace o konkrétních hodnotách proměnné.
              \item Chtěli bychom proto nějak přehledně zapsat číselné hodnoty výběru a k tomu nám slouží právě číslicový histogram.
              \item Navíc nám tento graf dává dobrou představu o šikmosti proměnné.
              \item Příklad: \textit{Představme si proměnnou představující průměrné měsíční platy zaměstnanců ve státní správě}.
                    \begin{table}[H]
                        \centering
                        \begin{tabular}{|l|}
                            \hline
                            \textbf{Průměrný měsíční plat [Kč]}                                       \\ \hline
                            10 654, 9 765, 8 675, 12 435, 9 675, 10 343, 18 786, 15 420, 8 675,	7 132, \\
                            6 732,6 878, 15 657, 9 754, 9 543, 9 435, 10 647, 12 453, 9 987, 10 342.  \\ \hline
                        \end{tabular}
                    \end{table}
                    \begin{figure}[H]
                        \centering
                        \includegraphics[width=0.6\textwidth]{assets/13_cislicovy_hist}
                    \end{figure}
                    \begin{itemize}
                        \item Pro naší informaci nejsou tak důležité koruny ani desetikoruny rozdílu. V tomto případě se nám jedná přinejmenším o stokoruny.
                        \item Co kdybychom tedy informaci o \uv{nedůležitých} řádech zanedbali a znázornili setříděná data pouze na základě vyšších řádů? My jsme se rozhodli, že důležitý řád jsou pro nás stokoruny.
                        \item Hodnoty stojící o řád výš (v našem případě tisíce) zapíšeme setříděné pod sebe, tak, že tvoří jakýsi stonek (\textbf{lodyhu}), přičemž pod graf uvedeme tzv. \textbf{šířku lodyhy}, která udává koeficient, jímž se hodnoty uvedené v grafu násobí.
                        \item Druhý sloupec grafu, \textbf{listy}, budou tvořit číslice, reprezentujíci zvolený \uv{důležitý} řád, zapisované do příslušných řádků (opět seřazené podle velikosti).
                        \item Třetí sloupec udává absolutní četnosti příslušné daným řádkům.
                    \end{itemize}
          \end{itemize}
\end{itemize}

\section{Statistické charakteristiky kvalitativních (kategoriálních) proměnných}
\subsection{Nominální proměnná}
Nominální proměnná nabývá v rámci souboru \textbf{různých}, \textbf{avšak rovnocenných kategorií}. Počet těchto kategorií nebývá příliš vysoký, a proto první statistickou charakteristikou, kterou k popisu proměnné použijeme je četnost.
\begin{itemize}
    \item \textbf{Četnost} $\mathbf{n_i}$ (absolutní četnost, \uv{frequency}) --  je definována jako počet výskytu dané varianty kvalitativní proměnné. V případě, že kvalitativní proměnná ve statistickém souboru o rozsahu $n$ hodnot  nabývá $k$ různých variant, jejichž četnosti označíme $n_1, n_2, ... , n_k$, musí zřejmě platit $n_1 + n_2 + ... + n_k = \sum\limits_{i=1}^k n_i = n$. \\
          Chceme-li vyjádřit, jakou část souboru tvoří proměnné s některou variantou, použijeme pro popis proměnné relativní četnost.

    \item \textbf{Relativní četnost} $\mathbf{p_i}$ (\uv{relative frequency}) je definována jako $p_i = \frac{n_i}{n}$, popř. $p_i = \frac{n_i}{n} \cdot 100 [\%]$. Pro relativní četnosti musí platit $p_1 + p_2 + ... + p_k = \sum\limits_{i=1}^k p_i = p$. \\
          Při zpracování kvalitativní proměnné je vhodné četnosti i relativní četnosti uspořádat do tzv. \textbf{tabulky rozdělení četnosti} (\uv{frequency table})
          \begin{figure}[H]
              \centering
              \includegraphics[width=0.6\textwidth]{assets/13_tabl_cetnost}
          \end{figure}
    \item \textbf{Modus} -- s definujeme jako název varianty proměnné vykazující nejvyšší četnost.\\
    \item \textbf{Grafické znázornění nominální proměnné}
          \begin{itemize}
              \item \textbf{Histogram} -- je klasickým grafem, v němž na jednu osu vynášíme varianty proměnné a na druhou osu jejich četnosti. Jednotlivé hodnoty četností jsou pak zobrazeny jako výšky sloupců (obdélníků, popř. hranolů, kuželů...)
              \item \textbf{Výsečový graf} -- prezentuje relativní četnosti jednotlivých variant proměnné, přičemž jednotlivé relativní četnosti jsou úměrně reprezentovány plochami příslušných kruhových výsečí. (Změnou kruhu na elipsu dojde k trojrozměrnému efektu.)
          \end{itemize}
\end{itemize}
\subsection{Ordinální proměnná}
Ordinální proměnná, stejně jako proměnná nominální, nabývá v rámci souboru různých slovních variant, avšak tyto varianty mají \textbf{přirozené uspořádání}, tj. můžeme určit, která je \uv{menší} a která \uv{větší}.
\begin{itemize}
    \item \textbf{Kumulativní četnost} $\mathbf{m_i}$ i (\uv{cumulative frequency}) -- definujeme jako počet hodnot proměnné, které nabývají varianty nižší nebo rovné i-té variantě.

          \textit{Uvažte např. proměnnou \uv{známka ze statistiky}, která nabývá variant: \uv{výborně}, \uv{velmi dobře}, \uv{prospěl}, \uv{neprospěl}, pak např. kumulativní četnost pro variantu \uv{prospěl} bude rovna počtu studentů, kteří ze statistiky získali známku \uv{prospěl} nebo lepší.}

          Jsou-li jednotlivé varianty uspořádány podle své \uv{velikosti}(\uv{$x_1 < x_2 < ... < x_k$}), platí $m_i = \sum\limits_{i=1}^i n_j$, Je tedy zřejmé, že kumulativní četnost $k$--té (\uv{nejvyšší}) varianty je rovna rozsahu proměnné $– mk = n$.

    \item \textbf{Kumulativní relativní četnost} $\mathbf{F_i}$ (\uv{cumulative relative frequency}) -- vyjadřuje jakou část souboru tvoří hodnoty nabývající i-té a nižší varianty. $F_i = \sum\limits_{j=1}^i p_j,$ což není nic jiného než relativní vyjádření kumulativní četnosti: $F_i = \frac{m_i}{n}$.

          Obdobně jako pro nominální proměnné, můžeme i pro proměnné ordinální prezentovat statistické charakteristiky pomocí tabulky rozdělení četnosti. Ta obsahuje ve srovnání s tabulkou rozdělení četností pro nominální proměnnou navíc hodnoty kumulativních a kumulativních relativních četností.
          \begin{figure}[H]
              \centering
              \includegraphics[width=0.8\textwidth]{assets/13_tabl_cetnost_ord}
          \end{figure}
    \item \textbf{Grafické znázornění ordinální proměnné}
          \begin{itemize}
              \item \textbf{Lorenzova křivka} ((polygon kumulativních četností, Galtonova ogiva, S křivka) -- S křivka) je spojnicovým grafem, který získáme tak, že na vodorovnou osu vynášíme jednotlivé varianty proměnné v pořadí od \uv{nejmenší} do \uv{největší} a na svislou osu příslušné hodnoty kumulativních četností. Znázorněné body spojíme úsečkami.
                    \begin{figure}[H]
                        \centering
                        \includegraphics[width=0.5\textwidth]{assets/13_lorenzo}
                    \end{figure}
              \item \textbf{Paretova analýza} -- lze formulovat tak, že 80\% následků pramení z 20\% příčin (20\% lidí vlastní 80\% celkového bohatství). V praxi pak bývá snahou nalézt toto malé spektrum příčin (životně důležitá menšina), které tak významně ovlivňuje výsledek.
          \end{itemize}
\end{itemize}


\pagebreak
\section{Metody statistické indukce. Intervalové odhady. Princip testování hypotéz.}
%Metody statistické indukce. Intervalové odhady. Princip testování hypotéz.

\section{Statistická indukce}
Statistická indukce je metoda, která dovoluje stanovit vlastnost celku (\textbf{základního souboru}) na základě pozorování jeho částí (\textbf{náhodného výběru}).
\begin{figure}[H]
    \centering
    \includegraphics[width=0.6\textwidth]{assets/14_stat_ind}
\end{figure}

\subsection{Základní soubor (populace)}
\begin{itemize}
    \item Je množina všech teoreticky možných objektů (např. jedinců) v uvažované situaci = statistický soubor, který je vymezen cílem výzkumu a pro který vyvozujeme závěry výzkumného šetření.
    \item Charakterizuje se \textbf{parametrem}, což je např. výška, váha, IQ, atp.
    \item Má konečný nebo nekonečný (hypotetický) \textbf{rozsah}, který je dán N (např.: N = 150 lidí, opic, rostlin,...).
\end{itemize}
\subsection{Výběrový soubor (výběr)}
\begin{itemize}
    \item Je část populace vybrané na základě předem stanovených kritérii resp. pravidel (podmnožina základního souboru).
    \item \textbf{O náhodném výběru} uvažujeme, když splňuje dvě základní vlastnosti:
          \begin{itemize}
              \item \textbf{pravděpodobnost} zařazení do vzorku je pro všechny statistické jednotky populace \textbf{nenulová},
              \item statistické jednotky jsou do vzorku vybrané \textbf{nezávisle} jedna od druhé.
          \end{itemize}
    \item \textbf{O reprezentativním výběru} uvažujeme, když výběrový soubor dobře odráží strukturu celého zkoumaného souboru.
\end{itemize}
\subsection{Principy statistického usuzování}
\begin{enumerate}
    \item Statistické usuzování znamená zobecňování z výběrových statistik na parametry rozdělení.
    \item Abychom mohli provést statistické usuzování, musíme mít nějakou teorii, jež popisuje náhodné chování sledovaných proměnných.
    \item Existují dva typy výběrových chyb: \textbf{náhodné výběrové chyby} a \textbf{systematické chyby}. Získáním náhodného výběru zmenšujeme systematickou chybu a získáváme podklad pro odhad náhodné výběrové chyby.
    \item Výběrová rozdělení statistik jsou teoretická \textbf{pravděpodobnostní rozdělení}, která popisují vztah mezi výběrovou statistikou a populací.
    \item Směrodatná odchylka výběrového rozdělení statistiky (odhad parametru) se nazývá směrodatná chyba. Odhaduje náhodnou výběrovou chybu vypočítané statistiky (odhadu parametru).
    \item Jak roste velikost výběru, výběrová chyba a směrodatná chyba se zmenšují.
    \item Směrodatná chyba se používá k získání intervalového odhadu parametrů i k testování hypotéz o parametrech rozdělení.
\end{enumerate}
\section{Základní metody statistické indukce}
\begin{itemize}
    \item \textbf{Intervalové odhady} (confidence intervals) -- umožnují odhadnout \textbf{nejistotu} v odhadu parametru náhodné veličiny.
    \item \textbf{Testování hypotéz}(hypothesis testing) -- umožnuje posoudit, zda experimentálně získaná data nepopírají předpoklad, který jsem \textbf{před} provedením testování učinili.
\end{itemize}
\begin{figure}[H]
    \centering
    \includegraphics[width=0.6\textwidth]{assets/14_metody_stat_ind}
\end{figure}
\subsection{Intervalové odhady}
\begin{itemize}
    \item V praktických aplikacích často určujeme \textbf{odhad příslušného parametru} pomocí intervalového odhadu.
    \item Tento odhad je reprezentován intervalem $<t_D, t_H>$, v němž hledaný parametr leží s předem určenou pravděpodobností (spolehlivostí), kterou označujeme $(1 − \alpha)$.
    \item neboli parametr populace aproximujeme intervalem, v němž s velkou pravděpodobností příslušný populační parametr leží.
\end{itemize}

\subsection{Interval spolehlivosti (konfidenční interval)}
Interval spolehlivosti (konfidenční interval) pro parametr $\theta$ se spolehlivostí $1−\alpha$, kde $\alpha \in <0; 1>$, je \textbf{taková dvojice statistik} $(T_D, T_H)$, že $P(T_D \leq \theta \leq T_H) = 1 − \alpha$.
\begin{figure}[H]
    \centering
    \includegraphics[width=0.6\textwidth]{assets/14_inter_odhad_terminologie}
\end{figure}
\begin{itemize}
    \item \textbf{Intervalový odhad} $t_D,t_H$ je jednou \textbf{z realizací intervalu spolehlivosti}.
    \item Požadavky na interval spolehlivosti:
          \begin{itemize}
              \item Co \textbf{největší spolehlivost} odhadu.
              \item Co \textbf{nejmenší šírka} intervalu spolehlivost. (s rostoucí spolehlivostí se zvětšuje šířka intervalového odhadu a tím \textbf{klesá významnost} takto získané informace. S rostoucím rozsahem výběru se šířka intervalového odhadu snižuje.)
          \end{itemize}
\end{itemize}
\subsection*{Typy intervalů spolehlivosti}
\begin{itemize}
    \item \textbf{oboustranné}
          \begin{equation*}
              P(\theta < T_D) = P(\theta > T_H) = \frac{\alpha}{2}
          \end{equation*}
          Tyto dvě podmínky zaručují, že $P(T_D \leq \theta \leq T_H) = 1 - \alpha$
    \item \textbf{jednostranné} (odhadujeme--li například délku života nějakého zařízení, je pro nás důležitá pouze dolní mez)
          \begin{itemize}
              \item \textbf{levostranné} $P(\theta \geq T_D^*) = 1 - \alpha$
              \item \textbf{pravostranné} $P(\theta \leq T_H^*) = 1 - \alpha$
          \end{itemize}
\end{itemize}
\begin{figure}[H]
    \centering
    \includegraphics[width=0.6\textwidth]{assets/14_spolehlivost_odhadu}
\end{figure}
\textbf{Co to znamená}, že spolehlivost odhadu je $1- \alpha$? -- Simulace 100 intervalových odhadů  (obrázek výše) střední hodnoty (spolehlivost $0,95$) získaných na základě opakovaných výběrů o rozsahu 30 z populace se střední hodnotou 100. 6 intervalů ze 100 \textbf{neobsahuje skutečnou střední hodnou}.
\section{Jak najít intervalový odhad parametru $\theta$?}
\textbf{Obecně:}
\begin{enumerate}
    \item Zvolíme \textbf{vhodnou výběrovou charakteristiku} $T(\mathbf{X})$, jejíž rozdělení známe.
    \item \begin{equation*}
              \begin{split}
                  P(x_\frac{\alpha}{2} \leq T(\mathbf{X}) \leq x_{1 - \frac{\alpha}{2}}) &= 1 - \alpha, \\
                  P(T(\mathbf{X}) \leq x_{1-\alpha}) &= 1 - \alpha, \\
                  P(T(\mathbf{X}) \geq x_\alpha) &= 1 - \alpha.
              \end{split}
          \end{equation*}
\end{enumerate}

\section{Testování hypotéz}
\textbf{Statistická hypotéza} -- předpoklad (tvrzení) o rozdělení náhodné veličiny.
\begin{itemize}
    \item Zdrojem statistických hypotéz jsou například předchozí zkušenosti, teorie, kterou je třeba doložit, požadavky na kvalitu produktu, dohady založené na náhodném pozorování.
    \item \textbf{Příklady} statistických hypotéz:
          \begin{itemize}
              \item Střední životnost žárovek Ed je nižší než výrobcem udávaných 5 let.
              \item Mortalita je u laparoskopických operací nižší než u operací konvenčních.
              \item Průměrné výsledky srovnávacích testů závisí na typu absolvované střední školy.
              \item Pořízený datový soubor je výběrem z populace mající normální rozdělení.
          \end{itemize}
    \item \textbf{Parametrická statistická hypotéza} -- tvrzení ohledně efektu:
          \begin{itemize}
              \item Hypotézy \textbf{o parametru jedné populace} (o střední hodnotě, rozptylu, mediánu, parametru binomického rozdělení,...).
              \item Hypotézy \textbf{o parametrech dvou populací} (srovnávací testy).
              \item Hypotézy \textbf{o parametrech více než dvou populací} (ANOVA, Kruskalův--Wallisův test,...).
          \end{itemize}
    \item \textbf{Neparametrická statistická hypotéza} -- tvrzení o \textbf{jiné vlastnosti} (rozdělení náhodné veličiny) než o jejím parametru (např. hypotézy o typu rozdělení NV, hypotézy o závislosti NV,...)
\end{itemize}
\textbf{Příklad, ověření, zda statistická hypotéza je pravdivá}: Domníváme se, že střední hodnota obsahu cholesterolu v krvu je u české populace 4,7 mmol/l.
\begin{equation*}
    \begin{split}
        H_0 : \mu = 4.7   \\
        H_A : \mu \not = 4.7
    \end{split}
\end{equation*}
\subsection*{Jak tento předpoklad ověřit?}
\begin{itemize}
    \item Zjistíme údaje o obsahu cholesterolu v krvi u 100 náhodně vybraných Čechů.
    \item Průměrný obsah cholesterolu v krvi probandů (tj. jedinců, kteří jsou předmětem zkoumání) byl 5,4 mmol/l.
\end{itemize}
\subsection*{Jsou tyto výsledky v souladu s naší hypotézou?}
\begin{itemize}
    \item[$\circ$] I kdyby byla testovaná hypotéza pravdivá, nelze očekávat, že průměrná hodnota pozorovaná ve výběru bude přesně 4,7 mmol/l.
    \item[$\circ$] \textbf{Nulovou hypotézu zamítneme, pokud získané uspořádání výberu bude za předpokladu platnosti nulové hypotézy velmi nepravděpodobné}.
\end{itemize}
Rozhodovací proces, v němž proti sobě stojí nulová a alternativní hypotéza:
\begin{itemize}
    \item \textbf{Nulová hypotéza $\mathbf{H_0}$} -- tvrzení, že efekt je nulový, resp. že neexistuje závislost, že data mají určitý typ rozdělení, ...
    \item \textbf{Alternativní hypotéza $\mathbf{H_A}$ ($\mathbf{H_1}$)} -- tvrzení, popírající hypotézu nulovou (obvykle to, co chceme dokázat).
\end{itemize}

\subsection{Klasický přístup při testování hypotéz}
\begin{enumerate}
    \item Formulujeme \textbf{nulovou a alternativní hypotézu}.
    \item Zvolíme tzv. \textbf{testovací statistiku}, tj. výběrovou charakteristiku, jejíž rozdělení závisí na testovaném parametru $\theta$. (Rozdělení testované statistiky za předpokladu platnosti nulové analýzy nazýváme \textbf{nulové rozdělení}.)
    \item Ověříme \textbf{předpoklady testu}! %Hanzi, to musíš zařvat na Edu!
    \item Určíme \textbf{kritický obor} $W^*$, tj. množinu, v níž se, za předpokladu platnosti $H_0$, hodnoty testované statistiky vyskytují s velmi malou pravděpodností.
          \begin{itemize}
              \item Doplňkem k $W^*$ je tzv. \textbf{obor přijetí} $V^*$.
              \item Hranici mezi kritickým oborem a oborem přijetí označujeme jako \textbf{kritická hodnota testu} $t_{krit}$.
          \end{itemize}
    \item Na základě konkrétní realizace výběru určímě \textbf{pozorovanou hodnotu} $X_{OBS}$ testované statistiky.
    \item Na základě vztahu mezi $X_{OBS}$ a $t_{krit}$ rozhodneme o výsledku testu (\uv{Zamítáme $H_0$.} nebo \uv{Nezamítáme $H_0$.})
\end{enumerate}

\subsection{Chyba I. a II. druhu}
\begin{figure}[H]
    \centering
    \includegraphics[width=0.6\textwidth]{assets/14_chyba_tab}
\end{figure}
Jestliže nulová hypotéza je ve skutečnosti platná a my ji přesto zamítneme, dopouštíme se chyby, označované jako chyba \textbf{I. druhu}. Pravděpodobnost, že k takovémuto pochybení dojde, nazýváme \textbf{hladina významnosti} a označujeme ji $\alpha$. Platí-li nulová hypotéza a my jsme ji nezamítli, rozhodli jsme správně. Pravděpodobnost tohoto rozhodnutí označujeme $1 − \alpha$ a nazýváme ji \textbf{spolehlivost testu}. Správným rozhodnutím je rovněž \textbf{zamítnutí nulové hypotézy v případě, že je platná hypotéza alternativní}. Tohoto rozhodnutí se dopouštíme s pravděpodobností $1 − \beta$, což bývá označováno jako síla testu. \textbf{Chybou II. druhu} je nezamítnutí nulové hypotézy v případě, že je platná hypotéza alternativní. Pravděpodobnost této chyby označujeme $\beta$.
\begin{figure}[H]
    \centering
    \includegraphics[width=0.6\textwidth]{assets/14_chyba_graf}
    \caption{Demonstrace pravděpodobností chyb I. a II. druhu}
\end{figure}
Při testování hypotéz se samozřejmě snažíme postupovat tak, abychom minimalizovali obě chyby, tj. dosáhnout vysoké síly testu (nízkého $\beta$) při co nejnižší hladině významnosti $\alpha$. To však není možné, neboť snížením $\beta$ se zvýší hladina významnosti $\alpha$ a naopak. Proto je třeba najít kompromis mezi požadavky na $\alpha$ a $\beta$.


\subsection{Parametrická statistická hypotéza}
\textbf{Jednovýběrové testy}
\begin{itemize}
    \item Test o \textbf{střední hodnotě} (z--test, t--test).
    \item Test o \textbf{rozptylu}.
    \item Test o \textbf{parametru binomického rozdělení}.
    \item Test o \textbf{mediánu} (Wilcoxonův test, Mediánový test).
\end{itemize}
\textbf{Dvouvýběrové testy}
\begin{itemize}
    \item Test o shodě \textbf{dvou středních hodnot} (t--test, Aspinové--Welchův test).
    \item Test o shodě \textbf{rozptylů} (F--test).
    \item Test o shodě \textbf{parametrů dvou binomických rozdělení} (test homogenity dvou binomických rozdělení).
    \item Test o shodě \textbf{mediánů} (Mannův--Whitneyův test).
    \item \textbf{Párové testy} (párový t--test, párový znaménkový test).
\end{itemize}
\textbf{Vícevýběrové testy}
\begin{itemize}
    \item Testy \textbf{shody rozptylů} (Bartletův test, Hartleyův test, Cochranův test, Leveneův test).
    \item \textbf{Analýza rozptylu} (tzv. ANOVA, tj. shody středních hodnot) -- post hoc analýza pro analýzu rozptylu.
    \item Kruskalův--Wallisův test (test \textbf{shody mediánů}) -- post hoc analýza Kruskal--Wallisův test.
\end{itemize}
\textbf{ANOVA}
\begin{itemize}
    \item Test umožnující \textbf{srovnání průměrů více než dvou výběrových souborů}.
    \item Můžeme například zkoumat, zda:
          \begin{itemize}
              \item typ absolvované střední školy ovlivňuje počet bodů dosažených studenty u přijímací zkoušky z matematiky,
              \item použitá medikace ovlivňuje krevní tlak pacientů,
              \item typ použitého hnojiva ovlivňuje výnosy určité plodiny,
              \item pracovní výkon dělníka závisí na umístění stroje, apod.
          \end{itemize}
\end{itemize}

\end{document}
