Termín asociační pravidla široce zpopularizoval počátkem 90. let v souvislosti s analýzou nákupního košíku. Při této analýze se zjišťuje, jaké druhy zboží si současně kupují zákazníci v supermarketech (např. pivo a párek). \textbf{Jde} tedy \textbf{o hledání vzájemných vazeb} (\textbf{asociací}) \textbf{mezi různými položkami} sortimentu prodejny. Přitom není upřednostňován žádný speciální druh zboží jako závěr pravidla.

\subsection{Základní charakteristika pravidel}
U pravidel vytvořených z dat nás obvykle zajímá kolik příkladů splňuje \textbf{předpoklad} a kolik \textbf{závěr} pravidla, kolik příkladů splňuje předpoklad i závěr \textbf{současně}, kolik příkladů splňuje předpoklad a \textbf{nesplňuje} závěr…. Tedy, zajímá nás, jak pro pravidlo:
\begin{equation}
    Ant \Rightarrow Suc, \quad\textrm{ kde } Ant, Suc \subseteq I \textrm{ (položky)}
\end{equation}
kde $ Ant $ (\textbf{předpoklad}, levá strana pravidla, \textbf{antecedent}) a $ Suc $ (\textbf{závěr}, pravá strana pravidla, \textbf{sukcedent}) jsou kombinace kategorií, pro něž příslušná \textbf{kontingenční tabulka} vypadá následovně:
\begin{table}[H]
    \centering
    \begin{tabular}{l|ll}
                     & $ Suc $ & $ \neg Suc $ \\\hhline
        $ Ant $      & a       & b            \\
        $ \neg Ant $ & c       & d
    \end{tabular}
\end{table}
\begin{itemize}
    \item $Ant \land Suc$ -- \textbf{a} je počet objektů {pokrytých současně předpokladem i závěrem},
    \item $Ant \land \neg Suc)$ -- \textbf{b} je počet objektů {pokrytých předpokladem a nepokrytých závěrem},
    \item $\neg Ant \land Suc)$ -- \textbf{c} je počet příkladů {nepokrytých předpokladem ale pokrytých závěrem},
    \item $\neg Ant \land \neg Suc)$ -- \textbf{d} je počet příkladů {nepokrytých ani předpokladem ani závěrem}.
\end{itemize}

\subsection{Základní charakteristiky asociačních pravidel}
\begin{itemize}
    \item \textbf{Support (podpora)} -- relativní četnost objektů splňující předpoklad i závěr, jinými slovy {\scriptsize$\frac{\textrm{počet splňující výstup}}{\textrm{počet položek}}$}:
          \begin{equation}
              sup(Ant \Rightarrow Suc) = \frac{a}{a + b + c + d}, \, \in \, \langle 0; 1 \rangle.
          \end{equation}
    \item \textbf{Confidence (spolehlivost)} -- podmíněná pravděpodobnost závěru pokud platí předpoklad, tedy {\scriptsize$\frac{\textrm{podpora obou}}{\textrm{podpora závěru}}$}:
          \begin{equation}
              conf(Ant \Rightarrow Suc) = \frac{sup(Ant \cup Suc)}{sup(Suc)} = \frac{a}{a + b}.
          \end{equation}
    \item Další: \textbf{pokrytí}, \textbf{zajímavost}, \textbf{závislost}.
\end{itemize}

\section{Hledání často se opakujících množin položek)}
Frequent item set je množina, kde $sup(K) \geq \gamma$, máme tedy stanovenou \textbf{minimální podporu}. Pokud je např. $\gamma = 0,3$ pak je minimální podpora 30\%.

\subsection{Generování kombinací}
Základem všech algoritmů pro hledání asociačních pravidel je \textbf{generování kombinací (konjunkcí) hodnot atributů}. Při generování vlastně procházíme (prohledáváme) prostor všech přípustných konjunkcí. Metod je několik:
\begin{itemize}
    \item do \textbf{hloubky},
    \item do \textbf{šířky},
    \item \textbf{heuristicky},
\end{itemize}

\subsection{Algoritmus apriori}
Jedná se o nejznámějším algoritmus pro hledání asociačních pravidel Jádrem algoritmu je \textbf{hledání často se opakujících množin} položek (frequent itemsets). Jedná se kombinace (konjunkce) kategorií, které dosahují předem zadané četnosti (\textbf{minimální podpory}) v datech.

\begin{figure}[H]
    \centering
    \includegraphics[width=0.6\textwidth]{assets/apriori.png}
\end{figure}

Při hledání kombinací délky $ k $, které mají vysokou četnost se využívá toho, že \textbf{již známe kombinace délky} $ k-1 $. Při vytváření kombinace délky $ k $ spojujeme kombinace délky $ k-1 $.

Jde tedy o \textbf{generování kombinací ,,do šířky''}. Přitom pro vytvoření jedné kombinace délky $ k $ požadujeme, aby všechny její podkombinace délky $ k-1 $ \textbf{splňovaly požadavek na četnosti}. Tedy např. ze tříčlenných kombinací $\{A_1A_2A_3,  \,A_1A_2A_4, \, A_1A_3A_4, \,A_1A_3A_5,  \,A_2A_3A_4\}$ dosahujících požadované četnosti vytvoříme \textbf{pouze jedinou čtyřčlennou} kombinaci $ A_1A_2A_3A_4 $. Kombinaci $ A_1A_3A_4A_5 $ sice lze vytvořit spojením $ A_1A_3A_4 $ a
$ A_1A_3A_5 $, ale mezi tříčlennými kombinacemi chybí $ A_1A_4A_5 $ i $ A_3A_4A_5 $.

\subsection{Algoritmus Next Closure}
Slouží k vytváření formálních kontextů, \textbf{vyhledáváním nejmenších intentů}, postup:
\begin{enumerate}
    \item Začnu s následující tabulkou:
          \begin{table}[H]
              \centering
              \begin{tabular}{l|l|l|l|l|p{6cm}}
                  \multirow{2}{*}{$ A $} & \multirow{2}{*}{$ i $} & \multirow{2}{*}{$ A \cap \{1 \ldots i - 1\} \cup \{i\} = B' $ } & \multirow{2}{*}{$ cl(B') = B $ } & \multirow{2}{*}{$B \setminus A $  } & je-li $B \setminus A = \{i\}$ nebo větší $\rightarrow$ ANO  \\
                                         &                        &                                                                 &                                  &                                     & je-li $B \setminus A = \{j\}$, kde $j < i$ $\rightarrow$ NE \\\hhline
                  $\emptyset$            & 5                      &                                                                 &                                  &                                     &
              \end{tabular}
          \end{table}
    \item Do $A$ vložím prázdnou množinu a $i$ nastavím na nejvyšší intent (pořadí).
    \item Udělám průnik s $A \cap \{1 \ldots i - 1\} \cup \{i\} = B'$ $\rightarrow$ průnik Ačka s intenty od $ 1 $ do $ i-1 $, k tomu přidám $i$. Př.: $A = \{1, 3, 4\}, i = \{3\} \Rightarrow B' = \{1, 3\}$.
    \item Udělám closure ($B'$) $\rightarrow$ intent na extent $\rightarrow$ extent na intent $\Rightarrow B'^{\downarrow \uparrow}$.
    \item Od $B$ odečtu $A$.
    \item Je-li:
          \begin{itemize}
              \item $B \setminus A = \{i\}$ a větší tak ANO [je-li nejmenší prvek z $B \setminus A$ roven nebo větší než $\{i\}$],
              \item $B \setminus A = \{j\}$ kde $j < i$ tak NE [je-li nejmenší prvek z $B \setminus A$ menší než i pak NE].
          \end{itemize}
    \item Pokud:
          \begin{itemize}
              \item ANO $\rightarrow$ do $A$ dosadíme $cl(B') = B$ a $i$ nastavíme na nejvyšší intent,
              \item NE $\rightarrow$ do $A$ neměním a snížíme $i$ o $-1$.
          \end{itemize}
    \item Skončím když je $A$ rovno celé množině extentů.
\end{enumerate}

\subsubsection{DÚLEŽITÉ} V $i$ přeskakuju hodnoty, které jsou v $A$ [výjde pro ně $\emptyset \rightarrow$ neřeším].

\subsection*{Příklad}
\begin{table}[H]
    \centering
    \begin{tabular}{l|lllll}
              & $y_1$ & $y_2$ & $y_3$ & $y_4$ & $y_5$ \\\hhline
        $x_1$ & 0     & 1     & 0     & 1     & 1     \\
        $x_2$ & 0     & 1     & 1     & 0     & 0     \\
        $x_3$ & 1     & 1     & 0     & 1     & 1     \\
        $x_4$ & 1     & 1     & 1     & 0     & 0     \\
        $x_5$ & 1     & 0     & 1     & 1     & 0     \\
        $x_6$ & 1     & 0     & 0     & 1     & 0
    \end{tabular}
\end{table}


\begin{tabular}{l|l|l|l|l|p{3cm}}
    \multirow{2}{*}{$ A $} & \multirow{2}{*}{$ i $} & \multirow{2}{*}{$ A \cap \{1 \ldots i - 1\} \cup \{i\} = B' $ } & \multirow{2}{*}{$ cl(B') = B $ } & \multirow{2}{*}{$B \setminus A $  } & \multirow{2}{*}{ANO / NE }             \\
                           &                        &                                                                 &                                  &                                     &                                        \\ \hhline
    $\emptyset$            & 5                      & 5                                                               & 2, 4, 5                          & 2, 4, 5                             & N [2 < 5]                              \\
    $\emptyset$            & 4                      & 4                                                               & 4                                & 4                                   & A [4 $\geq$ 4] $\leftarrow$ $i_n$      \\
    4                      & 5                      & 4, 5                                                            & 2, 4, 5                          & 2, 5                                & N [2 < 5]                              \\
    4                      & 3 [skip $i$]           & 3                                                               & 3                                & 3                                   & A [3$\geq$ 3] $\leftarrow$ $i_{n - 1}$ \\
    3                      & 5                      & 3, 5                                                            & 1, 2, 3, 4, 5                    & 1, 2, 4, 5                          & N [1 < 5]                              \\
                           &                        & \vdots                                                          &                                  &                                     &                                        \\
    1, 2, 3, 4, 5          &                        & KONEC                                                           &                                  &                                     & A $i_{1}$
\end{tabular}