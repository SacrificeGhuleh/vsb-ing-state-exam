\section{Metrický prostor}
Metrický prostor je \textbf{matematická struktura}, pomocí které lze formálním způsobem definovat pojem \textbf{vzdálenosti}. Na metrických prostorech se poté definují další topologické vlastnosti jako např. \textbf{otevřenost} a \textbf{uzavřenost} množin, jejichž zobecnění pak vede na ještě abstraktnější matematický pojem \textbf{topologického prostoru}.

\subsection{Formální definice}
Metrický prostor je dvojice $(M, \rho)$, kde $M$ je libovolná neprázdná množina a $\rho$ je \textbf{metrika}, což je zobrazení:
\begin{equation*}
    \rho: M \times M \rightarrow \mathbb{R},
\end{equation*}
které splňuje následující axiomy:
\begin{enumerate}
    \item \textbf{Nezápornost}: $\forall x, y \, \rho(x, y) \geq 0$.
    \item \textbf{Totožnost}: $\forall x, y \, \rho (x, y)  = 0 \Leftrightarrow x = y$.
    \item \textbf{Symetrie}: $\forall x, y \in M: \, \rho(x, y) = \rho(y,x)$.
    \item \textbf{Trojúhelníková nerovnost}: $\forall x, y, z \in M: \, \rho(x, y) + \rho(y,z) \geq \rho(x, z)$.
\end{enumerate}

\subsection{Metriky v $\mathbb{R}^n$}
U metrik v $\mathbb{R}^n$ platí:
\begin{itemize}
    \item Každý normovaný vektorový prostor je metrickým prostorem.
    \item Množina reálných čísel spolu s metrikou $\rho(x, y) = |x - y|$, kde $x, y$ jsou libovolné body množiny $\mathbb{R}$ tvoří \textbf{úplný metrický prostor}.
\end{itemize}
Mezi nejpoužívanější \textbf{metriky} na Euklidovském prostoru $\mathbb{R}^n$ patří:
\begin{enumerate}
    \item \textbf{Manhattanská} -- $\rho_1(x, y) = \sum_{i = 0}^n |x_i - y_i|$.
    \item \textbf{Euklidovská} -- $\rho_2(x, y) = \sqrt{\sum_{i = 0}^n |x_i - y_i|^2}$.
    \item \textbf{Minkowského} -- $\rho_3(x, y) = (\sum_{i = 0}^n |x_i - y_i|^P)^{1/P}$, kde $P \geq 1; \, P \in \mathbb{R}$.
    \item \textbf{Čebyševova (Maximova)} -- $\rho_{\max}(x, y) = \max_{\forall i} |x_i - y_i|$, speciální případ Minkowského metriky pro $P = \infty$.
\end{enumerate}

\subsection{Podobnosti a nepodobnosti}
\begin{itemize}
    \item \textbf{Cosinova podobnost} -- míra podobnosti dvou vektorů, která se získá výpočtem kosinu úhlu těchto vektorů:
          \begin{equation*}
              S_c(x, y) = \frac{x \cdot y}{||x|| \, ||y||} = \frac{\sum_{i = 0}^{n} x_i - y_i}{\sqrt{\sum_{i = 0}^{n} x_i^2} \sqrt{\sum_{i = 0}^{n} y_i^2}}
          \end{equation*}
    \item \textbf{Jaccardova podobnost} -- $S_j(X, Y) = \frac{|X \cap Y|}{X \cup Y}$ používá se pro porovnání podobnosti dvou množin.
    \item \textbf{Jaccardova nepodobnost} (svým způsobem vzdálenost) -- $d = 1 - S$, $d = \frac{1}{S} - 1$ pro $S \neq 0$.
\end{itemize}

\subsection{Vzdálenost mezi slovy}
K určení vzdálenosti mezi textovými řetězci definujeme tyto \textbf{vzdálenosti}:
\begin{enumerate}
    \item \textbf{Hammingova} -- počet rozdílných písmen na stejných pozicích: $d_h(\textrm{Karolin, Kathrin}) = 3$, lze použít pouze pro \textbf{stejně dlouhá slova!}.
    \item \textbf{LCS (Longest common subsequence)} -- počet operací \textbf{vkládání} a \textbf{mazání} nutných k převodu jednoho slova na druhé: $d_{\textrm{LCS}  }(\textrm{kitten, sitting}) \Rightarrow \textrm{itten [-K]} \rightarrow \textrm{sitten [+S]} \rightarrow \textrm{sittn [-E]} \rightarrow \textrm{sittin [+I]} \rightarrow \textrm{sitting [+G]} =$ \textbf{5 operací}.
    \item \textbf{Levenshteinova} -- počet operací \textbf{vkládání}, \textbf{mazání}, \textbf{substituce}  nutných k převodu jednoho slova na druhé: $d_l(\textrm{kittne, sitting}) \Rightarrow \textrm{sitten [K }\sim \textrm{ S]} \rightarrow \textrm{sittin [E }\sim \textrm{ I]} \rightarrow \textrm{sitting [+G]} =$ \textbf{3 operace}.
\end{enumerate}
\textbf{Editační vzdálenost} patří zde LSC a Levenstheinova vzdálenost (při určení se používají úpravy).

\subsection{Normalizace}
Snažíme se dostat všechny hodnoty atributů ve sloupci do intervalu $\langle 0, 1\rangle$, proto dělíme celý sloupce jeho maximem (dělíme v rámci daného sloupce, pro každý sloupec zvlášť vlastním maximem).

\subsection*{Příklad}
$d'_1$ a $d'_2$ reprezentují \textbf{normovaný tvar} vzdáleností $d_1$ a $d_2$:
\begin{table}[H]
    \centering
    \begin{tabular}{l|lllllll}
               & $t_1$ & $t_2$ & $t_3$ & $t_4$         & $t_5$         & $t_6$ & $t_7$ \\\hhline
        $d_1$  & 1     & 0     & 0     & 1             & 3             & 0     & 1     \\
        $d_2$  & 1     & 0     & 0     & 2             & 2             & 0     & 0     \\
        $d'_1$ & 1     & 0     & 0     & $\frac{1}{2}$ & 1             & 0     & 1     \\
        $d'_2$ & 1     & 0     & 0     & 1             & $\frac{2}{3}$ & 0     & 0
    \end{tabular}
\end{table}
\begin{table}[H]
    \centering
    \begin{tabular}{l|l|l|l|l|l|l}
                             & $i = 1$        & $i = 2$       & $i = 3$                  & $i = \infty$ & Cosinova                          & Jaccard        \\\hhline
        $\rho_i(d_1, d_2)$   & 3              & $\sqrt{3}$    & $ \sqrt{3}^3 $           & 1            & $\frac{1}{6 \sqrt(3)}$            & $\frac{3}{4}?$ \\
        $\rho_i(d'_1, d'_2)$ & $\frac{11}{6}$ & $\frac{7}{6}$ & $\frac{3 \sqrt{251}}{6}$ & 1            & $\frac{5 \sqrt(154)}{3 \sqrt(2)}$ & $\frac{3}{4}?$
    \end{tabular}
\end{table}

\section{Topologický prostor}
Jedná se o \textbf{rozšíření (zobecnění) metrického prostoru}. Cílem topologie je studium \textbf{vlastností prostorů}. Na rozdíl od teorie metrických prostorů se v topologii \textbf{nezajímáme o vzdálenosti mezi body} prostoru a prostory považujeme za stejné, pokud \textbf{se na sebe dají vzájemně přeměnit} nějakou spojitou deformací. Takže např. nerozlišujeme mezi koulí a krychlí, ostatně koule se změní v krychli již při přechodu mezi dvěma ekvivalentními metrikami v $\mathbb{R}^3$.

Základním pojmem, který se v topologii studuje je \textbf{spojitost zobrazení}. Proto není úplně potřeba vědět přesně, jak jsou od sebe které body daleko. Vystačíme si s informacemi, že jisté body se nekonečně blíží k nějakému bodu prostoru.

\subsection{Formální definice}
Topologickým prostorem nazveme množinu $X$ společně s kolekcí $\tau$ podmnožin $X$, tedy \textbf{dvojici} $(X, \tau)$, splňující následující axiomy:
\begin{enumerate}
    \item $\emptyset, X \in \tau$,
    \item $\forall A, B \in \tau \Rightarrow  A \cup B \in \tau $, tedy \textbf{sjednocení} libovolného počtu (tj. konečného, spočetného i nespočetného) množin z $\tau$ leží v $\tau$,
    \item $\forall A, B \in \tau \Rightarrow A \cap B \in \tau $, tedy \textbf{průnik} konečného počtu množin z $\tau$ leží v $\tau$.
\end{enumerate}

\subsection{Uzavřená, otevřená množina}
Pro topologický prostor $(X, \tau)$ je každá množina $A \in \tau$ otevřená množina, a její doplněk je uzavřená množina.

\subsection{Souvislost nesouvislost}
Pokud sjednocením dvou neprázdných množin z $\tau$ získáme všechny prvky topologie (celé $ X $), tak je topologie \textbf{nesouvislá}. Příklad:

\begin{center}
    \begin{minipage}[t]{0.50\textwidth}
        $X = \{a, b, c\}$\\
        $\tau = \{\emptyset, X, \{a, b\}, \{c\}\}$\\
        $\{a, b\} \cup \{c\}  = \{a, b, c\} \Rightarrow \, \tau$ je \textbf{nesouvislá}
    \end{minipage}
    \begin{minipage}[t]{0.40\textwidth}
        \textbf{Poznámka:} sjednocované množiny musí být \textbf{disjunktní}, tedy nesmí mít společné prvky, např.: $\{a, b\}$ a $\{a, c\}$ již disjunktní nejsou, ale nenaruší souvislost.
    \end{minipage}
\end{center}
\smallskip
Topologický prostor $ \tau $, je \textbf{souvislý} právě tehdy, když jedině podmnožiny v $\tau$, které jsou současně otevřené i uzavřené jsou $X$ a $\emptyset$. V opačném případě je $\tau$ \textbf{nesouvislý}.

\subsection{Uzávěrový systém}
Uzávěrový systém $C$ nad množinou $X$ obsahuje $X$ a $\forall A, B \in C$ platí, že $A \cap B \in C$.
\begin{equation*}
    \begin{aligned}
        R_i \subseteq A \times A \quad R_i^* = \textrm{[tranzitivně-reflexivní uzávěr]} \\
        \textrm{pro } R_i^* \textrm{ platí } \quad R_1^* \neq R_2^*; \quad R_1^* \cap R_2^* \in G;\quad R_1^* \cup R_2^* \notin G
    \end{aligned}
\end{equation*}

\subsection{Uzávěrový operátor $cl(A)$}
Uzávěr (\textit{closure}) $A \cup C \rightarrow cl(A)$ je \textbf{nejmenší uzavřená množina obsahující daný prvek}. Analogie u konceptů $cl(B) = B^{\downarrow\uparrow}$ a $cl(A) = A^{\uparrow\downarrow}$. Vlastnosti uzávěrového operátoru:
\begin{enumerate}
    \item \textbf{Idempotence} -- $cl(cl(A)) = cl(A)$.
    \item \textbf{Extensionalita} -- $A \subseteq cl(A)$.
    \item \textbf{Monotónost} -- $A \subseteq B \Rightarrow cl(A) \subseteq cl(B)$.
\end{enumerate}

Platí-li navíc $cl(\emptyset) = \emptyset$ a $cl(A \cup B) = cl(A) \cup cl(B)$, pak je uzávěr $A = cl(A)$. Jinými slovy se jedná o nejmenší komplement (doplněk) množin TP obsahující daný prvek.
\\
\begin{figure}[H]
    \centering
    \includegraphics[width=0.4\textwidth]{assets/9_tpl_space}
\end{figure}


