\documentclass[openany]{book}
\usepackage[czech]{babel}

\usepackage[fit, breakall]{truncate}
\usepackage{fancyhdr}

% \usepackage[utf8]{inputenc}
\usepackage[useregional]{datetime2}
%\usepackage[T1]{fontenc}
\usepackage[a4paper, top=1.6cm, bottom=1.6cm, left=2cm, right=2cm]{geometry}
\usepackage[thinlines]{easytable}
\usepackage{graphicx}
\usepackage[ampersand]{easylist}
% \usepackage{changepage}
\usepackage{float}
\usepackage{color}
\usepackage{xcolor}
\usepackage{siunitx}
\usepackage{enumitem}
\usepackage{latexsym}
\usepackage[unicode=true]{hyperref}
\usepackage{amssymb}
\usepackage{amsmath}
\usepackage{amsfonts}
\usepackage{minted}
\usepackage{caption}
\usepackage{makecell}
\usepackage{tikz}
\usepackage{multirow}
\usepackage{sectsty}
\usepackage{xevlna}

\raggedbottom

\chaptertitlefont{\Large}

\graphicspath{{./1_it_mpzz/}{./2_swi/}{./3_dais/}{./4_ps/}{./5_pgo/}}

%\renewcommand{\baselinestretch}{1.2} 
%\setitemize{itemsep=0pt}
%\setenumerate{itemsep=0pt}

\newcommand{\hhline}{\Xhline{2\arrayrulewidth}}
\newcommand{\resline}{\vspace{1.5mm}\hrule\vspace{1.5mm}}
\setcounter{tocdepth}{0}

\title{Otázky k SZZ\\ING 2021}
\author{Richard Zvonek}

\fancypagestyle{plain}{
\fancyhf{}
\fancyhead[LE, RO]{\tiny \truncate{0.9 \textwidth} \leftmark}
%\fancyhead[RE, LO]{\tiny \truncate{0.47 \textwidth} \rightmark}
\fancyfoot[LE,RO]{\thepage}
\renewcommand{\headrulewidth}{0pt}%
\renewcommand{\footrulewidth}{0.4pt}}


\pagestyle{fancy}
\fancyhf{}
\fancyhead[LE, RO]{\tiny \truncate{0.47 \textwidth} \leftmark}
\fancyhead[RE, LO]{\tiny \truncate{0.47 \textwidth} \rightmark}
\fancyfoot[LE,RO]{\thepage}

\makeatletter
\renewcommand{\@makechapterhead}[1]{%
  {\noindent\raggedright\normalfont% Alignment and font reset
   \Large\bfseries\thechapter~~#1\par\nobreak}% Formatting
  \vspace{\baselineskip}% ...just a little space
}


\begin{document}
\tableofcontents

\part{Počítačová grafika a analýza obrazu}
\chapter{Systémy barev v počítačové grafice, nelinearita grafického výstupu (gamma korekce), kompozice rastrových obrazů (alfa kanál), HDR.}
% Systémy barev v počítačové grafice, nelinearita grafického výstupu (gamma korekce), kompozice rastrových obrazů (alfa kanál), HDR.
\section{Systémy barev v PG}
\begin{itemize}
    \item Základem barevného prostoru je \textbf{barevný model}, který nám dává abstraktní matematický popis, jak lze barvy vyjádřit pomocí n-tic čísel, nejčastěji trojic.
    \item Mezi nejznámější barevné modely v dnešní době patří \textbf{RGB model}.
    \item Model RGB pracuje se třemi základními barvami: \textbf{červenou, zelenou} a \textbf{modrou}, z nichž se odvíjí i jeho název.
    \item Tyto barvy byly zvoleny na základně toho, jak \textbf{čípky v lidském oku} vnímají jednotlivé záření.
    \item Zároveň je RGB \textbf{aditivní barevný model}, což znamená, že se jednotlivé barevné složky \textbf{míchají} (nové barvy získáváme přidáváním větší intenzity jednotlivých složek) a výsledkem jsou další barevné odstíny, případně vyšší intenzita barvy.
    \item Když k tomuto modelu definujeme, jak mají být tyto n-tice interpretovány, dostáváme \textbf{barevný prostor} -- je předem definovaná množina barev, kterou je schopno určité zařízení snímat, zobrazit nebo reprodukovat.
    \item Barevný prostor je tedy \textbf{definován rozsahem barev}, které dokáže zobrazit.
    \item Tomuto rozsahu se také říká \textbf{gamut}. Ten se zpravidla zobrazuje jako oblast v CIE 1931 chromatickém diagramu
\end{itemize}
\begin{figure}[H]
    \centering
    \includegraphics[width=0.3\textwidth]{assets/1_rgb_gamut}
\end{figure}
\subsection{RGB}
\begin{itemize}
    \item Nejrozšířenější barevný prostor postavený na RGB barevném modelu je \textbf{sRGB} - standardní RGB.
    \item Jeho určení je pro zobrazování \textbf{na monitorech} nebo \textbf{kódování barev} na internetu.
    \item Pro všechny tři barevné složky má definované barvy v \textbf{chromatickém diagramu}, které vymezují jeho gamut.
    \item Každá barva, kterou tento prostor zobrazuje, je dána zastoupením jednotlivých barevných složek, buďto relativně (hodnoty jsou v rozmezí 0 - 1) nebo absolutně (konkrétní \uv{bitové} hodnoty, zpravidla 0 - 255, 24-bitů).
    \item RGB je možné zobrazit jako krychli.
    \item Často se přidává \textbf{Alpha kanál} pro průhlednost - \textbf{RGBA} (32-bitů).
          \begin{figure}[H]
              \centering
              \includegraphics[width=0.6\textwidth]{assets/1_rgb_gamut_krychle}
          \end{figure}
\end{itemize}

\subsection{HSV a HSL}
\begin{itemize}
    \item \textbf{Hue, Saturation, Value/Lightness} - barevný model, který nejvíce odpovídá lidskému vnímání barev.
    \item Barvy popisuje pomocí 3 hodnot, které však samy barvy nereprezentují:
          \begin{itemize}
              \item \textbf{Hue} - \textbf{barevný tón}, převládající. Neboli \textbf{odstín} - barva \textbf{odražená} nebo \textbf{procházející} objektem. Měří se jako poloha na standardním barevném kole (\ang{0} až \ang{360}). Obecně se odstín označuje názvem barvy. \ang{0} - červená, \ang{120} - zelená, \ang{240} - modrá.
              \item \textbf{Saturation} - \textbf{sytost} barvy, příměs jiné barvy. Někdy též chroma, síla nebo čistota barvy, představuje množství šedi v poměru k odstínu, měří se v procentech od 0\% (šedá) do 100\% (plně sytá barva). Na barevném kole vzrůstá sytost od středu k okrajům.
              \item \textbf{Value} - \textbf{hodnota jasu}, množství bílého světla. Relativní světlost nebo tmavost barvy. Jas vyjadřuje \textbf{kolik světla barva odráží}, dalo by se také říct přidávání černé do základní barvy.
          \end{itemize}
    \item Nejčastěji se tato reprezentace (popř. \textbf{HSL}) používají v grafických nástrojích jako komponenty pro výběr barvy, protože je mnohem intuitivnější než RGB.
    \item Vyberu si odstín, jak má být sytý a jasný a hotovo. Není třeba řešit jak smíchat 3 barevné složky, abych dostal to co chci.
    \item Dále se využívá často v případě detekce objektů, kdy hodnota HUE (odstín), je nezávislý na osvětlení scény. Problém však nastává u bílých a černých objektů (kdy HUE může být různé), ty lze na základě value a staturation mapovat do podobných barev (žlutá a černá).
    \item Mimo níže uvedené zobrazení \textbf{válcem}, lze také zobrazit \textbf{kuželem} a
          \begin{figure}[H]
              \centering
              \includegraphics[width=0.9\textwidth]{assets/1_hsv_hsl}
          \end{figure}
\end{itemize}

\subsection{CMY a CMYK}
\begin{itemize}
    \item Substraktivní barevné systémy (barvy se ,,\textbf{odečítají}'' od bílé, přidáváním jednotlivých složek až po černou), \textbf{C}yan, \textbf{M}agenta, \textbf{Y}ellow a \textbf{K}ey (Blac\textbf{K}).
    \item Používá se \textbf{pro tisk}.
    \item Černá se přidala, protože smíchání CMY nedává plně černou barvu, navíc je černý inkoust levnější než barevný.
    \item Nevýhodou je, že \textbf{nedokáže správně zobrazit} sytě červenou, zelenou a modrou.
    \item Při tisku to však není poznat.
    \item Před tiskem se RGB obraz převádí do CMYK.
    \item To provádí buďto ovladač tiskárny nebo RIP (Raster Image Processor - u profi tiskáren).
    \item RGB se používá pro aktivní zdroje světla, CMYK jsou \textbf{pasivní} (světlo pouze \textbf{odrážejí}), proto nedokáží udělat tak jasné odstíny.
\end{itemize}
\begin{figure}[H]
    \centering
    \includegraphics[width=0.6\textwidth]{assets/1_cmyk}
\end{figure}

\subsection{YCbCr}
\begin{itemize}
    \item Barva je reprezentována \textbf{jasovou složkou} Y a modrou a červenou \textbf{chrominanční} komponentou.
    \item Není to absolutní barevný model, jedná se o \textbf{způsob kódování RGB} informací.
    \item Využívá se nejčastěji u videa a barevných obrázků, kde je využito faktu, že \textbf{lidské oko nejvíce vnímá jas}, který je reprezentovaný složkou Y. Barvy už tak důležité nejsou a proto se můžou například více \textbf{komprimovat} bez výraznější ztráty kvality obrazu (JPEG).
    \item Jasová složka je kódována v intervalu $\langle0, 1\rangle$ a chrominanční složky v intervalu $\langle-0.5, 0.5\rangle$
\end{itemize}

\section{Nelinearita grafického výstupu (gamma korekce)}
První CRT monitory zobrazovaly jas nelineárně. To znamená, že dvojnásobné napětí neznamená dvojnásobný jas, křivka jasu byla zhruba exponenciální. Tento způsob zobrazení jasu přetrvává i v dnešních monitorech. Proto je potřeba upravit i zobrazované barvy. Mapování barev je potřeba provést nelineárně. Nelineární vstup v kombinaci s exponenciální křivkou jasu ve výsledku vede k k jasu, který je vnímán jako lineární. 
\begin{figure}[H]
    \centering
    \includegraphics[width=0.3\textwidth]{assets/1_gamma_correction_gamma_curves.png}
\end{figure}
Problém nastává při práci s barvami. Pracovat s barvami je potřeba v lineárním prostoru, aby byly výsledky např. matematických operací korektní. Před zpracováním je tedy nutné převést barvu do lineárního prostou, po zpracování je potřeba před zobrazením opět převézt barvu zpět do nelineárního prostoru.

\begin{equation}
    C_{linear} = C_{sRGB}^{\text{gamma}}
\end{equation}

\begin{equation}
    C_{sRGB} = C_{linear}^{\frac{1}{\text{gamma}}}
\end{equation}

\section{Kompozice rastrových obrazů (alfa kanál)}
Alfa kanál v obrazech se používá v počítačové grafice pro průhlednost obrazů. Obrazy mohou obsahovat plnou nebo částečnou průhlednost. Plně transparentní obraz propouští veškeré barvy podkladu, částečně transparentní obraz mixuje část bravy podkladu a část vlastní barvy. Např. výsledná barva obrazu s hodnotou alfa 0.5 vytvoří mix barev tvořený z $50\%$ barvou podkladu a z $50\%$ barvou transparentního objektu. V případě překryvu více transparentních objektů je potřeba objekty setřídit dle vzdálenosti a vypočítat transparentnost postupně.   

\section{HDR}
Barvy jsou tradičně reprezentovány trojicí 8bitových hodnot v intervalu  $\langle-0, 255\rangle$. Takto popsaný obraz ale může ztrácet detaily z důvodu limitovaných možností hodnot pro rozložení barev. Přesnější metodou je ukládání jednotlivých složek jako 16 nebo 32bitových hodnot, pomocí desetiných čísel. Zde ale nastává problém s hodnotami většími než 1. Je teda nutné provést tzv. tone mapping. Proces tone mapping v zásadě převádí hodnoty do intervalu $\langle0, 1\rangle$. \par
Jednou z nejjednodušších metod je Reinhardova metoda, hodnota se vypočítá následujícím vzorcem: 
\begin{equation}
    C_{out} = \frac{C_{in}}{C_{in} + 1}    
\end{equation}
Tato metoda zachová poměrně dobře kontrast pro oblasti obrazu s nízkým jasem, oblasti obrazu s vysokým jasem jsou ale méně kontrastní. 
Pokročilejší metodou je mapování hodnot pomocí expozice, následujícím vzorcem:
\begin{equation}
    C_{out} = 1 - e^{C_{in} * \exp}
\end{equation}
Pomocí hodnoty expozice je pak možné nastavit celkové podání barev obrazu. Hodnota expozice by měla být zvolena v závislosti na aktuálním vstupu, je možné hodnotu expozice volit automaticky. 
% \pagebreak

\chapter{Afinní a projektivní prostor. Afinní a projektivní transformace a jejich matematický zápis. Modelovací a zobrazovací transformace v počítačové grafice.}
\input{5_pgo/includes/2.tex}
% \pagebreak

\chapter{Metody získávání fotorealistických obrazů, rekurzivní sledování paprsku, radiometrie, zobrazovací rovnice, Monte Carlo přístupy ve výpočtu osvětlení, urychlovací metody.}
\input{5_pgo/includes/3.tex}
% \pagebreak

\chapter{Standardní zobrazovací řetězec a realizace jeho jednotlivých kroků, modely osvětlení a stínovací algoritmy, řešení viditelnosti, možnosti výpočtu globálního osvětlení v reálném čase, stručná charakteristika standardu OpenGL.}
\input{5_pgo/includes/4.tex}
% \pagebreak

\chapter{Komprese obrazu a videa, principy úprav obrazu v prostorové a frekvenční doméně.}
\input{5_pgo/includes/5.tex}
% \pagebreak

\chapter{Základní metody úpravy a segmentace obrazu (filtrace, prahování, hrany, oblasti, rohy).}
\input{5_pgo/includes/6.tex}
% \pagebreak

\chapter{Základní metody rozpoznávání objektů, příznakové rozpoznávání. Univerzální příznaky pro rozpoznávání (např. HOG), trénovací klasifikátory (např. SVM).}
\input{5_pgo/includes/7.tex}
% \pagebreak

\chapter{Hluboké neuronové sítě (např. konvoluční, popis jednotlivých vrstev).}
\section{Deep learning}
\begin{itemize}
    \item Deep learning neboli \textbf{hluboké učení}, známé také jako hierarchické učení, je \textbf{sbírka algoritmů} používaných ve strojovém učení.
    \item Používají se k~modelování abstrakcí na vysoké úrovni v~datech za pomocí modelových architektur, které se skládají z~několika nelineárních transformací.
    \item Hluboké učení je součástí široké skupiny metod používané pro strojové učení, které jsou založeny na učení reprezentace dat.
\end{itemize}
Hluboké strukturované učení může být:
\begin{itemize}
    \item{\textbf{Kontrolované (s~učitelem)} - všechna data jsou kategorizovaná do tříd, algoritmy se učí předpovídat výstup ze vstupních dat.}
    \item{\textbf{Částečně kontrolované} - data jsou částečně kategorizovaná do tříd. Pří tomto přístupu učení lze využít kombinaci kontrolovaného a~nekontrolovaného přístupu učení.}
    \item{\textbf{Nekontrolované (bez učitele)} - data nejsou kategorizovaná do tříd, algoritmy se učí ze struktury vstupních dat.}
\end{itemize}
Hluboké učení je specifický přístup, použitý k~budování a~učení neuronových sítí, které jsou považovány za velmi spolehlivé rozhodovací uzly. Jestliže vstupní data algoritmu procházejí řadou nelinearit a~nelineárních transformací, tak tento algoritmus je považován za \uv{deep} algoritmus.

Odstraňuje také ruční identifikaci příznaků (obrázek \ref{fig:ml_vs_ann}) z~dat a~místo toho se spoléhá na jakýkoliv trénovací proces, které má za úkol zjistit užitečné vzory ve vstupních příkladech. To dělá neuronovou síť jednodušší a~rychlejší, a~může přinést lepší výsledky než z~oblasti umělé inteligence.

\begin{figure}[H]
    \centering
    \includegraphics[width=.5\linewidth]{assets/9_ml_vs_ann}
    \caption{Hlavním rozdílem mezi strojovým a~hlubokým učením je ten, že u~strojového se příznaky musí extrahovat manuálně.}
    \label{fig:ml_vs_ann}
\end{figure}

\subsection{Konvoluční neuronové sítě \textit{CNN} - Convolution neural network}
\begin{itemize}
    \item Speciálním druhem vícevrstvých neuronových sítí a~jsou navrženy tak, aby rozpoznaly vizuální vzory přímo z~pixelu obrazu s~minimálním předzpracováním.
    \item Mohou rozpoznat vzory s~extrémní variabilitou (například ručně psané znaky) a~odolnost vůči deformacím a~jednoduchým geometrickým transformacím.
    \item Síť využívá matematickou operaci zvanou konvoluce alespoň v~jedné jejich vrstvě.
\end{itemize}

Nejznámější a~nejvíce používanou konvoluční neuronovou sítí jsou modely LeNet.
Hlavní kroky LeNet sítě jsou:
\begin{itemize}
    \item{\textbf{Konvoluce} - tyto vrstvy provádějí konvoluci nad vstupy do neuronové sítě.}
    \item{\textbf{Nelinearita (ReLU)} - tato vrstva je použita po každé konvoluční vrstvě a~jejím cílem je nahrazení všech negativních pixelů nulou ve výstupu této vrstvy (příznaková mapa).}
    \item{\textbf{Pooling/sub sampling} - ze vstupního obrazu vyextrahuje pouze zajímavé části pomocí některých matematických operací (max, avg, sum), a~tím se \textbf{redukuje jeho dimenzionalita}.}
    \item{\textbf{Fully connected layer/klasifikace} - tato vrstva vychází z~původních umělých neuronových sítí, konkrétně z~vícevrstvého perceptronu. Tato vrstva je typicky umístěna na konci sítě a~je propojena s~klasifikační vrstvou pro predikci.}
\end{itemize}
\begin{figure}[H]
    \centering
    \includegraphics[width=.9\linewidth]{assets/9_cnn.pdf}
    \caption{Řetězec LeNet konvoluční neuronové sítě}
    \label{fig:cnn}
\end{figure}
% \pagebreak

\chapter{Rekonstrukce 3D objektů z 2D obrazů (základní principy).}
\section{Structure From Motion}
Metoda Structure From Motion se zabývá převodem sady snímků objektu do jeho reprezentace ve 3D. Pro získání hloubkové mapy z obrazu jsou potřeba minimálně dva snímky, pro kompletní rekonstrukci 3D modelu je poté potřeba pracovat s co největší datovou sadou. Je potřeba mít snímky vytvořené z různých úhlů, různých vzdáleností. \par
V podstatě se celý proces dá shrnout do následujících kroků: Po vytvoření snímků je potřeba danou datovou sadu zpracovat. Nejprve jsou v obraze nalezeny klíčové body - tzn. nějaké body zájmu. Např. hrany, rohy... Následně je potřeba najít koresponcence mezi body zájmu v jednotlivých obrazech. Po nalezení korespondencí je možné zpětně vypočítat pozici kamer vůči jednotlivými body. Výsledkem je poté mračno bodů, ze kterého lze pak aproximovat povrch tělesa.

Lidé vnímají mnoho informací o trojrozměrné struktuře ve svém prostředí tím, že se kolem ní pohybují. Když se pozorovatel pohybuje, objekty kolem nich se pohybují různě, v závislosti na jejich vzdálenosti od pozorovatele. Toto je známé jako pohybová paralaxa a z této hloubky lze informace použít ke generování přesné 3D reprezentace světa kolem nich.

Hledání struktury z pohybu představuje podobný problém jako hledání struktury ze stereofonního vidění. V obou případech je třeba najít shodu mezi obrázky a rekonstrukcí 3D objektu.

Chceme-li najít korespondenci mezi obrázky, sledují se příznaky - body zájmu jako rohové body (hrany s přechody ve více směrech) z jednoho obrázku na druhý. Jedním z nejpoužívanějších detektorů příznaků je transformace příznaků invariantních vůči měřítku (SIFT - scale-invariant feature transform). Jako rysy používá maxima z pyramidy DOG (Difference-of-Gaussians). Prvním krokem v SIFT je nalezení dominantního směru gradientu. Aby byla rotace invariantní, deskriptor se otočí tak, aby odpovídal této orientaci. Dalším běžným detektorem vlastností je SURF (speeded-up robust features). V SURF je DOG nahrazen detektorem tzv. blobů na bázi hesenské matice. Místo vyhodnocení gradientních histogramů SURF také počítá pro součty složek gradientu a součty jejich absolutních hodnot. Jeho použití integrálních obrazů umožňuje extrémně rychle detekovat příznaky s vysokou mírou detekce. Ve srovnání s SIFT je tedy SURF rychlejším detektorem vlastností s nevýhodou menší přesnosti v pozicích příznaků. Dalším typem příznaku, jsou obecné křivky (např. lokální hrany s přechody v jednom směru, součást technologie známé jako pointless SfM) užitečné, když jsou bodové příznaky nedostatečné, běžné v umělých prostředích.

Příznaky detekované ze všech obrázků budou poté porovnány. Jedním z algoritmů shody, který sleduje příznaky z jednoho obrázku na druhý, je Lukas – Kanade tracker.

Někdy jsou některé z uzavřených příznaků nesprávně spárovány. Z tohoto důvodu by měly být zápasy filtrovány. RANSAC (konsensus náhodných vzorků) je algoritmus, který se obvykle používá k odstranění odlehlých korespondencí. V příspěvku Fischlera a Bollese se RANSAC používá k řešení problému určování polohy (LDP), kde cílem je určit body v prostoru, které se promítají na obraz do souboru orientačních bodů se známými místy. 

Trajektorie příznaků v čase se poté použijí k rekonstrukci jejich 3D pozic a pohybu kamery. Alternativu dávají takzvané přímé přístupy, kdy jsou geometrické informace (3D struktura a pohyb kamery) přímo odhadovány z obrazů, bez mezilehlé abstrakce na příznaky nebo rohy.

Existuje několik přístupů ke struktuře z pohybu. V přírůstkovém SFM jsou pozice kamer řešeny a přidávány jeden po druhém do datov0 sady. V globálním SFM jsou pozice všech kamer řešeny současně. Kompromisem je out-of-core SFM, kde se počítá několik částečných rekonstrukcí, které jsou poté integrovány do globálního řešení. 

\begin{figure}[H]
 \centering
 \includegraphics[width=0.4\textwidth]{assets/10_SFM.png}
\end{figure}

\section{Structured Light}
Na sledovaný objekt jsou vysílány různě velké liniové paprsky, které se zpětně snímají kamerou. Podle zakřívení povrchu je vyslaný paprsek deformován. Ze snímků deformovaných paprsků je pak možné provést triangulaci a následně zpětně rekonstruovat povrch tělesa. 

3D skener strukturovaného světla je zařízení pro měření trojrozměrného tvaru objektu užitímpromítaného svetelného vzoru a kamerového systému.V dnešní době je strukturované světlo běžně využíváno pro různé trojrozměrné profilometrické zkoumání povrchů díky jeho nízké cenně a vysoké rychlosti, proto je jedním z nejčastějších způsobů, jak velmi rychle a efektivně zjistit žádané informace o povrchu objektu. Měřící čidla na bázi strukturovaného světla (SL – structured light) jsou využívány v mnoha odvětvích, kromě strojírenství a kontroly kvality je to např. k uchování uměleckých děl, v zábavním průmyslu, medicíně nebo zabezpečení. Je to především pro jeho bezkonkurenční rozlišení, bezkontaktní zajištění rekonstrukce celého pole objektů ve vysokém rozlišení a rychlost. K výhodám se dá také jistě zařadit kompaktnost skenerů, jelikož měřící proces je realizován pouze profilometrickým systémem, který se skládá z jednotky pro zpracování a analýzu (PC), projekční jednotky (zpravidla videoprojektoru) a vizuální jednotky(CCD/CMOS kamera).Při zaznamenávání stojícího objektu je nutné, aby byl objekt stabilní. Využitím zařízení se strukturovaným světlem lze zaznamenat i geometrii pohybujícíchse těles, to se však děje na úkor snížení kvality digitalizace. Při tomto dynamickém skenování objektů se ukázalo, že základní vzory jsou nedostačující, proto začaly vznikat strukturálně složitější vzory.
\subsection{Princip}
Promítání uzkých pásů světla na trojrozměrně tvarovaný povrch vytváří linie osvětlení, které jsou zkreslené z jiného úhlu než se nachází projektor a mohou být použity pro přesnou geometrickou rekonstrukci tvaru povrchu. Používá se mnoho různých variant strukturovaného světla, ale vodorovné pruhy jsou nejčastější.


\begin{figure}[H]
    \centering
    \includegraphics[width=0.4\textwidth]{assets/10_3-proj2cam.svg.png}
    \includegraphics[width=0.4\textwidth]{assets/10_triangulaceSL.png}
\end{figure}


Modulovaný vzor je porovnán se vzorem promítaným, tj. porovnávají se odpovídající si pixely projektoru a snímacího zařízení. Výsledkem tohoto porovnání a využitím vhodného algoritmu se vytvoří body orientované v prostoru, tzv. mračna bodů.S těmito body lze dále pracovat a vytvořit souvislý povrch.
% \pagebreak

\part{Matematické základy informatiky}

\chapter{Konečné automaty, regulární výrazy, uzávěrové vlastnosti třídy regulárních jazyků.}
% Systémy barev v počítačové grafice, nelinearita grafického výstupu (gamma korekce), kompozice rastrových obrazů (alfa kanál), HDR.
\section{Systémy barev v PG}
\begin{itemize}
    \item Základem barevného prostoru je \textbf{barevný model}, který nám dává abstraktní matematický popis, jak lze barvy vyjádřit pomocí n-tic čísel, nejčastěji trojic.
    \item Mezi nejznámější barevné modely v dnešní době patří \textbf{RGB model}.
    \item Model RGB pracuje se třemi základními barvami: \textbf{červenou, zelenou} a \textbf{modrou}, z nichž se odvíjí i jeho název.
    \item Tyto barvy byly zvoleny na základně toho, jak \textbf{čípky v lidském oku} vnímají jednotlivé záření.
    \item Zároveň je RGB \textbf{aditivní barevný model}, což znamená, že se jednotlivé barevné složky \textbf{míchají} (nové barvy získáváme přidáváním větší intenzity jednotlivých složek) a výsledkem jsou další barevné odstíny, případně vyšší intenzita barvy.
    \item Když k tomuto modelu definujeme, jak mají být tyto n-tice interpretovány, dostáváme \textbf{barevný prostor} -- je předem definovaná množina barev, kterou je schopno určité zařízení snímat, zobrazit nebo reprodukovat.
    \item Barevný prostor je tedy \textbf{definován rozsahem barev}, které dokáže zobrazit.
    \item Tomuto rozsahu se také říká \textbf{gamut}. Ten se zpravidla zobrazuje jako oblast v CIE 1931 chromatickém diagramu
\end{itemize}
\begin{figure}[H]
    \centering
    \includegraphics[width=0.3\textwidth]{assets/1_rgb_gamut}
\end{figure}
\subsection{RGB}
\begin{itemize}
    \item Nejrozšířenější barevný prostor postavený na RGB barevném modelu je \textbf{sRGB} - standardní RGB.
    \item Jeho určení je pro zobrazování \textbf{na monitorech} nebo \textbf{kódování barev} na internetu.
    \item Pro všechny tři barevné složky má definované barvy v \textbf{chromatickém diagramu}, které vymezují jeho gamut.
    \item Každá barva, kterou tento prostor zobrazuje, je dána zastoupením jednotlivých barevných složek, buďto relativně (hodnoty jsou v rozmezí 0 - 1) nebo absolutně (konkrétní \uv{bitové} hodnoty, zpravidla 0 - 255, 24-bitů).
    \item RGB je možné zobrazit jako krychli.
    \item Často se přidává \textbf{Alpha kanál} pro průhlednost - \textbf{RGBA} (32-bitů).
          \begin{figure}[H]
              \centering
              \includegraphics[width=0.6\textwidth]{assets/1_rgb_gamut_krychle}
          \end{figure}
\end{itemize}

\subsection{HSV a HSL}
\begin{itemize}
    \item \textbf{Hue, Saturation, Value/Lightness} - barevný model, který nejvíce odpovídá lidskému vnímání barev.
    \item Barvy popisuje pomocí 3 hodnot, které však samy barvy nereprezentují:
          \begin{itemize}
              \item \textbf{Hue} - \textbf{barevný tón}, převládající. Neboli \textbf{odstín} - barva \textbf{odražená} nebo \textbf{procházející} objektem. Měří se jako poloha na standardním barevném kole (\ang{0} až \ang{360}). Obecně se odstín označuje názvem barvy. \ang{0} - červená, \ang{120} - zelená, \ang{240} - modrá.
              \item \textbf{Saturation} - \textbf{sytost} barvy, příměs jiné barvy. Někdy též chroma, síla nebo čistota barvy, představuje množství šedi v poměru k odstínu, měří se v procentech od 0\% (šedá) do 100\% (plně sytá barva). Na barevném kole vzrůstá sytost od středu k okrajům.
              \item \textbf{Value} - \textbf{hodnota jasu}, množství bílého světla. Relativní světlost nebo tmavost barvy. Jas vyjadřuje \textbf{kolik světla barva odráží}, dalo by se také říct přidávání černé do základní barvy.
          \end{itemize}
    \item Nejčastěji se tato reprezentace (popř. \textbf{HSL}) používají v grafických nástrojích jako komponenty pro výběr barvy, protože je mnohem intuitivnější než RGB.
    \item Vyberu si odstín, jak má být sytý a jasný a hotovo. Není třeba řešit jak smíchat 3 barevné složky, abych dostal to co chci.
    \item Dále se využívá často v případě detekce objektů, kdy hodnota HUE (odstín), je nezávislý na osvětlení scény. Problém však nastává u bílých a černých objektů (kdy HUE může být různé), ty lze na základě value a staturation mapovat do podobných barev (žlutá a černá).
    \item Mimo níže uvedené zobrazení \textbf{válcem}, lze také zobrazit \textbf{kuželem} a
          \begin{figure}[H]
              \centering
              \includegraphics[width=0.9\textwidth]{assets/1_hsv_hsl}
          \end{figure}
\end{itemize}

\subsection{CMY a CMYK}
\begin{itemize}
    \item Substraktivní barevné systémy (barvy se ,,\textbf{odečítají}'' od bílé, přidáváním jednotlivých složek až po černou), \textbf{C}yan, \textbf{M}agenta, \textbf{Y}ellow a \textbf{K}ey (Blac\textbf{K}).
    \item Používá se \textbf{pro tisk}.
    \item Černá se přidala, protože smíchání CMY nedává plně černou barvu, navíc je černý inkoust levnější než barevný.
    \item Nevýhodou je, že \textbf{nedokáže správně zobrazit} sytě červenou, zelenou a modrou.
    \item Při tisku to však není poznat.
    \item Před tiskem se RGB obraz převádí do CMYK.
    \item To provádí buďto ovladač tiskárny nebo RIP (Raster Image Processor - u profi tiskáren).
    \item RGB se používá pro aktivní zdroje světla, CMYK jsou \textbf{pasivní} (světlo pouze \textbf{odrážejí}), proto nedokáží udělat tak jasné odstíny.
\end{itemize}
\begin{figure}[H]
    \centering
    \includegraphics[width=0.6\textwidth]{assets/1_cmyk}
\end{figure}

\subsection{YCbCr}
\begin{itemize}
    \item Barva je reprezentována \textbf{jasovou složkou} Y a modrou a červenou \textbf{chrominanční} komponentou.
    \item Není to absolutní barevný model, jedná se o \textbf{způsob kódování RGB} informací.
    \item Využívá se nejčastěji u videa a barevných obrázků, kde je využito faktu, že \textbf{lidské oko nejvíce vnímá jas}, který je reprezentovaný složkou Y. Barvy už tak důležité nejsou a proto se můžou například více \textbf{komprimovat} bez výraznější ztráty kvality obrazu (JPEG).
    \item Jasová složka je kódována v intervalu $\langle0, 1\rangle$ a chrominanční složky v intervalu $\langle-0.5, 0.5\rangle$
\end{itemize}

\section{Nelinearita grafického výstupu (gamma korekce)}
První CRT monitory zobrazovaly jas nelineárně. To znamená, že dvojnásobné napětí neznamená dvojnásobný jas, křivka jasu byla zhruba exponenciální. Tento způsob zobrazení jasu přetrvává i v dnešních monitorech. Proto je potřeba upravit i zobrazované barvy. Mapování barev je potřeba provést nelineárně. Nelineární vstup v kombinaci s exponenciální křivkou jasu ve výsledku vede k k jasu, který je vnímán jako lineární. 
\begin{figure}[H]
    \centering
    \includegraphics[width=0.3\textwidth]{assets/1_gamma_correction_gamma_curves.png}
\end{figure}
Problém nastává při práci s barvami. Pracovat s barvami je potřeba v lineárním prostoru, aby byly výsledky např. matematických operací korektní. Před zpracováním je tedy nutné převést barvu do lineárního prostou, po zpracování je potřeba před zobrazením opět převézt barvu zpět do nelineárního prostoru.

\begin{equation}
    C_{linear} = C_{sRGB}^{\text{gamma}}
\end{equation}

\begin{equation}
    C_{sRGB} = C_{linear}^{\frac{1}{\text{gamma}}}
\end{equation}

\section{Kompozice rastrových obrazů (alfa kanál)}
Alfa kanál v obrazech se používá v počítačové grafice pro průhlednost obrazů. Obrazy mohou obsahovat plnou nebo částečnou průhlednost. Plně transparentní obraz propouští veškeré barvy podkladu, částečně transparentní obraz mixuje část bravy podkladu a část vlastní barvy. Např. výsledná barva obrazu s hodnotou alfa 0.5 vytvoří mix barev tvořený z $50\%$ barvou podkladu a z $50\%$ barvou transparentního objektu. V případě překryvu více transparentních objektů je potřeba objekty setřídit dle vzdálenosti a vypočítat transparentnost postupně.   

\section{HDR}
Barvy jsou tradičně reprezentovány trojicí 8bitových hodnot v intervalu  $\langle-0, 255\rangle$. Takto popsaný obraz ale může ztrácet detaily z důvodu limitovaných možností hodnot pro rozložení barev. Přesnější metodou je ukládání jednotlivých složek jako 16 nebo 32bitových hodnot, pomocí desetiných čísel. Zde ale nastává problém s hodnotami většími než 1. Je teda nutné provést tzv. tone mapping. Proces tone mapping v zásadě převádí hodnoty do intervalu $\langle0, 1\rangle$. \par
Jednou z nejjednodušších metod je Reinhardova metoda, hodnota se vypočítá následujícím vzorcem: 
\begin{equation}
    C_{out} = \frac{C_{in}}{C_{in} + 1}    
\end{equation}
Tato metoda zachová poměrně dobře kontrast pro oblasti obrazu s nízkým jasem, oblasti obrazu s vysokým jasem jsou ale méně kontrastní. 
Pokročilejší metodou je mapování hodnot pomocí expozice, následujícím vzorcem:
\begin{equation}
    C_{out} = 1 - e^{C_{in} * \exp}
\end{equation}
Pomocí hodnoty expozice je pak možné nastavit celkové podání barev obrazu. Hodnota expozice by měla být zvolena v závislosti na aktuálním vstupu, je možné hodnotu expozice volit automaticky. 
% \pagebreak

\chapter{Bezkontextové gramatiky a jazyky. Zásobníkové automaty, jejich vztah k bezkontextovým gramatikám.}
\input{1_it_mpzz/includes/2.tex}
% \pagebreak

\chapter{Matematické modely algoritmů -Turingovy stroje a stroje RAM. Složitost algoritmu, asymptotické odhady. Algoritmicky nerozhodnutelné problémy.}
\input{1_it_mpzz/includes/3.tex}
% \pagebreak

\chapter{Třídy složitosti problémů. Třída PTIME a NPTIME, NP-úplné problémy.}
\input{1_it_mpzz/includes/4.tex}
% \pagebreak

\chapter{Jazyk predikátové logiky prvního řádu. Práce s kvantifikátory a ekvivalentní transformace formulí.}
\input{1_it_mpzz/includes/5.tex}
% \pagebreak

\chapter{Pojem relace, operace s relacemi, vlastnosti relací. Typy binárních relací. Relace ekvivalence a relace uspořádání.}
\input{1_it_mpzz/includes/6.tex}
% \pagebreak

\chapter{Pojem operace a obecný pojem algebra. Algebry s jednou a dvěma binárními operacemi.}
\input{1_it_mpzz/includes/7.tex}
% \pagebreak

\chapter{FCA – formální kontext, formální koncept, konceptuální svazy.}
\section{Formální konceptuální analýza (FCA)}
Metoda analýzy tabulkových dat (objektů a jejich vlastností), umožňuje jiný pohled na data (využívá se např. u data miningu). \textbf{Vstupem} pro FCA jsou \textbf{tabulková data}, která jsou uspořádána následovně: \textbf{objekty} (řádky) a \textbf{atributy} (sloupce). Tyto tabulková data vytváří tzv. \textbf{kontexty}.

\begin{table}[H]
    \centering
    \begin{tabular}{l|l|l}
                        & \textbf{červené} & \textbf{bílé} \\
        \hhline
        \textbf{jablko} & $\times$         &               \\
        \textbf{zelí}   & $\times$         & $\times$
    \end{tabular}
\end{table}

\section{Formální Kontext}
Formální kontext $K$ obsahuje objekty z množiny $O$ a atributy z množiny $A$. Vztahy mezi objekty a atributy jsou charakterizovány binární relací $ R $. Obecně se pro popis kontextu používá výraz:
\begin{equation}
    K = (O, A, I).
\end{equation}
Takto vymezený formální kontext je dobře zobrazitelný tabulkou, ve které jsou \textbf{řádky} obsazeny \textbf{objekty}, \textbf{sloupce} \textbf{atributy} a incidenční data ($I \subseteq O \times A$) vyjadřují relaci $ R $ ($I$ je relace incidence).

\subsection{Galoisovy konexe}
Umožňují přecházet z množiny objektů na jejich společné atributy ($\uparrow$ \textbf{intent}) a naopak ($\downarrow$ \textbf{extent}).

\begin{itemize}
    \item \textbf{Intent} $\uparrow$ -- $2^X \rightarrow 2^Y; A \subseteq X; A^\uparrow = \{y \in Y; \forall x \in A (x, y) \in I\}$ [\textit{z objektu na atributy}].
    \item \textbf{Extent} $\downarrow$ -- $2^Y \rightarrow 2^Y; B \subseteq Y; B^\downarrow = \{x \in X; \forall y \in B (x, y) \in I\}$ [\textit{z atributů na objekt}].
\end{itemize}

\subsection*{Příklad}
$A_1 = \{\textrm{jablka, zelí}\}, A_1^\uparrow = \{\textrm{červené}\}$ \quad $A_2 = \{\textrm{zelí}\}, A_2^\uparrow = \{\textrm{červené, bílé}\}$\\
$B_1 = \{\textrm{\rm červené, bílé}\}, B_1^\downarrow = \{\textrm{zelí}\}$ \quad $B_2 = \{\textrm{červené}\}, B_2^\downarrow = \{\textrm{zelí, jablko}\}$

\section{Formální koncept}
Formální koncept je dvojice $(A, B)$, kde $A$ je množina objektů, a $B$ je množina atributů, které jsou \textbf{společné pro všechny objekty} z množiny $A$. Koncept $(A, B)$ je tedy dán jako: $(A, B) \Leftrightarrow A = B^\downarrow \land B = A^\uparrow$, kde $A \subseteq X$, $B \subseteq Y$, kde $X$ a $Y$ jsou objekty a atributy výše uvedené tabulky.

\subsection{Uzávěrový operátor $\uparrow \downarrow$}
Jak již znační $A^{\uparrow \downarrow} = C(A)$ vypovídá, k určení uzávěru atributů množiny $A$ se nejprve provede \textbf{intent} a poté \textbf{extent}. Uzávěr má tyto vlastnosti:
\begin{enumerate}
    \item \textbf{Idempotence} -- $C(C(A)) = C(A)$.
    \item \textbf{Extensionalita} -- $A \subseteq C(A)$.
    \item \textbf{Monotonie} -- $A_1 \subseteq A_2 \Rightarrow C(A_1) \subseteq C(A_2)$.
\end{enumerate}

\section{Konceptuální svazy}
\textbf{Uspořádaná množina konceptů} tvoří tzv. konceptuální svaz. Ten lze graficky znázornit \textbf{Hasseovým diagramem} -- každý vrchol grafu reprezentuje \textbf{jeden koncept}. Koncepty $ K_i $ a $ K_j $ jsou v grafu spojeny, pokud $ K_i \leq K_j$, přičemž $K_j$ je umístěn výše než $K_i$ a neexistuje takové $K_k$, že $ K_i \leq K_k \leq K_j$.

\subsection{Návod k vytvoření konceptuálního svazu}
\begin{enumerate}
    \item Vytvořím a postupně si zapíšu množinu všech extentů na jednotlivých atributech (v grafu zapisuji \textbf{zdola nahoru}).
    \item Vytvořím jedinečné průniky extentů.
    \item Nesmím zapomenout na zahrnutí průniku s prázdnou množinou = $\emptyset$.
    \item Přidám extent zahrnující všechny objekty.
    \item Pro odpovídající extenty vytvořím intenty (v grafu zapisuji \textbf{shora dolů}).
\end{enumerate}

\subsection*{Příklad}
\begin{table}[H]
    \centering
    \begin{tabular}{l|l|l|l|l}
                         & \textbf{2 nohy} & \textbf{4 nohy} & \textbf{mléko} & \textbf{vlna} \\\hhline
        \textbf{pes}     &                 & $\times$        &                &               \\
        \textbf{ovce}    &                 & $\times$        & $\times$       & $\times$      \\
        \textbf{koza}    &                 & $\times$        & $\times$       &               \\
        \textbf{slepice} & $\times$        &                 &                &               \\
    \end{tabular}
\end{table}

\begin{minipage}[t]{0.5\textwidth}
    $e_7 = \{$pes, ovce, koza, slepice$\}$\\
    $e_2 = \{\textrm{4 nohy}\}^\downarrow = \{\textrm{pes, ovce, koza}\}$\\
    $e_3 = \{\textrm{mléko}\}^\downarrow = \{\textrm{ovce, koza}\}$\\
    $e_1 = \{\textrm{2 nohy}\}^\downarrow = \{\textrm{slepice}\}$\\
    $e_4 = \{\textrm{vlna}\}^\downarrow = \{\textrm{ovce}\}$\\
    $e_5 = e_2 \cap e_3 = \{\textrm{ovce, koza}\}$\\
    $e_6 = \emptyset$
\end{minipage}
\begin{minipage}[t]{0.5\textwidth}
    $i_7 = e_7^\uparrow = \emptyset$\\
    $i_2 = e_2^\uparrow = \{\textrm{4 nohy}\}$\\
    $i_3 = e_3^\uparrow = \{\textrm{4 nohy, mléko}\}$\\
    $i_1 = e_1^\uparrow = \{\textrm{2 nohy}\}$\\
    $i_4 = e_4^\uparrow = \{\textrm{4 nohy, mléko, vlna}\}$\\
    $i_5 = e_5^\uparrow = \{\textrm{4 nohy, mléko}\}$\\
    $i_6 = e_6^\uparrow = \{\textrm{2 nohy, 4 nohy, mléko, vlna}\}$\\
\end{minipage}

\begin{center}
    \begin{tikzpicture}[scale=.7]
        \node (e6) at (0,-4) {$\emptyset$; \{2 nohy, 4 nohy, mléko, vlna\}};
        \node (e5) at (-8,-2) {\{pes\}; \{4 nohy\}};
        \node (e4) at (0,-2) {\{ovce\}; \{4 nohy, mléko, vlna\}};
        \node (e1) at (8,-2) {\{slepice\}; \{2 nohy\}};
        \node (e3) at (0,0) {\{ovce, koza\}; \{4 nohy, mléko\}};
        \node (e2) at (0,2) {\{pes, ovce, koza\}; \{4 nohy\}};
        \node (e7) at (0,4) {\{pes, ovce, koza, slepice\}; $\emptyset$};
        \draw (e6) -- (e5) -- (e2) -- (e7) -- (e1) -- (e6) -- (e4) -- (e3) -- (e2);
    \end{tikzpicture}
\end{center}

\section{Svazy (pro doplnění)}
Svaz je algebra $(L, \cap, \cup)$ s dvěma základními binárními operacemi $x\cap{}y$ \textbf{spojení (suprémum)} (sup(x, y)) a $x\cup{}y$ \textbf{průsek (infimum)} (inf(x, y)), které mají následující vlastnosti:
\begin{enumerate}
    \item \textbf{Univerzalita (jednoznačnost)} -- $\forall x,y \,\exists z \,\,\, x \cap y = z\quad|\quad\forall x,y \exists\, z \,\,\, x \cup y = z$.
    \item \textbf{Asociativita} -- $x \cap (y \cap z) = (x \cap y) \cap z \quad|\quad x \cup (y \cup z) = (x \cup y) \cup z$.
    \item \textbf{Komutativita} -- $x \cap y = y \cap x \quad|\quad x \cup y = y \cup x$.
    \item \textbf{Absorbce} -- $x \cap (x \cup y) = x \quad|\quad x \cup (x \cap y) = x $.
\end{enumerate}

\subsection{Typy svazů}
\begin{enumerate}
    \item \textbf{Distributivní} -- platí zde axiomy distributivity a neobsahuje ani \textbf{diamant} ani \textbf{pentagon}: $x \cup (y \cap z) = (x \cup y) \cap (x \cup z) \quad|\quad x \cap (y \cup z) = (x \cap y) \cup (x \cap z) $.

          \begin{figure}[H]
              \centering
              \includegraphics[width=.4\textwidth]{assets/pentagon_diamant}
          \end{figure}
    \item \textbf{Modulární} -- slabší reprezentace distributivity, \textbf{nesmí} obsahovat \textbf{pentagon}, $a \geq c: a \land (b \lor c) = (a \land b) \lor c$.
    \item \textbf{Komplementární} -- platí zde, že pro každý prvek $ x $ existuje komplement $x' $, kdy: $x \cap x' = $ [svazová 0] a $x \cup x' = $ [svazová 1]
    \item \textbf{Booleovský svaz} -- komplementární $\land$ distributivní
\end{enumerate}

\subsection{Vlastnosti svazů}
Pro každé dva prvky v množině existuje \textbf{sup} a \textbf{inf}. \textbf{Úplný svaz} -- nastane tehdy, zda pro \textbf{libovolné neprázdné podmnožiny existuje} sup a inf. U svazů můžeme dále získat tyto vlastnosti:
\begin{itemize}
    \item \textbf{Minimum a maximum} -- žádný není menší/větší než $a$ (vrchol diagramu).
    \item \textbf{Nejmenší a největší} -- \textbf{pouze jeden} nejmenší/největší prvek, pokud je jich více, \textbf{neexistuje} největší/nejmenší prvek.
    \item \textbf{Dolní (L(A)) a horní (U(A)) závora} -- všechny prvky jsou $\leq/\geq$ než A,
    \item \textbf{Infimum} -- nejmenší prvek dolní závory.
    \item \textbf{Supremum} -- nejmenší prvek horní závory.
\end{itemize}
% \pagebreak

\chapter{Asociační pravidla, hledání často se opakujících množin položek.}
Termín asociační pravidla široce zpopularizoval počátkem 90. let v souvislosti s analýzou nákupního košíku. Při této analýze se zjišťuje, jaké druhy zboží si současně kupují zákazníci v supermarketech (např. pivo a párek). \textbf{Jde} tedy \textbf{o hledání vzájemných vazeb} (\textbf{asociací}) \textbf{mezi různými položkami} sortimentu prodejny. Přitom není upřednostňován žádný speciální druh zboží jako závěr pravidla.

\subsection{Základní charakteristika pravidel}
U pravidel vytvořených z dat nás obvykle zajímá kolik příkladů splňuje \textbf{předpoklad} a kolik \textbf{závěr} pravidla, kolik příkladů splňuje předpoklad i závěr \textbf{současně}, kolik příkladů splňuje předpoklad a \textbf{nesplňuje} závěr…. Tedy, zajímá nás, jak pro pravidlo:
\begin{equation}
    Ant \Rightarrow Suc, \quad\textrm{ kde } Ant, Suc \subseteq I \textrm{ (položky)}
\end{equation}
kde $ Ant $ (\textbf{předpoklad}, levá strana pravidla, \textbf{antecedent}) a $ Suc $ (\textbf{závěr}, pravá strana pravidla, \textbf{sukcedent}) jsou kombinace kategorií, pro něž příslušná \textbf{kontingenční tabulka} vypadá následovně:
\begin{table}[H]
    \centering
    \begin{tabular}{l|ll}
                     & $ Suc $ & $ \neg Suc $ \\\hhline
        $ Ant $      & a       & b            \\
        $ \neg Ant $ & c       & d
    \end{tabular}
\end{table}
\begin{itemize}
    \item $Ant \land Suc$ -- \textbf{a} je počet objektů {pokrytých současně předpokladem i závěrem},
    \item $Ant \land \neg Suc)$ -- \textbf{b} je počet objektů {pokrytých předpokladem a nepokrytých závěrem},
    \item $\neg Ant \land Suc)$ -- \textbf{c} je počet příkladů {nepokrytých předpokladem ale pokrytých závěrem},
    \item $\neg Ant \land \neg Suc)$ -- \textbf{d} je počet příkladů {nepokrytých ani předpokladem ani závěrem}.
\end{itemize}

\subsection{Základní charakteristiky asociačních pravidel}
\begin{itemize}
    \item \textbf{Support (podpora)} -- relativní četnost objektů splňující předpoklad i závěr, jinými slovy {\scriptsize$\frac{\textrm{počet splňující výstup}}{\textrm{počet položek}}$}:
          \begin{equation}
              sup(Ant \Rightarrow Suc) = \frac{a}{a + b + c + d}, \, \in \, \langle 0; 1 \rangle.
          \end{equation}
    \item \textbf{Confidence (spolehlivost)} -- podmíněná pravděpodobnost závěru pokud platí předpoklad, tedy {\scriptsize$\frac{\textrm{podpora obou}}{\textrm{podpora závěru}}$}:
          \begin{equation}
              conf(Ant \Rightarrow Suc) = \frac{sup(Ant \cup Suc)}{sup(Suc)} = \frac{a}{a + b}.
          \end{equation}
    \item Další: \textbf{pokrytí}, \textbf{zajímavost}, \textbf{závislost}.
\end{itemize}

\section{Hledání často se opakujících množin položek)}
Frequent item set je množina, kde $sup(K) \geq \gamma$, máme tedy stanovenou \textbf{minimální podporu}. Pokud je např. $\gamma = 0,3$ pak je minimální podpora 30\%.

\subsection{Generování kombinací}
Základem všech algoritmů pro hledání asociačních pravidel je \textbf{generování kombinací (konjunkcí) hodnot atributů}. Při generování vlastně procházíme (prohledáváme) prostor všech přípustných konjunkcí. Metod je několik:
\begin{itemize}
    \item do \textbf{hloubky},
    \item do \textbf{šířky},
    \item \textbf{heuristicky},
\end{itemize}

\subsection{Algoritmus apriori}
Jedná se o nejznámějším algoritmus pro hledání asociačních pravidel Jádrem algoritmu je \textbf{hledání často se opakujících množin} položek (frequent itemsets). Jedná se kombinace (konjunkce) kategorií, které dosahují předem zadané četnosti (\textbf{minimální podpory}) v datech.

\begin{figure}[H]
    \centering
    \includegraphics[width=0.6\textwidth]{assets/apriori.png}
\end{figure}

Při hledání kombinací délky $ k $, které mají vysokou četnost se využívá toho, že \textbf{již známe kombinace délky} $ k-1 $. Při vytváření kombinace délky $ k $ spojujeme kombinace délky $ k-1 $.

Jde tedy o \textbf{generování kombinací ,,do šířky''}. Přitom pro vytvoření jedné kombinace délky $ k $ požadujeme, aby všechny její podkombinace délky $ k-1 $ \textbf{splňovaly požadavek na četnosti}. Tedy např. ze tříčlenných kombinací $\{A_1A_2A_3,  \,A_1A_2A_4, \, A_1A_3A_4, \,A_1A_3A_5,  \,A_2A_3A_4\}$ dosahujících požadované četnosti vytvoříme \textbf{pouze jedinou čtyřčlennou} kombinaci $ A_1A_2A_3A_4 $. Kombinaci $ A_1A_3A_4A_5 $ sice lze vytvořit spojením $ A_1A_3A_4 $ a
$ A_1A_3A_5 $, ale mezi tříčlennými kombinacemi chybí $ A_1A_4A_5 $ i $ A_3A_4A_5 $.

\subsection{Algoritmus Next Closure}
Slouží k vytváření formálních kontextů, \textbf{vyhledáváním nejmenších intentů}, postup:
\begin{enumerate}
    \item Začnu s následující tabulkou:
          \begin{table}[H]
              \centering
              \begin{tabular}{l|l|l|l|l|p{6cm}}
                  \multirow{2}{*}{$ A $} & \multirow{2}{*}{$ i $} & \multirow{2}{*}{$ A \cap \{1 \ldots i - 1\} \cup \{i\} = B' $ } & \multirow{2}{*}{$ cl(B') = B $ } & \multirow{2}{*}{$B \setminus A $  } & je-li $B \setminus A = \{i\}$ nebo větší $\rightarrow$ ANO  \\
                                         &                        &                                                                 &                                  &                                     & je-li $B \setminus A = \{j\}$, kde $j < i$ $\rightarrow$ NE \\\hhline
                  $\emptyset$            & 5                      &                                                                 &                                  &                                     &
              \end{tabular}
          \end{table}
    \item Do $A$ vložím prázdnou množinu a $i$ nastavím na nejvyšší intent (pořadí).
    \item Udělám průnik s $A \cap \{1 \ldots i - 1\} \cup \{i\} = B'$ $\rightarrow$ průnik Ačka s intenty od $ 1 $ do $ i-1 $, k tomu přidám $i$. Př.: $A = \{1, 3, 4\}, i = \{3\} \Rightarrow B' = \{1, 3\}$.
    \item Udělám closure ($B'$) $\rightarrow$ intent na extent $\rightarrow$ extent na intent $\Rightarrow B'^{\downarrow \uparrow}$.
    \item Od $B$ odečtu $A$.
    \item Je-li:
          \begin{itemize}
              \item $B \setminus A = \{i\}$ a větší tak ANO [je-li nejmenší prvek z $B \setminus A$ roven nebo větší než $\{i\}$],
              \item $B \setminus A = \{j\}$ kde $j < i$ tak NE [je-li nejmenší prvek z $B \setminus A$ menší než i pak NE].
          \end{itemize}
    \item Pokud:
          \begin{itemize}
              \item ANO $\rightarrow$ do $A$ dosadíme $cl(B') = B$ a $i$ nastavíme na nejvyšší intent,
              \item NE $\rightarrow$ do $A$ neměním a snížíme $i$ o $-1$.
          \end{itemize}
    \item Skončím když je $A$ rovno celé množině extentů.
\end{enumerate}

\subsubsection{DÚLEŽITÉ} V $i$ přeskakuju hodnoty, které jsou v $A$ [výjde pro ně $\emptyset \rightarrow$ neřeším].

\subsection*{Příklad}
\begin{table}[H]
    \centering
    \begin{tabular}{l|lllll}
              & $y_1$ & $y_2$ & $y_3$ & $y_4$ & $y_5$ \\\hhline
        $x_1$ & 0     & 1     & 0     & 1     & 1     \\
        $x_2$ & 0     & 1     & 1     & 0     & 0     \\
        $x_3$ & 1     & 1     & 0     & 1     & 1     \\
        $x_4$ & 1     & 1     & 1     & 0     & 0     \\
        $x_5$ & 1     & 0     & 1     & 1     & 0     \\
        $x_6$ & 1     & 0     & 0     & 1     & 0
    \end{tabular}
\end{table}


\begin{tabular}{l|l|l|l|l|p{3cm}}
    \multirow{2}{*}{$ A $} & \multirow{2}{*}{$ i $} & \multirow{2}{*}{$ A \cap \{1 \ldots i - 1\} \cup \{i\} = B' $ } & \multirow{2}{*}{$ cl(B') = B $ } & \multirow{2}{*}{$B \setminus A $  } & \multirow{2}{*}{ANO / NE }             \\
                           &                        &                                                                 &                                  &                                     &                                        \\ \hhline
    $\emptyset$            & 5                      & 5                                                               & 2, 4, 5                          & 2, 4, 5                             & N [2 < 5]                              \\
    $\emptyset$            & 4                      & 4                                                               & 4                                & 4                                   & A [4 $\geq$ 4] $\leftarrow$ $i_n$      \\
    4                      & 5                      & 4, 5                                                            & 2, 4, 5                          & 2, 5                                & N [2 < 5]                              \\
    4                      & 3 [skip $i$]           & 3                                                               & 3                                & 3                                   & A [3$\geq$ 3] $\leftarrow$ $i_{n - 1}$ \\
    3                      & 5                      & 3, 5                                                            & 1, 2, 3, 4, 5                    & 1, 2, 4, 5                          & N [1 < 5]                              \\
                           &                        & \vdots                                                          &                                  &                                     &                                        \\
    1, 2, 3, 4, 5          &                        & KONEC                                                           &                                  &                                     & A $i_{1}$
\end{tabular}

\chapter{Metrické a topologické prostory – metriky a podobnosti.}
\section{Structure From Motion}
Metoda Structure From Motion se zabývá převodem sady snímků objektu do jeho reprezentace ve 3D. Pro získání hloubkové mapy z obrazu jsou potřeba minimálně dva snímky, pro kompletní rekonstrukci 3D modelu je poté potřeba pracovat s co největší datovou sadou. Je potřeba mít snímky vytvořené z různých úhlů, různých vzdáleností. \par
V podstatě se celý proces dá shrnout do následujících kroků: Po vytvoření snímků je potřeba danou datovou sadu zpracovat. Nejprve jsou v obraze nalezeny klíčové body - tzn. nějaké body zájmu. Např. hrany, rohy... Následně je potřeba najít koresponcence mezi body zájmu v jednotlivých obrazech. Po nalezení korespondencí je možné zpětně vypočítat pozici kamer vůči jednotlivými body. Výsledkem je poté mračno bodů, ze kterého lze pak aproximovat povrch tělesa.

Lidé vnímají mnoho informací o trojrozměrné struktuře ve svém prostředí tím, že se kolem ní pohybují. Když se pozorovatel pohybuje, objekty kolem nich se pohybují různě, v závislosti na jejich vzdálenosti od pozorovatele. Toto je známé jako pohybová paralaxa a z této hloubky lze informace použít ke generování přesné 3D reprezentace světa kolem nich.

Hledání struktury z pohybu představuje podobný problém jako hledání struktury ze stereofonního vidění. V obou případech je třeba najít shodu mezi obrázky a rekonstrukcí 3D objektu.

Chceme-li najít korespondenci mezi obrázky, sledují se příznaky - body zájmu jako rohové body (hrany s přechody ve více směrech) z jednoho obrázku na druhý. Jedním z nejpoužívanějších detektorů příznaků je transformace příznaků invariantních vůči měřítku (SIFT - scale-invariant feature transform). Jako rysy používá maxima z pyramidy DOG (Difference-of-Gaussians). Prvním krokem v SIFT je nalezení dominantního směru gradientu. Aby byla rotace invariantní, deskriptor se otočí tak, aby odpovídal této orientaci. Dalším běžným detektorem vlastností je SURF (speeded-up robust features). V SURF je DOG nahrazen detektorem tzv. blobů na bázi hesenské matice. Místo vyhodnocení gradientních histogramů SURF také počítá pro součty složek gradientu a součty jejich absolutních hodnot. Jeho použití integrálních obrazů umožňuje extrémně rychle detekovat příznaky s vysokou mírou detekce. Ve srovnání s SIFT je tedy SURF rychlejším detektorem vlastností s nevýhodou menší přesnosti v pozicích příznaků. Dalším typem příznaku, jsou obecné křivky (např. lokální hrany s přechody v jednom směru, součást technologie známé jako pointless SfM) užitečné, když jsou bodové příznaky nedostatečné, běžné v umělých prostředích.

Příznaky detekované ze všech obrázků budou poté porovnány. Jedním z algoritmů shody, který sleduje příznaky z jednoho obrázku na druhý, je Lukas – Kanade tracker.

Někdy jsou některé z uzavřených příznaků nesprávně spárovány. Z tohoto důvodu by měly být zápasy filtrovány. RANSAC (konsensus náhodných vzorků) je algoritmus, který se obvykle používá k odstranění odlehlých korespondencí. V příspěvku Fischlera a Bollese se RANSAC používá k řešení problému určování polohy (LDP), kde cílem je určit body v prostoru, které se promítají na obraz do souboru orientačních bodů se známými místy. 

Trajektorie příznaků v čase se poté použijí k rekonstrukci jejich 3D pozic a pohybu kamery. Alternativu dávají takzvané přímé přístupy, kdy jsou geometrické informace (3D struktura a pohyb kamery) přímo odhadovány z obrazů, bez mezilehlé abstrakce na příznaky nebo rohy.

Existuje několik přístupů ke struktuře z pohybu. V přírůstkovém SFM jsou pozice kamer řešeny a přidávány jeden po druhém do datov0 sady. V globálním SFM jsou pozice všech kamer řešeny současně. Kompromisem je out-of-core SFM, kde se počítá několik částečných rekonstrukcí, které jsou poté integrovány do globálního řešení. 

\begin{figure}[H]
 \centering
 \includegraphics[width=0.4\textwidth]{assets/10_SFM.png}
\end{figure}

\section{Structured Light}
Na sledovaný objekt jsou vysílány různě velké liniové paprsky, které se zpětně snímají kamerou. Podle zakřívení povrchu je vyslaný paprsek deformován. Ze snímků deformovaných paprsků je pak možné provést triangulaci a následně zpětně rekonstruovat povrch tělesa. 

3D skener strukturovaného světla je zařízení pro měření trojrozměrného tvaru objektu užitímpromítaného svetelného vzoru a kamerového systému.V dnešní době je strukturované světlo běžně využíváno pro různé trojrozměrné profilometrické zkoumání povrchů díky jeho nízké cenně a vysoké rychlosti, proto je jedním z nejčastějších způsobů, jak velmi rychle a efektivně zjistit žádané informace o povrchu objektu. Měřící čidla na bázi strukturovaného světla (SL – structured light) jsou využívány v mnoha odvětvích, kromě strojírenství a kontroly kvality je to např. k uchování uměleckých děl, v zábavním průmyslu, medicíně nebo zabezpečení. Je to především pro jeho bezkonkurenční rozlišení, bezkontaktní zajištění rekonstrukce celého pole objektů ve vysokém rozlišení a rychlost. K výhodám se dá také jistě zařadit kompaktnost skenerů, jelikož měřící proces je realizován pouze profilometrickým systémem, který se skládá z jednotky pro zpracování a analýzu (PC), projekční jednotky (zpravidla videoprojektoru) a vizuální jednotky(CCD/CMOS kamera).Při zaznamenávání stojícího objektu je nutné, aby byl objekt stabilní. Využitím zařízení se strukturovaným světlem lze zaznamenat i geometrii pohybujícíchse těles, to se však děje na úkor snížení kvality digitalizace. Při tomto dynamickém skenování objektů se ukázalo, že základní vzory jsou nedostačující, proto začaly vznikat strukturálně složitější vzory.
\subsection{Princip}
Promítání uzkých pásů světla na trojrozměrně tvarovaný povrch vytváří linie osvětlení, které jsou zkreslené z jiného úhlu než se nachází projektor a mohou být použity pro přesnou geometrickou rekonstrukci tvaru povrchu. Používá se mnoho různých variant strukturovaného světla, ale vodorovné pruhy jsou nejčastější.


\begin{figure}[H]
    \centering
    \includegraphics[width=0.4\textwidth]{assets/10_3-proj2cam.svg.png}
    \includegraphics[width=0.4\textwidth]{assets/10_triangulaceSL.png}
\end{figure}


Modulovaný vzor je porovnán se vzorem promítaným, tj. porovnávají se odpovídající si pixely projektoru a snímacího zařízení. Výsledkem tohoto porovnání a využitím vhodného algoritmu se vytvoří body orientované v prostoru, tzv. mračna bodů.S těmito body lze dále pracovat a vytvořit souvislý povrch.
% \pagebreak

\chapter{Shlukování.}
\input{1_it_mpzz/includes/10.tex}
% \pagebreak

\chapter{Náhodná veličina. Základní typy náhodných veličin. Funkce určující rozdělení náhodných veličin.}
\input{1_it_mpzz/includes/11.tex}
% \pagebreak

\chapter{Vybraná rozdělení diskrétní a spojité náhodné veličiny - binomické, hypergeometrické, negativně binomické, Poissonovo, exponenciální, Weibullovo, normální rozdělení.}
\input{1_it_mpzz/includes/12.tex}
% \pagebreak

\chapter{Popisná statistika. Číselné charakteristiky a vizualizace kategoriálních a kvantitativních proměnných.}
\input{1_it_mpzz/includes/13.tex}
% \pagebreak

\chapter{Metody statistické indukce. Intervalové odhady. Princip testování hypotéz.}
\input{1_it_mpzz/includes/14.tex}
% \pagebreak

\part{Softwarové inženýrství}

\chapter{Softwarový proces. Jeho definice, modely a úrovně vyspělosti.}
% Systémy barev v počítačové grafice, nelinearita grafického výstupu (gamma korekce), kompozice rastrových obrazů (alfa kanál), HDR.
\section{Systémy barev v PG}
\begin{itemize}
    \item Základem barevného prostoru je \textbf{barevný model}, který nám dává abstraktní matematický popis, jak lze barvy vyjádřit pomocí n-tic čísel, nejčastěji trojic.
    \item Mezi nejznámější barevné modely v dnešní době patří \textbf{RGB model}.
    \item Model RGB pracuje se třemi základními barvami: \textbf{červenou, zelenou} a \textbf{modrou}, z nichž se odvíjí i jeho název.
    \item Tyto barvy byly zvoleny na základně toho, jak \textbf{čípky v lidském oku} vnímají jednotlivé záření.
    \item Zároveň je RGB \textbf{aditivní barevný model}, což znamená, že se jednotlivé barevné složky \textbf{míchají} (nové barvy získáváme přidáváním větší intenzity jednotlivých složek) a výsledkem jsou další barevné odstíny, případně vyšší intenzita barvy.
    \item Když k tomuto modelu definujeme, jak mají být tyto n-tice interpretovány, dostáváme \textbf{barevný prostor} -- je předem definovaná množina barev, kterou je schopno určité zařízení snímat, zobrazit nebo reprodukovat.
    \item Barevný prostor je tedy \textbf{definován rozsahem barev}, které dokáže zobrazit.
    \item Tomuto rozsahu se také říká \textbf{gamut}. Ten se zpravidla zobrazuje jako oblast v CIE 1931 chromatickém diagramu
\end{itemize}
\begin{figure}[H]
    \centering
    \includegraphics[width=0.3\textwidth]{assets/1_rgb_gamut}
\end{figure}
\subsection{RGB}
\begin{itemize}
    \item Nejrozšířenější barevný prostor postavený na RGB barevném modelu je \textbf{sRGB} - standardní RGB.
    \item Jeho určení je pro zobrazování \textbf{na monitorech} nebo \textbf{kódování barev} na internetu.
    \item Pro všechny tři barevné složky má definované barvy v \textbf{chromatickém diagramu}, které vymezují jeho gamut.
    \item Každá barva, kterou tento prostor zobrazuje, je dána zastoupením jednotlivých barevných složek, buďto relativně (hodnoty jsou v rozmezí 0 - 1) nebo absolutně (konkrétní \uv{bitové} hodnoty, zpravidla 0 - 255, 24-bitů).
    \item RGB je možné zobrazit jako krychli.
    \item Často se přidává \textbf{Alpha kanál} pro průhlednost - \textbf{RGBA} (32-bitů).
          \begin{figure}[H]
              \centering
              \includegraphics[width=0.6\textwidth]{assets/1_rgb_gamut_krychle}
          \end{figure}
\end{itemize}

\subsection{HSV a HSL}
\begin{itemize}
    \item \textbf{Hue, Saturation, Value/Lightness} - barevný model, který nejvíce odpovídá lidskému vnímání barev.
    \item Barvy popisuje pomocí 3 hodnot, které však samy barvy nereprezentují:
          \begin{itemize}
              \item \textbf{Hue} - \textbf{barevný tón}, převládající. Neboli \textbf{odstín} - barva \textbf{odražená} nebo \textbf{procházející} objektem. Měří se jako poloha na standardním barevném kole (\ang{0} až \ang{360}). Obecně se odstín označuje názvem barvy. \ang{0} - červená, \ang{120} - zelená, \ang{240} - modrá.
              \item \textbf{Saturation} - \textbf{sytost} barvy, příměs jiné barvy. Někdy též chroma, síla nebo čistota barvy, představuje množství šedi v poměru k odstínu, měří se v procentech od 0\% (šedá) do 100\% (plně sytá barva). Na barevném kole vzrůstá sytost od středu k okrajům.
              \item \textbf{Value} - \textbf{hodnota jasu}, množství bílého světla. Relativní světlost nebo tmavost barvy. Jas vyjadřuje \textbf{kolik světla barva odráží}, dalo by se také říct přidávání černé do základní barvy.
          \end{itemize}
    \item Nejčastěji se tato reprezentace (popř. \textbf{HSL}) používají v grafických nástrojích jako komponenty pro výběr barvy, protože je mnohem intuitivnější než RGB.
    \item Vyberu si odstín, jak má být sytý a jasný a hotovo. Není třeba řešit jak smíchat 3 barevné složky, abych dostal to co chci.
    \item Dále se využívá často v případě detekce objektů, kdy hodnota HUE (odstín), je nezávislý na osvětlení scény. Problém však nastává u bílých a černých objektů (kdy HUE může být různé), ty lze na základě value a staturation mapovat do podobných barev (žlutá a černá).
    \item Mimo níže uvedené zobrazení \textbf{válcem}, lze také zobrazit \textbf{kuželem} a
          \begin{figure}[H]
              \centering
              \includegraphics[width=0.9\textwidth]{assets/1_hsv_hsl}
          \end{figure}
\end{itemize}

\subsection{CMY a CMYK}
\begin{itemize}
    \item Substraktivní barevné systémy (barvy se ,,\textbf{odečítají}'' od bílé, přidáváním jednotlivých složek až po černou), \textbf{C}yan, \textbf{M}agenta, \textbf{Y}ellow a \textbf{K}ey (Blac\textbf{K}).
    \item Používá se \textbf{pro tisk}.
    \item Černá se přidala, protože smíchání CMY nedává plně černou barvu, navíc je černý inkoust levnější než barevný.
    \item Nevýhodou je, že \textbf{nedokáže správně zobrazit} sytě červenou, zelenou a modrou.
    \item Při tisku to však není poznat.
    \item Před tiskem se RGB obraz převádí do CMYK.
    \item To provádí buďto ovladač tiskárny nebo RIP (Raster Image Processor - u profi tiskáren).
    \item RGB se používá pro aktivní zdroje světla, CMYK jsou \textbf{pasivní} (světlo pouze \textbf{odrážejí}), proto nedokáží udělat tak jasné odstíny.
\end{itemize}
\begin{figure}[H]
    \centering
    \includegraphics[width=0.6\textwidth]{assets/1_cmyk}
\end{figure}

\subsection{YCbCr}
\begin{itemize}
    \item Barva je reprezentována \textbf{jasovou složkou} Y a modrou a červenou \textbf{chrominanční} komponentou.
    \item Není to absolutní barevný model, jedná se o \textbf{způsob kódování RGB} informací.
    \item Využívá se nejčastěji u videa a barevných obrázků, kde je využito faktu, že \textbf{lidské oko nejvíce vnímá jas}, který je reprezentovaný složkou Y. Barvy už tak důležité nejsou a proto se můžou například více \textbf{komprimovat} bez výraznější ztráty kvality obrazu (JPEG).
    \item Jasová složka je kódována v intervalu $\langle0, 1\rangle$ a chrominanční složky v intervalu $\langle-0.5, 0.5\rangle$
\end{itemize}

\section{Nelinearita grafického výstupu (gamma korekce)}
První CRT monitory zobrazovaly jas nelineárně. To znamená, že dvojnásobné napětí neznamená dvojnásobný jas, křivka jasu byla zhruba exponenciální. Tento způsob zobrazení jasu přetrvává i v dnešních monitorech. Proto je potřeba upravit i zobrazované barvy. Mapování barev je potřeba provést nelineárně. Nelineární vstup v kombinaci s exponenciální křivkou jasu ve výsledku vede k k jasu, který je vnímán jako lineární. 
\begin{figure}[H]
    \centering
    \includegraphics[width=0.3\textwidth]{assets/1_gamma_correction_gamma_curves.png}
\end{figure}
Problém nastává při práci s barvami. Pracovat s barvami je potřeba v lineárním prostoru, aby byly výsledky např. matematických operací korektní. Před zpracováním je tedy nutné převést barvu do lineárního prostou, po zpracování je potřeba před zobrazením opět převézt barvu zpět do nelineárního prostoru.

\begin{equation}
    C_{linear} = C_{sRGB}^{\text{gamma}}
\end{equation}

\begin{equation}
    C_{sRGB} = C_{linear}^{\frac{1}{\text{gamma}}}
\end{equation}

\section{Kompozice rastrových obrazů (alfa kanál)}
Alfa kanál v obrazech se používá v počítačové grafice pro průhlednost obrazů. Obrazy mohou obsahovat plnou nebo částečnou průhlednost. Plně transparentní obraz propouští veškeré barvy podkladu, částečně transparentní obraz mixuje část bravy podkladu a část vlastní barvy. Např. výsledná barva obrazu s hodnotou alfa 0.5 vytvoří mix barev tvořený z $50\%$ barvou podkladu a z $50\%$ barvou transparentního objektu. V případě překryvu více transparentních objektů je potřeba objekty setřídit dle vzdálenosti a vypočítat transparentnost postupně.   

\section{HDR}
Barvy jsou tradičně reprezentovány trojicí 8bitových hodnot v intervalu  $\langle-0, 255\rangle$. Takto popsaný obraz ale může ztrácet detaily z důvodu limitovaných možností hodnot pro rozložení barev. Přesnější metodou je ukládání jednotlivých složek jako 16 nebo 32bitových hodnot, pomocí desetiných čísel. Zde ale nastává problém s hodnotami většími než 1. Je teda nutné provést tzv. tone mapping. Proces tone mapping v zásadě převádí hodnoty do intervalu $\langle0, 1\rangle$. \par
Jednou z nejjednodušších metod je Reinhardova metoda, hodnota se vypočítá následujícím vzorcem: 
\begin{equation}
    C_{out} = \frac{C_{in}}{C_{in} + 1}    
\end{equation}
Tato metoda zachová poměrně dobře kontrast pro oblasti obrazu s nízkým jasem, oblasti obrazu s vysokým jasem jsou ale méně kontrastní. 
Pokročilejší metodou je mapování hodnot pomocí expozice, následujícím vzorcem:
\begin{equation}
    C_{out} = 1 - e^{C_{in} * \exp}
\end{equation}
Pomocí hodnoty expozice je pak možné nastavit celkové podání barev obrazu. Hodnota expozice by měla být zvolena v závislosti na aktuálním vstupu, je možné hodnotu expozice volit automaticky. 
% \pagebreak

\chapter{Vymezení fáze „sběr a analýza požadavků“. Diagramy UML využité v dané fázi.}
\input{2_swi/includes/2.tex}
% \pagebreak

\chapter{Vymezení fáze „Návrh“. Diagramy UML využité v dané fázi. Návrhové vzory – členění, popis a příklady. }
\input{2_swi/includes/3.tex}
% \pagebreak

\chapter{Objektově orientované paradigma. Pojmy třída, objekt, rozhraní. Základní vlastnosti objektu a vztah ke třídě. Základní vztahy mezi třídami a rozhraními. Třídní vs. instanční vlastnosti.}
\input{2_swi/includes/4.tex}
% \pagebreak

\chapter{Mapování UML diagramů na zdrojový kód.}
\input{2_swi/includes/5.tex}
% \pagebreak

\chapter{Správa paměti(v jazycích C/C++, Java , C\#, Python), virtuální stroj, podpora paralelního zpracování a vlákna.}
\input{2_swi/includes/6.tex}
% \pagebreak

\chapter{Zpracování chyb v moderních programovacích jazycích, princip datových proudů – pro vstup a výstup. Rozdíl mezi znakově a bytově orientovanými datovými proudy.}
\input{2_swi/includes/7.tex}
% \pagebreak

\chapter{Jazyk UML – typy diagramů a jejich využití v rámci vývoje.}
\section{Deep learning}
\begin{itemize}
    \item Deep learning neboli \textbf{hluboké učení}, známé také jako hierarchické učení, je \textbf{sbírka algoritmů} používaných ve strojovém učení.
    \item Používají se k~modelování abstrakcí na vysoké úrovni v~datech za pomocí modelových architektur, které se skládají z~několika nelineárních transformací.
    \item Hluboké učení je součástí široké skupiny metod používané pro strojové učení, které jsou založeny na učení reprezentace dat.
\end{itemize}
Hluboké strukturované učení může být:
\begin{itemize}
    \item{\textbf{Kontrolované (s~učitelem)} - všechna data jsou kategorizovaná do tříd, algoritmy se učí předpovídat výstup ze vstupních dat.}
    \item{\textbf{Částečně kontrolované} - data jsou částečně kategorizovaná do tříd. Pří tomto přístupu učení lze využít kombinaci kontrolovaného a~nekontrolovaného přístupu učení.}
    \item{\textbf{Nekontrolované (bez učitele)} - data nejsou kategorizovaná do tříd, algoritmy se učí ze struktury vstupních dat.}
\end{itemize}
Hluboké učení je specifický přístup, použitý k~budování a~učení neuronových sítí, které jsou považovány za velmi spolehlivé rozhodovací uzly. Jestliže vstupní data algoritmu procházejí řadou nelinearit a~nelineárních transformací, tak tento algoritmus je považován za \uv{deep} algoritmus.

Odstraňuje také ruční identifikaci příznaků (obrázek \ref{fig:ml_vs_ann}) z~dat a~místo toho se spoléhá na jakýkoliv trénovací proces, které má za úkol zjistit užitečné vzory ve vstupních příkladech. To dělá neuronovou síť jednodušší a~rychlejší, a~může přinést lepší výsledky než z~oblasti umělé inteligence.

\begin{figure}[H]
    \centering
    \includegraphics[width=.5\linewidth]{assets/9_ml_vs_ann}
    \caption{Hlavním rozdílem mezi strojovým a~hlubokým učením je ten, že u~strojového se příznaky musí extrahovat manuálně.}
    \label{fig:ml_vs_ann}
\end{figure}

\subsection{Konvoluční neuronové sítě \textit{CNN} - Convolution neural network}
\begin{itemize}
    \item Speciálním druhem vícevrstvých neuronových sítí a~jsou navrženy tak, aby rozpoznaly vizuální vzory přímo z~pixelu obrazu s~minimálním předzpracováním.
    \item Mohou rozpoznat vzory s~extrémní variabilitou (například ručně psané znaky) a~odolnost vůči deformacím a~jednoduchým geometrickým transformacím.
    \item Síť využívá matematickou operaci zvanou konvoluce alespoň v~jedné jejich vrstvě.
\end{itemize}

Nejznámější a~nejvíce používanou konvoluční neuronovou sítí jsou modely LeNet.
Hlavní kroky LeNet sítě jsou:
\begin{itemize}
    \item{\textbf{Konvoluce} - tyto vrstvy provádějí konvoluci nad vstupy do neuronové sítě.}
    \item{\textbf{Nelinearita (ReLU)} - tato vrstva je použita po každé konvoluční vrstvě a~jejím cílem je nahrazení všech negativních pixelů nulou ve výstupu této vrstvy (příznaková mapa).}
    \item{\textbf{Pooling/sub sampling} - ze vstupního obrazu vyextrahuje pouze zajímavé části pomocí některých matematických operací (max, avg, sum), a~tím se \textbf{redukuje jeho dimenzionalita}.}
    \item{\textbf{Fully connected layer/klasifikace} - tato vrstva vychází z~původních umělých neuronových sítí, konkrétně z~vícevrstvého perceptronu. Tato vrstva je typicky umístěna na konci sítě a~je propojena s~klasifikační vrstvou pro predikci.}
\end{itemize}
\begin{figure}[H]
    \centering
    \includegraphics[width=.9\linewidth]{assets/9_cnn.pdf}
    \caption{Řetězec LeNet konvoluční neuronové sítě}
    \label{fig:cnn}
\end{figure}
% \pagebreak

\part{Databázové a informační systémy}

\chapter{Modelování databázových systémů, konceptuální modelování, datová analýza, funkční analýza; nástroje a modely. }
% Systémy barev v počítačové grafice, nelinearita grafického výstupu (gamma korekce), kompozice rastrových obrazů (alfa kanál), HDR.
\section{Systémy barev v PG}
\begin{itemize}
    \item Základem barevného prostoru je \textbf{barevný model}, který nám dává abstraktní matematický popis, jak lze barvy vyjádřit pomocí n-tic čísel, nejčastěji trojic.
    \item Mezi nejznámější barevné modely v dnešní době patří \textbf{RGB model}.
    \item Model RGB pracuje se třemi základními barvami: \textbf{červenou, zelenou} a \textbf{modrou}, z nichž se odvíjí i jeho název.
    \item Tyto barvy byly zvoleny na základně toho, jak \textbf{čípky v lidském oku} vnímají jednotlivé záření.
    \item Zároveň je RGB \textbf{aditivní barevný model}, což znamená, že se jednotlivé barevné složky \textbf{míchají} (nové barvy získáváme přidáváním větší intenzity jednotlivých složek) a výsledkem jsou další barevné odstíny, případně vyšší intenzita barvy.
    \item Když k tomuto modelu definujeme, jak mají být tyto n-tice interpretovány, dostáváme \textbf{barevný prostor} -- je předem definovaná množina barev, kterou je schopno určité zařízení snímat, zobrazit nebo reprodukovat.
    \item Barevný prostor je tedy \textbf{definován rozsahem barev}, které dokáže zobrazit.
    \item Tomuto rozsahu se také říká \textbf{gamut}. Ten se zpravidla zobrazuje jako oblast v CIE 1931 chromatickém diagramu
\end{itemize}
\begin{figure}[H]
    \centering
    \includegraphics[width=0.3\textwidth]{assets/1_rgb_gamut}
\end{figure}
\subsection{RGB}
\begin{itemize}
    \item Nejrozšířenější barevný prostor postavený na RGB barevném modelu je \textbf{sRGB} - standardní RGB.
    \item Jeho určení je pro zobrazování \textbf{na monitorech} nebo \textbf{kódování barev} na internetu.
    \item Pro všechny tři barevné složky má definované barvy v \textbf{chromatickém diagramu}, které vymezují jeho gamut.
    \item Každá barva, kterou tento prostor zobrazuje, je dána zastoupením jednotlivých barevných složek, buďto relativně (hodnoty jsou v rozmezí 0 - 1) nebo absolutně (konkrétní \uv{bitové} hodnoty, zpravidla 0 - 255, 24-bitů).
    \item RGB je možné zobrazit jako krychli.
    \item Často se přidává \textbf{Alpha kanál} pro průhlednost - \textbf{RGBA} (32-bitů).
          \begin{figure}[H]
              \centering
              \includegraphics[width=0.6\textwidth]{assets/1_rgb_gamut_krychle}
          \end{figure}
\end{itemize}

\subsection{HSV a HSL}
\begin{itemize}
    \item \textbf{Hue, Saturation, Value/Lightness} - barevný model, který nejvíce odpovídá lidskému vnímání barev.
    \item Barvy popisuje pomocí 3 hodnot, které však samy barvy nereprezentují:
          \begin{itemize}
              \item \textbf{Hue} - \textbf{barevný tón}, převládající. Neboli \textbf{odstín} - barva \textbf{odražená} nebo \textbf{procházející} objektem. Měří se jako poloha na standardním barevném kole (\ang{0} až \ang{360}). Obecně se odstín označuje názvem barvy. \ang{0} - červená, \ang{120} - zelená, \ang{240} - modrá.
              \item \textbf{Saturation} - \textbf{sytost} barvy, příměs jiné barvy. Někdy též chroma, síla nebo čistota barvy, představuje množství šedi v poměru k odstínu, měří se v procentech od 0\% (šedá) do 100\% (plně sytá barva). Na barevném kole vzrůstá sytost od středu k okrajům.
              \item \textbf{Value} - \textbf{hodnota jasu}, množství bílého světla. Relativní světlost nebo tmavost barvy. Jas vyjadřuje \textbf{kolik světla barva odráží}, dalo by se také říct přidávání černé do základní barvy.
          \end{itemize}
    \item Nejčastěji se tato reprezentace (popř. \textbf{HSL}) používají v grafických nástrojích jako komponenty pro výběr barvy, protože je mnohem intuitivnější než RGB.
    \item Vyberu si odstín, jak má být sytý a jasný a hotovo. Není třeba řešit jak smíchat 3 barevné složky, abych dostal to co chci.
    \item Dále se využívá často v případě detekce objektů, kdy hodnota HUE (odstín), je nezávislý na osvětlení scény. Problém však nastává u bílých a černých objektů (kdy HUE může být různé), ty lze na základě value a staturation mapovat do podobných barev (žlutá a černá).
    \item Mimo níže uvedené zobrazení \textbf{válcem}, lze také zobrazit \textbf{kuželem} a
          \begin{figure}[H]
              \centering
              \includegraphics[width=0.9\textwidth]{assets/1_hsv_hsl}
          \end{figure}
\end{itemize}

\subsection{CMY a CMYK}
\begin{itemize}
    \item Substraktivní barevné systémy (barvy se ,,\textbf{odečítají}'' od bílé, přidáváním jednotlivých složek až po černou), \textbf{C}yan, \textbf{M}agenta, \textbf{Y}ellow a \textbf{K}ey (Blac\textbf{K}).
    \item Používá se \textbf{pro tisk}.
    \item Černá se přidala, protože smíchání CMY nedává plně černou barvu, navíc je černý inkoust levnější než barevný.
    \item Nevýhodou je, že \textbf{nedokáže správně zobrazit} sytě červenou, zelenou a modrou.
    \item Při tisku to však není poznat.
    \item Před tiskem se RGB obraz převádí do CMYK.
    \item To provádí buďto ovladač tiskárny nebo RIP (Raster Image Processor - u profi tiskáren).
    \item RGB se používá pro aktivní zdroje světla, CMYK jsou \textbf{pasivní} (světlo pouze \textbf{odrážejí}), proto nedokáží udělat tak jasné odstíny.
\end{itemize}
\begin{figure}[H]
    \centering
    \includegraphics[width=0.6\textwidth]{assets/1_cmyk}
\end{figure}

\subsection{YCbCr}
\begin{itemize}
    \item Barva je reprezentována \textbf{jasovou složkou} Y a modrou a červenou \textbf{chrominanční} komponentou.
    \item Není to absolutní barevný model, jedná se o \textbf{způsob kódování RGB} informací.
    \item Využívá se nejčastěji u videa a barevných obrázků, kde je využito faktu, že \textbf{lidské oko nejvíce vnímá jas}, který je reprezentovaný složkou Y. Barvy už tak důležité nejsou a proto se můžou například více \textbf{komprimovat} bez výraznější ztráty kvality obrazu (JPEG).
    \item Jasová složka je kódována v intervalu $\langle0, 1\rangle$ a chrominanční složky v intervalu $\langle-0.5, 0.5\rangle$
\end{itemize}

\section{Nelinearita grafického výstupu (gamma korekce)}
První CRT monitory zobrazovaly jas nelineárně. To znamená, že dvojnásobné napětí neznamená dvojnásobný jas, křivka jasu byla zhruba exponenciální. Tento způsob zobrazení jasu přetrvává i v dnešních monitorech. Proto je potřeba upravit i zobrazované barvy. Mapování barev je potřeba provést nelineárně. Nelineární vstup v kombinaci s exponenciální křivkou jasu ve výsledku vede k k jasu, který je vnímán jako lineární. 
\begin{figure}[H]
    \centering
    \includegraphics[width=0.3\textwidth]{assets/1_gamma_correction_gamma_curves.png}
\end{figure}
Problém nastává při práci s barvami. Pracovat s barvami je potřeba v lineárním prostoru, aby byly výsledky např. matematických operací korektní. Před zpracováním je tedy nutné převést barvu do lineárního prostou, po zpracování je potřeba před zobrazením opět převézt barvu zpět do nelineárního prostoru.

\begin{equation}
    C_{linear} = C_{sRGB}^{\text{gamma}}
\end{equation}

\begin{equation}
    C_{sRGB} = C_{linear}^{\frac{1}{\text{gamma}}}
\end{equation}

\section{Kompozice rastrových obrazů (alfa kanál)}
Alfa kanál v obrazech se používá v počítačové grafice pro průhlednost obrazů. Obrazy mohou obsahovat plnou nebo částečnou průhlednost. Plně transparentní obraz propouští veškeré barvy podkladu, částečně transparentní obraz mixuje část bravy podkladu a část vlastní barvy. Např. výsledná barva obrazu s hodnotou alfa 0.5 vytvoří mix barev tvořený z $50\%$ barvou podkladu a z $50\%$ barvou transparentního objektu. V případě překryvu více transparentních objektů je potřeba objekty setřídit dle vzdálenosti a vypočítat transparentnost postupně.   

\section{HDR}
Barvy jsou tradičně reprezentovány trojicí 8bitových hodnot v intervalu  $\langle-0, 255\rangle$. Takto popsaný obraz ale může ztrácet detaily z důvodu limitovaných možností hodnot pro rozložení barev. Přesnější metodou je ukládání jednotlivých složek jako 16 nebo 32bitových hodnot, pomocí desetiných čísel. Zde ale nastává problém s hodnotami většími než 1. Je teda nutné provést tzv. tone mapping. Proces tone mapping v zásadě převádí hodnoty do intervalu $\langle0, 1\rangle$. \par
Jednou z nejjednodušších metod je Reinhardova metoda, hodnota se vypočítá následujícím vzorcem: 
\begin{equation}
    C_{out} = \frac{C_{in}}{C_{in} + 1}    
\end{equation}
Tato metoda zachová poměrně dobře kontrast pro oblasti obrazu s nízkým jasem, oblasti obrazu s vysokým jasem jsou ale méně kontrastní. 
Pokročilejší metodou je mapování hodnot pomocí expozice, následujícím vzorcem:
\begin{equation}
    C_{out} = 1 - e^{C_{in} * \exp}
\end{equation}
Pomocí hodnoty expozice je pak možné nastavit celkové podání barev obrazu. Hodnota expozice by měla být zvolena v závislosti na aktuálním vstupu, je možné hodnotu expozice volit automaticky. 
% \pagebreak

\chapter{Relační datový model, SQL; funkční závislosti, dekompozice a normální formy.}
\input{3_dais/includes/2.tex}
% \pagebreak

\chapter{Transakce, zotavení, log, ACID, operace COMMIT a ROLLBACK; problémy souběhu, řízení souběhu: zamykání, úroveň izolacev SQL.}
\input{3_dais/includes/3.tex}
% \pagebreak

\chapter{Procedurální rozšíření SQL, PL/SQL, T-SQL, triggery, funkce, procedury, kurzory, hromadné operace.}
\input{3_dais/includes/4.tex}
% \pagebreak

\chapter{Základní fyzická implementace databázových systémů: tabulky a indexy; plán vykonávání dotazů.}
\input{3_dais/includes/5.tex}
% \pagebreak

\chapter{Objektově‐relační datový model a XML datový model: principy, dotazovací jazyky.}
\input{3_dais/includes/6.tex}
% \pagebreak

\chapter{Datová vrstva informačního systému; existující API, rámce a implementace, bezpečnost; objektově-relační mapování.}
\input{3_dais/includes/7.tex}
% \pagebreak

\chapter{Distribuované SŘBD, fragmentace a replikace.}
\section{Deep learning}
\begin{itemize}
    \item Deep learning neboli \textbf{hluboké učení}, známé také jako hierarchické učení, je \textbf{sbírka algoritmů} používaných ve strojovém učení.
    \item Používají se k~modelování abstrakcí na vysoké úrovni v~datech za pomocí modelových architektur, které se skládají z~několika nelineárních transformací.
    \item Hluboké učení je součástí široké skupiny metod používané pro strojové učení, které jsou založeny na učení reprezentace dat.
\end{itemize}
Hluboké strukturované učení může být:
\begin{itemize}
    \item{\textbf{Kontrolované (s~učitelem)} - všechna data jsou kategorizovaná do tříd, algoritmy se učí předpovídat výstup ze vstupních dat.}
    \item{\textbf{Částečně kontrolované} - data jsou částečně kategorizovaná do tříd. Pří tomto přístupu učení lze využít kombinaci kontrolovaného a~nekontrolovaného přístupu učení.}
    \item{\textbf{Nekontrolované (bez učitele)} - data nejsou kategorizovaná do tříd, algoritmy se učí ze struktury vstupních dat.}
\end{itemize}
Hluboké učení je specifický přístup, použitý k~budování a~učení neuronových sítí, které jsou považovány za velmi spolehlivé rozhodovací uzly. Jestliže vstupní data algoritmu procházejí řadou nelinearit a~nelineárních transformací, tak tento algoritmus je považován za \uv{deep} algoritmus.

Odstraňuje také ruční identifikaci příznaků (obrázek \ref{fig:ml_vs_ann}) z~dat a~místo toho se spoléhá na jakýkoliv trénovací proces, které má za úkol zjistit užitečné vzory ve vstupních příkladech. To dělá neuronovou síť jednodušší a~rychlejší, a~může přinést lepší výsledky než z~oblasti umělé inteligence.

\begin{figure}[H]
    \centering
    \includegraphics[width=.5\linewidth]{assets/9_ml_vs_ann}
    \caption{Hlavním rozdílem mezi strojovým a~hlubokým učením je ten, že u~strojového se příznaky musí extrahovat manuálně.}
    \label{fig:ml_vs_ann}
\end{figure}

\subsection{Konvoluční neuronové sítě \textit{CNN} - Convolution neural network}
\begin{itemize}
    \item Speciálním druhem vícevrstvých neuronových sítí a~jsou navrženy tak, aby rozpoznaly vizuální vzory přímo z~pixelu obrazu s~minimálním předzpracováním.
    \item Mohou rozpoznat vzory s~extrémní variabilitou (například ručně psané znaky) a~odolnost vůči deformacím a~jednoduchým geometrickým transformacím.
    \item Síť využívá matematickou operaci zvanou konvoluce alespoň v~jedné jejich vrstvě.
\end{itemize}

Nejznámější a~nejvíce používanou konvoluční neuronovou sítí jsou modely LeNet.
Hlavní kroky LeNet sítě jsou:
\begin{itemize}
    \item{\textbf{Konvoluce} - tyto vrstvy provádějí konvoluci nad vstupy do neuronové sítě.}
    \item{\textbf{Nelinearita (ReLU)} - tato vrstva je použita po každé konvoluční vrstvě a~jejím cílem je nahrazení všech negativních pixelů nulou ve výstupu této vrstvy (příznaková mapa).}
    \item{\textbf{Pooling/sub sampling} - ze vstupního obrazu vyextrahuje pouze zajímavé části pomocí některých matematických operací (max, avg, sum), a~tím se \textbf{redukuje jeho dimenzionalita}.}
    \item{\textbf{Fully connected layer/klasifikace} - tato vrstva vychází z~původních umělých neuronových sítí, konkrétně z~vícevrstvého perceptronu. Tato vrstva je typicky umístěna na konci sítě a~je propojena s~klasifikační vrstvou pro predikci.}
\end{itemize}
\begin{figure}[H]
    \centering
    \includegraphics[width=.9\linewidth]{assets/9_cnn.pdf}
    \caption{Řetězec LeNet konvoluční neuronové sítě}
    \label{fig:cnn}
\end{figure}
% \pagebreak

\part{Počítače a sítě}
\chapter{Architektura univerzálních procesorů. Principy urychlování činnosti procesorů. }
% Systémy barev v počítačové grafice, nelinearita grafického výstupu (gamma korekce), kompozice rastrových obrazů (alfa kanál), HDR.
\section{Systémy barev v PG}
\begin{itemize}
    \item Základem barevného prostoru je \textbf{barevný model}, který nám dává abstraktní matematický popis, jak lze barvy vyjádřit pomocí n-tic čísel, nejčastěji trojic.
    \item Mezi nejznámější barevné modely v dnešní době patří \textbf{RGB model}.
    \item Model RGB pracuje se třemi základními barvami: \textbf{červenou, zelenou} a \textbf{modrou}, z nichž se odvíjí i jeho název.
    \item Tyto barvy byly zvoleny na základně toho, jak \textbf{čípky v lidském oku} vnímají jednotlivé záření.
    \item Zároveň je RGB \textbf{aditivní barevný model}, což znamená, že se jednotlivé barevné složky \textbf{míchají} (nové barvy získáváme přidáváním větší intenzity jednotlivých složek) a výsledkem jsou další barevné odstíny, případně vyšší intenzita barvy.
    \item Když k tomuto modelu definujeme, jak mají být tyto n-tice interpretovány, dostáváme \textbf{barevný prostor} -- je předem definovaná množina barev, kterou je schopno určité zařízení snímat, zobrazit nebo reprodukovat.
    \item Barevný prostor je tedy \textbf{definován rozsahem barev}, které dokáže zobrazit.
    \item Tomuto rozsahu se také říká \textbf{gamut}. Ten se zpravidla zobrazuje jako oblast v CIE 1931 chromatickém diagramu
\end{itemize}
\begin{figure}[H]
    \centering
    \includegraphics[width=0.3\textwidth]{assets/1_rgb_gamut}
\end{figure}
\subsection{RGB}
\begin{itemize}
    \item Nejrozšířenější barevný prostor postavený na RGB barevném modelu je \textbf{sRGB} - standardní RGB.
    \item Jeho určení je pro zobrazování \textbf{na monitorech} nebo \textbf{kódování barev} na internetu.
    \item Pro všechny tři barevné složky má definované barvy v \textbf{chromatickém diagramu}, které vymezují jeho gamut.
    \item Každá barva, kterou tento prostor zobrazuje, je dána zastoupením jednotlivých barevných složek, buďto relativně (hodnoty jsou v rozmezí 0 - 1) nebo absolutně (konkrétní \uv{bitové} hodnoty, zpravidla 0 - 255, 24-bitů).
    \item RGB je možné zobrazit jako krychli.
    \item Často se přidává \textbf{Alpha kanál} pro průhlednost - \textbf{RGBA} (32-bitů).
          \begin{figure}[H]
              \centering
              \includegraphics[width=0.6\textwidth]{assets/1_rgb_gamut_krychle}
          \end{figure}
\end{itemize}

\subsection{HSV a HSL}
\begin{itemize}
    \item \textbf{Hue, Saturation, Value/Lightness} - barevný model, který nejvíce odpovídá lidskému vnímání barev.
    \item Barvy popisuje pomocí 3 hodnot, které však samy barvy nereprezentují:
          \begin{itemize}
              \item \textbf{Hue} - \textbf{barevný tón}, převládající. Neboli \textbf{odstín} - barva \textbf{odražená} nebo \textbf{procházející} objektem. Měří se jako poloha na standardním barevném kole (\ang{0} až \ang{360}). Obecně se odstín označuje názvem barvy. \ang{0} - červená, \ang{120} - zelená, \ang{240} - modrá.
              \item \textbf{Saturation} - \textbf{sytost} barvy, příměs jiné barvy. Někdy též chroma, síla nebo čistota barvy, představuje množství šedi v poměru k odstínu, měří se v procentech od 0\% (šedá) do 100\% (plně sytá barva). Na barevném kole vzrůstá sytost od středu k okrajům.
              \item \textbf{Value} - \textbf{hodnota jasu}, množství bílého světla. Relativní světlost nebo tmavost barvy. Jas vyjadřuje \textbf{kolik světla barva odráží}, dalo by se také říct přidávání černé do základní barvy.
          \end{itemize}
    \item Nejčastěji se tato reprezentace (popř. \textbf{HSL}) používají v grafických nástrojích jako komponenty pro výběr barvy, protože je mnohem intuitivnější než RGB.
    \item Vyberu si odstín, jak má být sytý a jasný a hotovo. Není třeba řešit jak smíchat 3 barevné složky, abych dostal to co chci.
    \item Dále se využívá často v případě detekce objektů, kdy hodnota HUE (odstín), je nezávislý na osvětlení scény. Problém však nastává u bílých a černých objektů (kdy HUE může být různé), ty lze na základě value a staturation mapovat do podobných barev (žlutá a černá).
    \item Mimo níže uvedené zobrazení \textbf{válcem}, lze také zobrazit \textbf{kuželem} a
          \begin{figure}[H]
              \centering
              \includegraphics[width=0.9\textwidth]{assets/1_hsv_hsl}
          \end{figure}
\end{itemize}

\subsection{CMY a CMYK}
\begin{itemize}
    \item Substraktivní barevné systémy (barvy se ,,\textbf{odečítají}'' od bílé, přidáváním jednotlivých složek až po černou), \textbf{C}yan, \textbf{M}agenta, \textbf{Y}ellow a \textbf{K}ey (Blac\textbf{K}).
    \item Používá se \textbf{pro tisk}.
    \item Černá se přidala, protože smíchání CMY nedává plně černou barvu, navíc je černý inkoust levnější než barevný.
    \item Nevýhodou je, že \textbf{nedokáže správně zobrazit} sytě červenou, zelenou a modrou.
    \item Při tisku to však není poznat.
    \item Před tiskem se RGB obraz převádí do CMYK.
    \item To provádí buďto ovladač tiskárny nebo RIP (Raster Image Processor - u profi tiskáren).
    \item RGB se používá pro aktivní zdroje světla, CMYK jsou \textbf{pasivní} (světlo pouze \textbf{odrážejí}), proto nedokáží udělat tak jasné odstíny.
\end{itemize}
\begin{figure}[H]
    \centering
    \includegraphics[width=0.6\textwidth]{assets/1_cmyk}
\end{figure}

\subsection{YCbCr}
\begin{itemize}
    \item Barva je reprezentována \textbf{jasovou složkou} Y a modrou a červenou \textbf{chrominanční} komponentou.
    \item Není to absolutní barevný model, jedná se o \textbf{způsob kódování RGB} informací.
    \item Využívá se nejčastěji u videa a barevných obrázků, kde je využito faktu, že \textbf{lidské oko nejvíce vnímá jas}, který je reprezentovaný složkou Y. Barvy už tak důležité nejsou a proto se můžou například více \textbf{komprimovat} bez výraznější ztráty kvality obrazu (JPEG).
    \item Jasová složka je kódována v intervalu $\langle0, 1\rangle$ a chrominanční složky v intervalu $\langle-0.5, 0.5\rangle$
\end{itemize}

\section{Nelinearita grafického výstupu (gamma korekce)}
První CRT monitory zobrazovaly jas nelineárně. To znamená, že dvojnásobné napětí neznamená dvojnásobný jas, křivka jasu byla zhruba exponenciální. Tento způsob zobrazení jasu přetrvává i v dnešních monitorech. Proto je potřeba upravit i zobrazované barvy. Mapování barev je potřeba provést nelineárně. Nelineární vstup v kombinaci s exponenciální křivkou jasu ve výsledku vede k k jasu, který je vnímán jako lineární. 
\begin{figure}[H]
    \centering
    \includegraphics[width=0.3\textwidth]{assets/1_gamma_correction_gamma_curves.png}
\end{figure}
Problém nastává při práci s barvami. Pracovat s barvami je potřeba v lineárním prostoru, aby byly výsledky např. matematických operací korektní. Před zpracováním je tedy nutné převést barvu do lineárního prostou, po zpracování je potřeba před zobrazením opět převézt barvu zpět do nelineárního prostoru.

\begin{equation}
    C_{linear} = C_{sRGB}^{\text{gamma}}
\end{equation}

\begin{equation}
    C_{sRGB} = C_{linear}^{\frac{1}{\text{gamma}}}
\end{equation}

\section{Kompozice rastrových obrazů (alfa kanál)}
Alfa kanál v obrazech se používá v počítačové grafice pro průhlednost obrazů. Obrazy mohou obsahovat plnou nebo částečnou průhlednost. Plně transparentní obraz propouští veškeré barvy podkladu, částečně transparentní obraz mixuje část bravy podkladu a část vlastní barvy. Např. výsledná barva obrazu s hodnotou alfa 0.5 vytvoří mix barev tvořený z $50\%$ barvou podkladu a z $50\%$ barvou transparentního objektu. V případě překryvu více transparentních objektů je potřeba objekty setřídit dle vzdálenosti a vypočítat transparentnost postupně.   

\section{HDR}
Barvy jsou tradičně reprezentovány trojicí 8bitových hodnot v intervalu  $\langle-0, 255\rangle$. Takto popsaný obraz ale může ztrácet detaily z důvodu limitovaných možností hodnot pro rozložení barev. Přesnější metodou je ukládání jednotlivých složek jako 16 nebo 32bitových hodnot, pomocí desetiných čísel. Zde ale nastává problém s hodnotami většími než 1. Je teda nutné provést tzv. tone mapping. Proces tone mapping v zásadě převádí hodnoty do intervalu $\langle0, 1\rangle$. \par
Jednou z nejjednodušších metod je Reinhardova metoda, hodnota se vypočítá následujícím vzorcem: 
\begin{equation}
    C_{out} = \frac{C_{in}}{C_{in} + 1}    
\end{equation}
Tato metoda zachová poměrně dobře kontrast pro oblasti obrazu s nízkým jasem, oblasti obrazu s vysokým jasem jsou ale méně kontrastní. 
Pokročilejší metodou je mapování hodnot pomocí expozice, následujícím vzorcem:
\begin{equation}
    C_{out} = 1 - e^{C_{in} * \exp}
\end{equation}
Pomocí hodnoty expozice je pak možné nastavit celkové podání barev obrazu. Hodnota expozice by měla být zvolena v závislosti na aktuálním vstupu, je možné hodnotu expozice volit automaticky. 
% \pagebreak

\chapter{Základní vlastnosti monolitických počítačů a jejich typické integrované periférie. Možnosti použití. }
\input{4_ps/includes/2.tex}
% \pagebreak

\chapter{Protokolová rodina TCP/IP. }
\input{4_ps/includes/4.tex}
% \pagebreak

\chapter{Metody sdíleného přístupu ke společnému kanálu. }
\input{4_ps/includes/5.tex}
% \pagebreak

\chapter{Problémy směrování v počítačových sítích. Adresování v IP, překlad adres (NAT). }
\input{4_ps/includes/6.tex}
% \pagebreak

\chapter{Bezpečnost počítačových sítí s TCP/IP: útoky, paketové filtry, stavový firewall. Šifrování a autentizace, virtuální privátní sítě.  }
\input{4_ps/includes/7.tex}
% \pagebreak

\end{document}