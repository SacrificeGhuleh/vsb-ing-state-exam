\documentclass[openany]{book}
\usepackage[czech]{babel}

\usepackage[fit, breakall]{truncate}
\usepackage{fancyhdr}

% \usepackage[utf8]{inputenc}
\usepackage[useregional]{datetime2}
%\usepackage[T1]{fontenc}
\usepackage[a4paper, top=1.6cm, bottom=1.6cm, left=2cm, right=2cm]{geometry}
\usepackage[thinlines]{easytable}
\usepackage{graphicx}
\usepackage[ampersand]{easylist}
% \usepackage{changepage}
\usepackage{float}
\usepackage{color}
\usepackage{xcolor}
\usepackage{siunitx}
\usepackage{enumitem}
\usepackage{latexsym}
\usepackage[unicode=true]{hyperref}
\usepackage{amssymb}
\usepackage{amsmath}
\usepackage{amsfonts}
\usepackage{minted}
\usepackage{caption}
\usepackage{makecell}
\usepackage{tikz}
\usepackage{multirow}
\usepackage{sectsty}
\usepackage{xevlna}

\raggedbottom

\chaptertitlefont{\Large}

\graphicspath{{./1_it_mpzz/}{./2_swi/}{./3_dais/}{./4_ps/}{./5_pgo/}}

%\renewcommand{\baselinestretch}{1.2} 
%\setitemize{itemsep=0pt}
%\setenumerate{itemsep=0pt}

\newcommand{\hhline}{\Xhline{2\arrayrulewidth}}
\newcommand{\resline}{\vspace{1.5mm}\hrule\vspace{1.5mm}}
\setcounter{tocdepth}{0}

\title{Otázky k SZZ\\ING 2021}
\author{Richard Zvonek}

\fancypagestyle{plain}{
\fancyhf{}
\fancyhead[LE, RO]{\tiny \truncate{0.9 \textwidth} \leftmark}
%\fancyhead[RE, LO]{\tiny \truncate{0.47 \textwidth} \rightmark}
\fancyfoot[LE,RO]{\thepage}
\renewcommand{\headrulewidth}{0pt}%
\renewcommand{\footrulewidth}{0.4pt}}


\pagestyle{fancy}
\fancyhf{}
\fancyhead[LE, RO]{\tiny \truncate{0.47 \textwidth} \leftmark}
\fancyhead[RE, LO]{\tiny \truncate{0.47 \textwidth} \rightmark}
\fancyfoot[LE,RO]{\thepage}

\makeatletter
\renewcommand{\@makechapterhead}[1]{%
  {\noindent\raggedright\normalfont% Alignment and font reset
   \Large\bfseries\thechapter~~#1\par\nobreak}% Formatting
  \vspace{\baselineskip}% ...just a little space
}


\begin{document}
\tableofcontents

\chapter{Počítačová grafika a analýza obrazu}
\section{Systémy barev v počítačové grafice, nelinearita grafického výstupu (gamma korekce), kompozice rastrových obrazů (alfa kanál), HDR.}

\section{Afinní a projektivní prostor. Afinní a projektivní transformace a jejich matematický zápis. Modelovací a zobrazovací transformace v počítačové grafice.}

\section{Metody získávání fotorealistických obrazů, rekurzivní sledování paprsku, radiometrie, zobrazovací rovnice, Monte Carlo přístupy ve výpočtu osvětlení, urychlovací metody.}

\section{Standardní zobrazovací řetězec a realizace jeho jednotlivých kroků, modely osvětlení a stínovací algoritmy, řešení viditelnosti, možnosti výpočtu globálního osvětlení v reálném čase, stručná charakteristika standardu OpenGL.}

\section{Komprese obrazu a videa, principy úprav obrazu v prostorové a frekvenční doméně.}

\section{Základní metody úpravy a segmentace obrazu (filtrace, prahování, hrany, oblasti, rohy).}

\section{Základní metody rozpoznávání objektů, příznakové rozpoznávání. Univerzální příznaky pro rozpoznávání (např. HOG), trénovací klasifikátory (např. SVM).}

\section{Hluboké neuronové sítě (např. konvoluční, popis jednotlivých vrstev).}

\section{Rekonstrukce 3D objektů z 2D obrazů (základní principy).}

\chapter{Matematické základy informatiky}

\section{Konečné automaty, regulární výrazy, uzávěrové vlastnosti třídy regulárních jazyků.}
\subsection{Konečný automat}
\begin{itemize}
    \item Výpočetní model, přijímá nebo zamítá slova reg. jazyka
    \item Znázornění tabulkou, grafem, stromem
    \item Deterministický
    \begin{itemize}
        \item Definován uspořádanou pěticí
        \begin{itemize}
            \item Množina stavů
            \item Množina vstupních symbolů - abeceda
            \item Přechodová funkce 
            \begin{itemize}
                \item Vrací jeden stav
            \end{itemize}
            \item Počáteční stav
            \item Množina koncových (přijímajících) stavů 
        \end{itemize}
    \end{itemize}
    \item Nedeterministický
    \begin{itemize}
        \item Definován uspořádanou pěticí
        \begin{itemize}
            \item Množina stavů
            \item Množina vstupních symbolů - abeceda
            \item Přechodová funkce
            \begin{itemize}
                \item Vrací množinu stavů
            \end{itemize}
            \item Množina počátečních stavů
            \item Množina koncových (přijímajících) stavů 
        \end{itemize}
        \item Na rozdíl od deterministického
        \begin{itemize}
            \item Z jednoho stavu libovolný počet přechodů se stejným symbolem
            \item Nemusí ošetřovat všechny varianty
            \item Slovo přijímá, pokud existuje alespoň jeden výpočet vedoucí k přijetí
            \item Může mít více než jeden počáteční stav
            \item Lze ho převést na deterministický (až \(2^n\) stavů v převedeném)
        \end{itemize}
    \end{itemize}
    \item Zobecněný nedeterministický
    \begin{itemize}
        \item Nedeterministický \(+\) nulové \(\epsilon\) (epsilon) přechody
    \end{itemize} 
\end{itemize}

\subsection{Regulární výraz}
\begin{itemize}
    \item Řetězec, popisující celou množinu řetězců (regulární jazyk)
    \item Popisuje také konečný automat
    \item Znaky abecedy, operace sjednocení, zřetězení a iterace.
    \item Obsahuje prázdné slovo \(\epsilon\) a prázdný jazyk \(\emptyset\)
\end{itemize}

\subsection{Uzávěrové vlastnosti třídy regulárních jazyků}
\begin{itemize}
    \item Uzavřenost -- Výsledek operace s prvky množiny bude opět spadat do dané množiny
    \item Regulární výrazy uzavřené vůči
    \begin{itemize}
        \item Sjednocení, průnik, doplněk
        \item Zřetězení, iterace
        \item Zrcadlový obraz
    \end{itemize}
\end{itemize}

\section{Bezkontextové gramatiky a jazyky. Zásobníkové automaty, jejich vztah k bezkontextovým gramatikám.}

\subsection{Bezkontextové gramatiky}
\begin{itemize}
    \item Definováno jako čtveřice
    \begin{itemize}
        \item Množina neterminálů
        \begin{itemize}
            \item Neterminály = proměnné
        \end{itemize}
        \item Množina terminálů
        \begin{itemize}
            \item Terminály = Konstanty
        \end{itemize}
        \item Počáteční neterminál
        \item Přepisovací pravidla
        \begin{itemize}
            \item Definují přepisování neterminálů
            \item Když není co přepisovat \(\rightarrow\) slovo, tvořené pouze neterminály
        \end{itemize}
    \end{itemize} 
    \item Přijímáno zásobníkovým automatem
    \item Uzavřeno na sjednocení, zřetězení, iteraci a zrcadlový obraz
    \item Derivace slova -- Konkrétní odvození slova z gramatiky
    \begin{itemize}
        \item Levá -- Přepisují se neterminály zleva
        \item Pravá -- Přepisují se neterminály zprava
    \end{itemize}
    \item Derivační strom -- Grafické znázornění derivace slova stromem
    \item Chomského normální forma
    \begin{itemize}
        \item Neterminál \(\rightarrow\) neterminál \(\left( A \rightarrow BC \right))\)
        \item Neterminál \(\rightarrow\) terminál \(\left( A \rightarrow ab \right))\)
        \item Neterminál \(\rightarrow\) epsilon \(\left( A \rightarrow \epsilon \right))\)
    \end{itemize}
    \item Nevypouštějící gramatika
    \begin{itemize}
        \item Neobsahuje epsilon
    \end{itemize}
\end{itemize}

\subsection{Zásobníkový automat}
\begin{itemize}
    \item V podstatě nedeterministický konečný automat rozšířený o zásobník
    \item Díky zásobníku si pamatuje, kolik a jaké znaky přečetl
    \item Slovo přijme, pokud přečetl celé slovo a zásobní je prázdný
\end{itemize}

\section{Matematické modely algoritmů -Turingovy stroje a stroje RAM. Složitost algoritmu, asymptotické odhady. Algoritmicky nerozhodnutelné problémy.}

\begin{itemize}
    \item Snaha popsat libovolný algoritmus \(\rightarrow\) Turingův a RAM stroj
    \item Teoretické modely univerzálních počítačů, programovacích jazyků
\end{itemize}

\subsection{Turingův stroj}
\begin{itemize}
    \item Oboustranně nekonečná páska obsahující vstupní slovo
    \item Čtení i zápis pomocí hlavy
    \item Pohyb po pásce oběma směry
    \item Pracuje s celou abecedou
    \item Turingovská úplnost
    \begin{itemize}
        \item Stroj, počítač, programovací jazyk, úloha\dots
        \item Lze v něm odsimulovat libovolný jiný Turingův stroj
        \item Má stejnou výpočetní sílu jako Turingův stroj
    \end{itemize}
    \item Church-Turingova teze
    \begin{itemize}
        \item Ke každému algoritmu existuje ekvivalentní Turingův stroj
    \end{itemize}
\end{itemize}

\subsection{RAM stroj}
\begin{itemize}
    \item Model vycházející ze skutečných počítačů
    \item Slouží zejména k analýze algoritmů
    \begin{itemize}
        \item Paměťová, časová složitost
    \end{itemize}
    \item Skládá se z:
    \begin{itemize}
        \item Programová jednotka
        \begin{itemize}
            \item Konečná posloupnost instrukcí \(\rightarrow\) program
        \end{itemize}
        \item Pracovní paměť
        \begin{itemize}
            \item Uspořádaná paměť
            \item Zápis celých čísel
            \item Indexace přirozenými čísly
            \begin{itemize}
                \item 0 \(\rightarrow\) Pracovní registr
                \item 1 \(\rightarrow\) Indexový registr
            \end{itemize}
        \end{itemize}
        \item Vstupní a výstupní páska
        \begin{itemize}
            \item Sekvenční zápis/čtení celých čísel
        \end{itemize}
        \item Centrální jednotka
        \begin{itemize}
            \item Programový registr
            \item Ukazatel na instrukci
            \item Inkrementace ukazatelu po provedení instrukce
            \begin{itemize}
                \item O 1 nebo o více v případě skoku
            \end{itemize}
        \end{itemize}
    \end{itemize}
\end{itemize}
\subsection{Složitost algoritmů}
\begin{itemize}
    \item Pro srovnávání algoritmů řešících stejný problém
    \item Obecně, čím nižší, tím lepší
    \item Časová složitost
    \begin{itemize}
        \item Jak závisí doba vykonávání na množství vstupních dat
    \end{itemize}
    \item Prostorová složitost
    \begin{itemize}
        \item Jak závisí potřebná paměť na množství vstupních dat
    \end{itemize}
    \item Neudávají se konkrétní čísla, udává se funkce závislá na vstupních datech
    \begin{itemize}
        \item Získáno simulací na RAM stroji
        \item Počítání instrukcí pro nejhorší možný vstup
    \end{itemize}
    \item Používá se asymptotická notace
    \begin{itemize}
        \item \(\mathcal{O}\left( f \right)\) --  Horní ohraničení -- Roste nejvýše tak rychle jako \(f\)
        \item \(\Theta\left( f \right)\) -- Roste stejně rychle jako  \(f\)
        \item \(\Omega\left( f \right)\) -- Dolní ohraničení -- Roste rychleji než \(f\)
        \item \(\mathcal{!}\) Zanedbávají se konstanty
        \item \(\mathcal{!}\) Analyzuje se nejhorší případ
        \begin{itemize}
            \item Relevantní může být spíše typický příklad
        \end{itemize}
    \end{itemize}
\end{itemize}
\subsection{Rozhodnutelnost problémů}
\begin{itemize}
    \item Problém je rozhodnutelný, pokud existuje turingův stroj, který jej řeší
    \item Problém není rozhodnutelný, pokud pro daný vstup nenalezne žádný algoritmus výstup
    \item Ano/Ne problémy
    \begin{itemize}
        \item Výstup z algoritmu může být Ano (true) nebo Ne (false)
        \item Dají se na něj převést všechny probélmy, pokud nepotřebujeme přesný výsledek
    \end{itemize}
    \item Riceova věta
    \begin{itemize}
        \item Nerozhodnutelná vlastnost, pokud je netriviální a vstupně/výstupní
        \item Triviální vlastnost
        \begin{itemize}
            \item Mají ji všechny programy nebo ji nemá žádný program 
            \item Triviální vlastnost je vždy vstupně/výstupní
        \end{itemize}
        \item Vstupně/výstupní vlastnost
        \begin{itemize}
            \item Pokud dva programy se stejnou IO tabulkou vlastnost mají nebo ne 
        \end{itemize}
    \end{itemize}
    \item Částečná rozhodnutelnost
    \begin{itemize}
        \item Pro očekávaný výstup ANO vrátí ANO
        \item Pro očekávaný výstup NE vrátí NE nebo se program nezastaví
    \end{itemize}
\end{itemize}

\subsection{Algoritmicky nerozhodnutelné problémy}
\begin{itemize}
    \item Halting problem
    \item Ekvivalence bezkontextových gramatik
    \item Nejednoznačnost bezkontextové gramatiky
\end{itemize}

\section{Třídy složitosti problémů. Třída PTIME a NPTIME, NP-úplné problémy.}
\subsection{Třída PTIME}
\begin{itemize}
    \item Rozhodovací (Ano/Ne) problémy
    \item Problémy, které lze řešit v polynomiálním čase
    \item Problémy v PTIME jsou prakticky zvládnutelné
\end{itemize}
\subsection{Třída NPTIME}
\begin{itemize}
    \item Rozhodovací (Ano/Ne) problémy
    \item Problémy, které lze řešit v polynomiálním čase \textbf{nedeterministickými algoritmy}
    \begin{itemize}
        \item Pro Ano stačí najít jedno řešení
        \item Pro Ne je třeba důkaz, že žádné řešení nevrací Ano
    \end{itemize}
\end{itemize}
\subsection{NP-těžký problém}
\begin{itemize}
    \item Problém, na který lze převést každý problém z NPTIME v polynomiálním čase 
\end{itemize}
\subsection{NP-úplné problémy}
\begin{itemize}
    \item Problémy, patřící do NPTIME a zároveň jsou NP-těžké
\end{itemize}


\section{Jazyk predikátové logiky prvního řádu. Práce s kvantifikátory a ekvivalentní transformace formulí.}

\section{Pojem relace, operace s relacemi, vlastnosti relací. Typy binárních relací. Relace ekvivalence a relace uspořádání.}

\section{Pojem operace a obecný pojem algebra. Algebry s jednou a dvěma binárními operacemi.}

\section{FCA – formální kontext, formální koncept, konceptuální svazy.}

\section{Asociační pravidla, hledání často se opakujících množin položek.}

\section{Metrické a topologické prostory – metriky a podobnosti.}

\section{Shlukování}

\section{Náhodná veličina. Základní typy náhodných veličin. Funkce určující rozdělení náhodných veličin.}

\section{Vybraná rozdělení diskrétní a spojité náhodné veličiny - binomické, hypergeometrické, negativně binomické, Poissonovo, exponenciální, Weibullovo, normální rozdělení.}

\section{Popisná statistika. Číselné charakteristiky a vizualizace kategoriálních a kvantitativních proměnných.}

\section{Metody statistické indukce. Intervalové odhady. Princip testování hypotéz.}

\chapter{Softwarové inženýrství}

\section{Softwarový proces. Jeho definice, modely a úrovně vyspělosti.}

\section{Vymezení fáze „sběr a analýza požadavků“. Diagramy UML využité v dané fázi.}

\section{Vymezení fáze „Návrh“. Diagramy UML využité v dané fázi. Návrhové vzory – členění, popis a příklady. }

\section{Objektově orientované paradigma. Pojmy třída, objekt, rozhraní. Základní vlastnosti objektu a vztah ke třídě. Základní vztahy mezi třídami a rozhraními. Třídní vs. instanční vlastnosti.}

\section{Mapování UML diagramů na zdrojový kód.}

\section{Správa paměti(v jazycích C/C++, Java , C\#, Python), virtuální stroj, podpora paralelního zpracování a vlákna.}

\section{Zpracování chyb v moderních programovacích jazycích, princip datových proudů – pro vstup a výstup. Rozdíl mezi znakově a bytově orientovanými datovými proudy.}

\section{Jazyk UML – typy diagramů a jejich využití v rámci vývoje.}

\chapter{Databázové a informační systémy}

\section{Modelování databázových systémů, konceptuální modelování, datová analýza, funkční analýza; nástroje a modely. }

\section{Relační datový model, SQL; funkční závislosti, dekompozice a normální formy.}

\section{Transakce, zotavení, log, ACID, operace COMMIT a ROLLBACK; problémy souběhu, řízení souběhu: zamykání, úroveň izolacev SQL.}

\section{Procedurální rozšíření SQL, PL/SQL, T-SQL, triggery, funkce, procedury, kurzory, hromadné operace.}

\section{Základní fyzická implementace databázových systémů: tabulky a indexy; plán vykonávání dotazů.}
\begin{itemize}
    \item Tabulkuy
    \item Indexy
    \item Materializované pohledy
    \item Rozdělení dat
\end{itemize}
\subsection{Tabulky}
\begin{itemize}
    \item Heap Table
    \begin{itemize}
        \item Implicitní
        \item Záznamy neuspořádány
        \item Rychlé vkládání, pomalé hledání
        \item Záznamy se označí jako smazané, fyzicky se mažou operací shrinking
    \end{itemize}
    \item B-Strom
    \begin{itemize}
        \item Pomalejší vkládání (je třeba třídit), rychlé vyhledávání
        \item Je možné určit, co bude ve stromu a co na haldě
    \end{itemize}
    \item Hash table
    \begin{itemize}
        \item Záznamy s podobnou hash hodnotou uloženy u sebe
        \item Plýtvá místem
        \item Je třeba znát přibližnou velikost dat
    \end{itemize}
\end{itemize}

\subsection{Indexy}
\begin{itemize}
    \item Umožňuje uložení klíčů a ROWID, rychlejší přístup k datům
\end{itemize}
\subsection{Plán vykonávání dotazů}
\begin{itemize}
    \item 3 fáze
    \begin{enumerate}
        \item Převod dotazu do interní formy
        \item Převod do kanonické formy
        \begin{itemize}
            \item Optimalizace dotazu
        \end{itemize}
        \item Vygenerování plánů dotazů a výběr nejlevnějšího plánu
        \begin{itemize}
            \item Generování jednotlivých variant, ohodnocení, výběr nejlepší varianty
        \end{itemize}
    \end{enumerate}
\end{itemize}

\section{Objektově‐relační datový model a XML datový model: principy, dotazovací jazyky.}

\section{Datová vrstva informačního systému; existující API, rámce a implementace, bezpečnost; objektově-relační mapování.}

\section{Distribuované SŘBD, fragmentace a replikace.}

\chapter{Počítače a sítě}
\section{Architektura univerzálních procesorů. Principy urychlování činnosti procesorů. }

\section{Základní vlastnosti monolitických počítačů a jejich typické integrované periférie. Možnosti použití. }

\section{Protokolová rodina TCP/IP. }

\section{Metody sdíleného přístupu ke společnému kanálu. }

\section{Problémy směrování v počítačových sítích. Adresování v IP, překlad adres (NAT). }

\section{Bezpečnost počítačových sítí s TCP/IP: útoky, paketové filtry, stavový firewall. Šifrování a autentizace, virtuální privátní sítě.  }

\end{document}